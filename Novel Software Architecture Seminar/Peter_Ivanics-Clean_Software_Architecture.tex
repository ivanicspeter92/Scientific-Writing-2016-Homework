\documentclass[conference]{IEEEtran}

\usepackage[utf8]{inputenc}
\usepackage{cite}
\ifCLASSINFOpdf
  \usepackage[pdftex]{graphicx}
\else
  \usepackage[dvips]{graphicx}
\fi
\usepackage{amsmath}
\usepackage{array}
\usepackage{url}

% correct bad hyphenation here
% \hyphenation{op-tical net-works semi-conduc-tor}

\begin{document}
% paper title
% Titles are generally capitalized except for words such as a, an, and, as,
% at, but, by, for, in, nor, of, on, or, the, to and up, which are usually
% not capitalized unless they are the first or last word of the title.
% Linebreaks \\ can be used within to get better formatting as desired.
% Do not put math or special symbols in the title.
\title{Clean Software Architecture}

\author{\IEEEauthorblockN{Péter Ivanics}
\IEEEauthorblockA{Department of Computer Science \\
University of Helsinki \\
Email: peter.ivanics@helsinki.fi \\
\url{http://pivanics.users.cs.helsinki.fi/portfolio/}}}

\maketitle

% As a general rule, do not put math, special symbols or citations
% in the abstract
\begin{abstract}
%<Context>
The construction and the long-term maintenance of robust software architecture is a demanding task. Software architects often face the problem to ensure the extensibility of systems so that new components can be added and existing ones can be removed or replaced easily. Linearly with software growth, these aspects of software engineering are getting more and more difficult because often times the separation of concerns and the decoupling of components are not considered.  

%<Objectives>
The Clean Architecture concept was introduced recently by professional software craftsmen in order to provide principles on how to achieve robustness in software architecture engineering. This approach builds on top of the clean separation of concerns, finding the right abstractions on every level of the software projects and the minimal creation of internal dependencies.

%<Methods and objectives>
The objective of this study is to review relevant professional literature and summarize the most important aspects of Clean Architecture. The paper describes the benefits compared to other, previously suggested practices. It is explained how this approach facilitates and why is it well-suited for Object Oriented principles and software projects of the present time. 

%<Results>
Results show that understanding the key concepts, such as The Dependency Rule and the four layers of Clean Architecture, help developers to maintain robustness in their software. Ultimately, the concepts explained by this approach are independent from technologies and therefore easy to apply in every application field of our industry. 
\end{abstract}

% no keywords
% \keywords{}

\section{Introduction}
% what is the context?
The construction and the maintenance of software is a difficult task from many aspects. Software development is a creative act of action where developers build intangible means from nothing \cite{cleancoder}. Linearly with the growth of the code base of a software project, its complexity and cost of change increases \cite{codecomplete} \cite{cleancode}, which makes developers' job even more difficult. 

% why talking about this is important? 
Software assets can be crucial for operations in industrial, business and research fields \cite{cleancode} \cite{cleancoder}. In the present time, more and more companies are established around providing software services to their customers \cite{cusumano2008changing}, and therefore software components can become business-critical. For this reason, the process of software development requires dedicated professionals with a lot of discipline, humility, dedication and commitment to their job \cite{cleancode}. 

% what can be the result of not paying attention to this and what are the ways to go?
Despite all the previously established standards and practices, developers often face troubles of conforming to the standards and understanding code written by others and even themselves \cite{cleancoder}. Keeping the code base maintainable, easy to understand and open for extensions is often a challenging task \cite{cleancoder}, but nevertheless essential to avoid paying the technical dept. For this reason, the concept of Clean Code \cite{cleancode} was established which further raised the attention towards maintainable software in the industry. Agile methods and principles became popular in order to facilitate developer productivity. 

% what is CC and what does it provide?
Clean Code is a concept introduced by Robert C. Martin \cite{cleancode} \cite{cleancoder}. This approach to software development aims for the delivery of elegant, easy to read, fully tested code base, in order to create robust, bug-free applications \cite{cleancode}. Additionally, the regular need for refactoring is emphasied as increases quality of software \cite{impactofrefactoring}, and therefore developers should not be afraid of performing this activity \cite{cleancoder}. 

% why and where is CC popular?
The Clean Code principles are getting more and more popular among wide range of software developers in the industry currently. On top of that, even universities consider to change their approach how to teach software development by introducing testing and maintainable code first to the students \cite{studentscleancode}. The reasons for its popularity are understandable: the principles are simple and clearly expressed, everyone who ever dealt with software development can relate to them and utilize the acquired knowledge right away. 

% how does this relate to softare architecture?
Software architectures establish the basis and create a good frame for quality software. During the last decades, many different approaches and design patterns were developed and proposed by professionals to facilitate robustness of programs \cite{codecomplete} \cite{onionarchitecture} \cite{gof}, which are widely used and appreciated by others worldwide. In order to ensure maintainable and robust software systems a loosely coupled, flexible and extensible frame should be provided as the basis, which is ready to accept continuously changing and new requirements. For this reason, an approach towards the Clean Software Architecture is emerging \cite{cleanarchitecture} building on top of Clean Code principles. 

% research questions of this paper
This paper aims to seek answers to the following research questions by discussing relevant literature in the topic: 

\begin{enumerate}
	\item what are the main aspects of the Clean principles in terms of software architecture?
	\item why such architecture is needed?
	\item how can any software project - independently from utilized technologies and platforms - benefit from the Clean Architecture principles?
\end{enumerate}  

% how is the rest of the paper structured? 
The rest of this paper is structured, as follows. The next section introduces... Chapter III focuses on...
 Chapter IV discusses... Finally, the last chapter concludes the findings of this research and establishes the directions for any further research. 
%TODO

\section{The need for a clean design}
% why is software development and extending the code base difficult?
Software development is collaborative work. More and more developers work in teams of various sizes on multiple technologies to develop their software day-by-day. Inevitably, components are developed simultaneously by multiple coders and getting bigger as by time. As time passes, developers tend to have difficulties understanding the code they have written, because they do not remember what the intention behind the lines were \cite{cleancoder}. Similarly, understanding decisions made and code written by fellow colleagues is even more difficult and challenging on every level of abstraction \cite{cleancoder}. As the code base grows, decisions that were previously made may need to be reconsidered and code may need to be refactored to enhance its reliability and to avoid technical dept. 

% how can intent-driven approach help to get over this?
For this reason, considering the intent behind a decision is key even on the lowest level of components, for instance classes, objects, functions or even variable names. The usage of intention-revealing names on all parts of software projects greatly enhance the readability of their code \cite{cleancode}. This enables developers to read and understand others' code more efficiently and work on code written by others with more confidence. 

Demonstrating the intent behind design decisions is a key characteristic of good design which helps developers to understand what the program code does and how it is structured. This is particularly important while debugging or refactoring detailed parts of the code. However, it is not limited to low-level components of the software. 

% why is it important to understand the intents on every level of abstracition? 
Naturally, a complex system is composed of multiple components on different levels. In order to understand the "big picture" of such system, one may want to look at it on the top level from different angles. If an "outsider" (i.e. a new developer or system architect to the software) looks at a complex system, they may find it inconvenient, hard to understand or confusing at first, if the components, connection points or relationships are not named properly. 

As soon as higher-level components hold the property of telling their role at once in the ecosystem, it becomes much easier to understand. Clean naming conventions, for all components and their relationship strongly enhance the understandability of the system for individuals. On top of that, intent-revealing components typically easy to adapt to the other key principles of clean design that are explained in the following chapter \cite{cleancode}.

\section{Principles}
% what is the next step after demonstrating intent?
Communicating the intent behind the components is a good start for a Clean Architecture. Nevertheless, there are some other principles which help developers to avoid tight coupling and complex connections between the objects. The aim should be an approach, where each component is standalone and are handled as plugins to other components \cite{cleancode} \cite{cleanarchitecture}. In other words, components should be independent, which leads to the first principle of Clean Architecture, namely, the onion-like separation of Architectural Layers \cite{cleanarchitecture}. 

\subsection{Architectural Layers}
% what is a typical problem of system architecture? 
Many of the commonly used system-level design patterns, such as Model-View-Controller (MVC) may be easy to understand and implement, but are also hard to scale up to bigger systems. Traditionally, this approach defines three main layers, namely user interface, business logic and database for the software. MVC is widely used in the industry due to its simplicity, however it carries the risk of tight coupling between the layers of the architecture as well as the difficult separation of concerns \cite{onionarchitecture}. 

\begin{figure}[!t]
\centering
\includegraphics[width=3in]{images/cleanarchitecture.jpg}
\caption{The layers of the Clean Architecture and their inward-pointing dependencies \cite{cleanarchitecture}.}
\label{fig_sim}
\end{figure}

% why MVC and similar patterns are not necesseraly scaleable? 
Tight coupling of the layers immediately leads to a hardly-scalable architecture unless the software is relatively small \cite{onionarchitecture}. As time passes, components grow bigger, at the same time technology develops and requirements change. Tightly coupled systems have hard time to follow-up on such changes, which would be inevitable in case of business-critical systems \cite{onionarchitecture}. Therefore, a flexible and scalable approach to architectural layers is needed, which is robust enough to support system growth. 

% what is the solution? what does the onion architecture?
The solution to achieve a flexible, largely scalable architecture lies within the definition of the correct layers and dependencies \cite{cleanarchitecture} \cite{onionarchitecture}. For instance, the Onion Architecture \cite{onionarchitecture} suggests to define the layers from inside-out, from core entities to to infrastructure. 

% what does the Onion Architecture suggest? 
The centermost layer, the Domain Model defines the domain specific business-entities. The responsibility of this layer is strictly limited to define entities and excludes any logic, which is delegated to the layers further away from the center. The Use Case/Domain Services layer around the models is responsible to implement the business logic upon the business objects. Interface adapters act as a facilitator between the business logic and the events happening in the user interface, external input sources and the database \cite{onionarchitecture} \cite{cleanarchitecture}. 

% why is this good? 
Decoupling of different roles and the separation of responsibilities are key aspects in this design. Every layer has one task to cover and each holds the property of being possible to remove and replace by another component that has similar behavior but different internal logic. This kind of isolation embraces layers to be plugins to eachother and therefore easy to modify, extend or even replace during future development.

\subsection{Separation of concerns}
A well-designed system-wide architecture also aims to separate concerns. By separating responsibilities in the application well, the code not only gets easier to understand but also more flexible, loosely coupled and easier to test, too \cite{cleancoder} \cite{cleanarchitecture} \cite{onionarchitecture}. This helps to tell the responsibility and place the components by their names to the correct layers described in the previous section.

Consequently, to allow understanding the software architecture on its highest level, clear decisions should be made on the responsibility of the components. Once the components are chosen, their name should tell right away the intent they were made for. Applying the same approach on every level of abstracting throughout the application is crucial as it usually gives a good direction for the development. 

\subsection{The Dependency Rule}
The approach to a layered architecture also requires the definition of carefully designed dependencies. Identifying the correct dependencies and keeping them minimal is key to a loosely coupled software \cite{cleancode}. The Clean Architecture explained in the previous chapters is designed in a way, that such dependencies are considered at the first place.

The Dependency Rules dictates that the inner layers of the architecture do not know anything about the layers that are higher in the architecture \cite{cleanarchitecture} \cite{onionarchitecture}. As R. C. Martin points out \cite{cleanarchitecture}: \textit{"The outer circles are mechanisms. The inner circles are policies"}. This means that the business logic is defined closely on top of the entities and the high level components, such as user interface or database control the entities through the rules indirectly. This greatly helps developers to distinguish the level of abstraction for every layer. 

\subsection{Testability}

% An example of a floating figure using the graphicx package.
% Note that \label must occur AFTER (or within) \caption.
% For figures, \caption should occur after the \includegraphics.
% Note that IEEEtran v1.7 and later has special internal code that
% is designed to preserve the operation of \label within \caption
% even when the captionsoff option is in effect. However, because
% of issues like this, it may be the safest practice to put all your
% \label just after \caption rather than within \caption{}.
%
% Reminder: the "draftcls" or "draftclsnofoot", not "draft", class
% option should be used if it is desired that the figures are to be
% displayed while in draft mode.
%
%\begin{figure}[!t]
%\centering
%\includegraphics[width=2.5in]{myfigure}
% where an .eps filename suffix will be assumed under latex, 
% and a .pdf suffix will be assumed for pdflatex; or what has been declared
% via \DeclareGraphicsExtensions.
%\caption{Simulation results for the network.}
%\label{fig_sim}
%\end{figure}

% Note that the IEEE typically puts floats only at the top, even when this
% results in a large percentage of a column being occupied by floats.


% An example of a double column floating figure using two subfigures.
% (The subfig.sty package must be loaded for this to work.)
% The subfigure \label commands are set within each subfloat command,
% and the \label for the overall figure must come after \caption.
% \hfil is used as a separator to get equal spacing.
% Watch out that the combined width of all the subfigures on a 
% line do not exceed the text width or a line break will occur.
%
%\begin{figure*}[!t]
%\centering
%\subfloat[Case I]{\includegraphics[width=2.5in]{box}%
%\label{fig_first_case}}
%\hfil
%\subfloat[Case II]{\includegraphics[width=2.5in]{box}%
%\label{fig_second_case}}
%\caption{Simulation results for the network.}
%\label{fig_sim}
%\end{figure*}
%
% Note that often IEEE papers with subfigures do not employ subfigure
% captions (using the optional argument to \subfloat[]), but instead will
% reference/describe all of them (a), (b), etc., within the main caption.
% Be aware that for subfig.sty to generate the (a), (b), etc., subfigure
% labels, the optional argument to \subfloat must be present. If a
% subcaption is not desired, just leave its contents blank,
% e.g., \subfloat[].


% An example of a floating table. Note that, for IEEE style tables, the
% \caption command should come BEFORE the table and, given that table
% captions serve much like titles, are usually capitalized except for words
% such as a, an, and, as, at, but, by, for, in, nor, of, on, or, the, to
% and up, which are usually not capitalized unless they are the first or
% last word of the caption. Table text will default to \footnotesize as
% the IEEE normally uses this smaller font for tables.
% The \label must come after \caption as always.
%
%\begin{table}[!t]
%% increase table row spacing, adjust to taste
%\renewcommand{\arraystretch}{1.3}
% if using array.sty, it might be a good idea to tweak the value of
% \extrarowheight as needed to properly center the text within the cells
%\caption{An Example of a Table}
%\label{table_example}
%\centering
%% Some packages, such as MDW tools, offer better commands for making tables
%% than the plain LaTeX2e tabular which is used here.
%\begin{tabular}{|c||c|}
%\hline
%One & Two\\
%\hline
%Three & Four\\
%\hline
%\end{tabular}
%\end{table}

\section{Conclusion}
The conclusion goes here.

\begin{thebibliography}{1}

  \bibitem{cleanarchitecture}
R. C. Martin. (2012, Aug. 13) \emph{The Clean Architecture} [Online]. Available: \url{https://8thlight.com/blog/uncle-bob/2012/08/13/the-clean-architecture.html}

\bibitem{cleancoder}
R. C. Martin. \emph{The Clean Coder: A Code of Conduct for Professional Programmers}, 1st ed. Boston, MA, Pearson Education, Inc. 2011. 

\bibitem{cleancode}
R. C. Martin. \emph{Clean Code: A Handbook of Agile Software Craftsmanship}, 1st ed. Boston, MA, Pearson Education, Inc. 2008.

\bibitem{onionarchitecture}
J. Palermo. \emph{The Onion Architecture} [Online]. Available: \url{http://jeffreypalermo.com/blog/the-onion-architecture-part-1/}

\bibitem{codecomplete}
S. Steve McConnell. \emph{Code Complete: A Practical Handbook of Software Construction}, 2nd ed. Microsoft Press. 2004.

\bibitem{cusumano2008changing}
M. Cusumano. \emph{The Changing Software Business: From Products to Services and Other New Business Models Paper 236}. MIT Center for Digital Business. 2007.

\bibitem{gof}
E. Gamma, R. Helm, R. Johnson, J. Vlissides \emph{Design Patterns: Elements of Reusable Object-Oriented Software}, 1st ed. Addison Wesley. 1994.

\bibitem{impactofrefactoring}
M. Wahler, U. Drofenik, W. Snipes. \emph{Improving Code Maintainability: A Case Study on the Impact of Refactoring}. 2016 IEEE International Conference on Software Maintenance and Evolution. 2016.

\bibitem{studentscleancode}
M. Doyle, B. Buckley, W. Hao,  J. Walden. \emph{Work in Progress - Does Maintenance First Improve Student's Understanding and Appreciation of Clean Code and Documentation}. Frontiers in Education Conference (FIE). 2011.
% http://hillside.net/patterns

% Swift Design Patterns: The Easy Way; Standard Solutions for Everyday Programming Problems; Great for: Game Programming, System Analysis, App Programming, ... & Database Systems (Design Patterns Series) https://www.amazon.com/Swift-Design-Patterns-Solutions-Programming-ebook/dp/B01N2KE1T7/ref=sr_1_8?ie=UTF8&qid=1488874232&sr=8-8&keywords=software+patterns

\end{thebibliography}
\end{document}
