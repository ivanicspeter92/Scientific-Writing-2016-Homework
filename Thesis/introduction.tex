\subsection{Background and motivation}
    % context, setting - B2C business gathers a lot of data
	Many software businesses collect enormous amount of data generated by end-users. End-user data incorporate essential information about how users interact with the system of discussion as well as tell a lot about the users themselves. For instance, such data can explain the preferences of individuals, what kind of content users like, how users interact with eachother and so on.
    
    % what is the challenge?
    Due to the size of the data, a significant part of the information might remain hidden and therefore the business-critical information remains unseen \cite{inmon2007tapping, wegener2010integrating, introtodatamining}. As a result, there is a growing need for all businesses to introduce data analysis processes in their daily activies with the aim of understanding users and data generated by them better.
	
	% what is the potential in analyzing user data? why is it important not only to collect but also analyze data?
	Data analysis tools and methods already exist to help businesses operate on their data. However, these applications often not utilized, because companies rather focus on developing their service package than understanding the previously gathered data \cite{inmon2007tapping}. As a result, essential knowledge concerning previously collected data is neglected, which would be key for the future development of the business. Careful data analysis may reveal interesting relationships and point out facts, what human eyes would never notice otherwise. Consequently, Knowledge Discovery in Databases on user data is important and is essential part of Business Intelligence applications \cite{zarsky2002mine}. 

    % how do mobile devices have an impact on the amount of data? 
    With the continuous increase of mobile devices, the amount of data increases significantly. Due to the wide availability and commonness of smart phones and tablets, anybody can easily generate rich location, media or textual data. This fact further increases the call for research in the field of mobile and user data analysis, because it can help researchers to understand the society, human behavior, preferences and public opinions better.

    % what kind of results were derived in the past from end-user data analysis? 
    Depending on the portfolio of the business, analysis of end-user data can reveal various interesting findings. For example, movie databases often contain user reviews on movies, actors and producers of all sort. Such databases are open and are available for the public, and therefore the amount of data has grown huge over the past years. Unsurprisingly, databases like the Internet Movie Database (IMDB) drawn the interest of researchers \cite{saraee2004data, kabinsingha2012movie, sumathi2013performance}. The successful application of statistical methods and data mining techniques on such revealed interesting findings, such as that larger budget does not necessarily result in good ratings by the public, while actors have higher impact on the opinion of the audience \cite{saraee2004data}. On top of deriving such conclusions, machine learning techniques are emerging to predict future movie rating data, based on prior reviews of users \cite{saraee2004data} or the analysis of genre and other attributes of movies \cite{kabinsingha2012movie}.

    % how do researches on social media website data see the user data analysis?  
    Social media data analysis is another trending topic in discovering the secrets of user data. Various researches have applied advanced data mining techniques on Instagram data \cite{jang2015noreciprocity, bakhshi2014faces, hu2014we, jang2016teensengagemorewithfewerphotos, han2016teensarefrommars}, more specifically on the tags and comments that are attached to the images. Similarly, like activities and user-generated content is studied by many scientists. It is revealed, that Instagram users can be divided into two groups based on their activities: specialists, who publish and seek content around a certain topic of interest; and generalists, who are interested in all kinds of genres in the social media site \cite{jang2015noreciprocity}. Data mining techniques also allowed researchers to conclude, that the teenager users of Instagram tend to be more active, faster to react and more open to communicate with other users on social media \cite{jang2016teensengagemorewithfewerphotos, han2016teensarefrommars}. Furthermore, it was discovered that media content with human faces are more engaging than other type of media \cite{bakhshi2014faces}. Finally, rich social media data allowed researchers to analyze behavior and user preferences among genders, age groups and locations \cite{farseev2015harvestingmultiplesources}.

    % what does user data analysis reveal in a corporate social media environment? [refer to \cite{guy2016whatsyourorganizationlike}] 
    Social media platforms in enterprise environment are studied similarly to regular social networks \cite{guy2016whatsyourorganizationlike}. Despite the fact that the two types of social media platforms share many features, analysis performed in enterprise environment can provide great insights about how employees interact and cooperate. Among many other findings, studies have shown that blogs posts in an enterprise social media site tend to be more engaging and contribute to form communities inside the organization \cite{guy2016whatsyourorganizationlike}. Such insights are essential for higher management, because it can be used for instance to identify departments that tend to be less interactive or engaged.  

    % structured and unstructured data is being investigated by different researchers

    % why are these researches interesting? What do we learn from the society by performing analysis on user data? 
    The brief examples above demonstrate the capabilities of data mining in the field of user data analysis. The proper application of data mining methods allow us to learn more about the society as well as human behavior, which was not possible in the past. This information is essential for business operations, because it gives insights on the user groups, their preferences and what kind of content keeps the audience engaged. In the past these kind of insights were unavailable to managers and content providers. In the present time, access to such information can facilitate business processes, help determining the future content, analyze trends and understand the target group of particular services better. As a result, modern data mining techniques created the potential to study human preferences and behavior from a different angle, which was not possible in the past. 

    There is a growing need for data mining applications that operate on user data. Social media sites are living their golden age in the present time, which closes the distance between businesses and their customers by creating a new channel of communication. Many software solutions are integrating authentication of user profiles via social media sites to their service packages. From the users' perspective this integration is convenient, because they can register and authenticate themselves with their social media credentials. From the viewpoint of the businesses gathering user data was never easier, because such action allows easy access to the public information of users. 

    % why is this research conducted? 
    One reason behind conducting this research are the stimulating challenges about studying user data and the wide possibilities of the information that may lie around in databases. Statistical analysis on data, such as like activities, user comments, tags and image content can reveal interesting findings about user behavior, as demonstrated in the prior examples. Secondly, as user demographics-related data can be retrieved easily through social network sites in the present time, it opens new possibilities to seek correlation between user segments, based on gender or location for instance. Studies conducted in this field can also be interesting from the point of view of human behavior, which is another motivation towards conducting studies in this area. 

    % what is the focus?
    The focus of this research is to analyze demographic data of users and frequently liked content on the Internet. More specifically, the Choicely voting/audience engagement platform is taken as a case study. Association analysis, image recognition and other statistical methods are utilized to reveal hidden details about the tendencies of user preferences in terms of votes spent on contest participants in the platform.   

    % what is the motivation behind the focus? 
    From the perspective of the case company and its customers, such research is interesting, because it helps them to 

    \begin{itemize}
        \item understand the composition of their audience better,
        \item increase business value,
        \item have more advanced quantifiable means on the collected data,
        \item understand the cultural differences between users and cultures. 
    \end{itemize} 

\subsection{Research scope and objectives}
    % what is the wide scope of the thesis? What will we begin with as a theoretical background?
    Initially, the study presents an overview on the field of Data Mining as the theoretical of the study. Being a large field of science, the scope is focused only on the most important portion of Data Mining techniques and methods, namely association analysis. On top of that, the study addresses the possibilities and challenges of analyses performed on user data more specifically.   

    % in what context is this theoretical background relevant? Answer "Why?" % how does that relate to Choicely, particularly? 
    Building on top of the theoretical framework, the study is oriented towards applying techniques for data mining purposes at Choicely. Specific topics from the field of Data Science are chosen to obtain the answers to the research questions that are listed in the next section. At the beginning of this research project, the company does not utilize any advanced data analysis tools. This thesis work is motivated in the direction to establish the basis of a data mining framework at the company and hence increase the business value of the firm.  

    % what is the objective? 
    One of the objectives of this thesis work is to develop advanced data analysis tools in order to help the company and its customers to gain a better understanding on the data at hand. In order to do that, the available data is presented and analyzed, the most interesting questions are stated and the information with the business highest value is identified. Afterwards, potential data analysis methods are discussed which are capable to retrieve such information from the given data set.  

    % what is the focus?
    More specifically, the research is oriented towards applying data mining techniques to gain a better understanding on the users and their level of engagement. Initially, the study is oriented towards utilizing statistical measures to analyze properties of the user data. In the second part of the study, association analysis techniques are utilized to retrieve the attributes of the content, which engages users. Computer vision is utilized to classify the content of the images that are uploaded to the platform, while the transactional user data is investigated using association analysis techniques. % TODO finish when clarified

\subsection{Research questions and methodology}
    In this section, the research questions are presented and described. The questions are as follows:

    \begin{enumerate}[label=RQ\arabic*:]
        \item \textbf{What kind of content is more engaging for users and draws the most attention in the Choicely platform?} The aim of this research question is to understand which type of voting contests tend to engage more audience. Secondly, the question targets to analyze the attributes of those contestants, who received high number of votes. In other words, the aim is to find common features of contestants, who are highly rated by public opinion.     
        \item \textbf{What are the behavioral characteristics of users by location and gender?} This research question targets to answer the question how users tend to use the platform. This covers the analysis of what kind of content users seek, how the votes are spent and what similarities can be observed in the data. To gain more specific understanding, the users will be grouped by their gender and location data.  
        \item \textbf{?}
    \end{enumerate}  

    The methods...

\subsection{Thesis structure}
    % how is the thesis structured? 

    % what is in each of the sections to follow? 
