\subsection{Background and motivation}
    % context, setting - B2C business gathers a lot of data
	Many software businesses collect enormous amount of data generated by end-users \cite{chinesemobilebankingusers, bigdatamanagementrevolution, inmon2007tapping}. End-user data incorporates essential information about how users interact with the system of discussion as well as information about the users themselves. For instance, such digital footprints can explain the preferences of users, what kind of content they like, how they interact with the system and eachother or how frequent their activity in the platform is.
    
    % what is the challenge?
    Depending on the portfolio of the business, analysis of end-user data can reveal various interesting findings. For instance, in banking industry Big Data tools are often used to analyze demographic characteristics to maintain and establish new client relationships \cite{chinesemobilebankingusers, bigdatamanagementrevolution}. Extracting the information from the data is challenging: most businesses are to collect more data than what is humanly possible to process and to analyze \cite{inmon2007tapping, wegener2010integrating}. As a result, significant part of the knowledge might remain hidden and hence business-critical information remains unseen \cite{inmon2007tapping, wegener2010integrating, introtodatamining, chinesemobilebankingusers}. The computational methods of today's world enable us not only to achieve previously impossible tasks, but also to create tools that may outperform humans \cite{youyou2015computer}. Consequently, there is a growing need for all businesses to introduce data analysis processes in their daily activities with the aim of understanding users and data generated by them better.
    
    % how do mobile devices have an impact on the amount of data? 
    With the continuous growth of mobile devices in numbers, the amount of generated data has increased significantly. Complementary to the popularity of mobile devices, social media sites have grown a lot in the past years \cite{ottoni2013ladies, hu2014we, bakhshi2014faces, waheed2017investigation}, allowing users rich ways of interaction. Due to the combination of these two trends, users leave digital footprints in form of location, media, numerical or textual data in numerous places around the Internet. People often express their opinion by sharing, liking or commenting on data over social networks. Based on this information, algorithms can predict their personality traits more reliably than other humans would do \cite{youyou2015computer}. These facts further increases the call for research in the field of mobile and user data analysis, because it can help researchers to understand the society, human behavior, preferences and public opinions better.

	% what is the potential in analyzing user data? why is it important not only to collect but also analyze data?
    Data analysis tools and methods already exist to help businesses to process user data. However, these applications often not utilized, because companies rather focus on developing their service package than understanding the previously gathered data \cite{inmon2007tapping, bigdatamanagementrevolution}. As a result, essential knowledge concerning previously collected data is neglected, which would be key for the future development of the business. Careful data analysis may reveal interesting relationships and point out facts, what human eyes would never notice otherwise. Consequently, Knowledge Discovery in Databases is important and is essential part of Business Intelligence applications as well \cite{zarsky2002mine, bigdatamanagementrevolution}. 
    
    % what is the goal? what is this thesis about?
    The goal of this research is to study understand the potential use of user data in business and research environments in the present time with the application of Data Mining methods. The research aims to study how content on the Internet is observed by the users, who access the content. Secondly, it is studied what kind of feedback is received from the engaged audience in terms of like activities and comments to the provided content. Thirdly, the tendencies among the demographic characteristics between users are studied in relation to the content which gets them engaged online. More specifically, the Choicely voting/audience engagement platform is taken as a case study. Association analysis, statistical methods and computer vision are utilized to understand the online content and to reveal yet unseen details about the tendencies of user preferences in terms of votes spent on contest participants in the platform.
    
    % why is this research conducted, what is the motivation from researcher's pov? 
    One reason behind conducting this research are the stimulating challenges about studying user data and the wide possibilities of the information that may lie around in databases. Previous research have proven the relevance of statistical analysis on data, such as like activities \cite{jang2015noreciprocity, jang2016teensengagemorewithfewerphotos, ottoni2013ladies, guy2016whatsyourorganizationlike, jang2015no, youyou2015computer}, user comments \cite{jang2016teensengagemorewithfewerphotos}, tags \cite{jang2016teensengagemorewithfewerphotos} and image content \cite{hu2014we, bakhshi2014faces} by revealing interesting findings about user behavior. Moreover, as service providers often get access to user demographics-related data through social network sites in the present time, new possibilities become available to seek correlation between user segments. 
    %, based on gender or location for instance. 
    Studies conducted in this field are also interesting from the point of view of human behavior research, which is another motivation towards conducting studies in this area. 

    % what is the motivation behind the focus from Choicely's point of view? 
    From the perspective of the case company and its customers such research is interesting, because it helps the company's management and customers to 

    \begin{itemize}
        \item understand the composition of their engaged user base better,
        \item have more advanced quantifiable means on the collected data,
        \item gain an understanding on the users' behavior, 
        \item increase business value.
    \end{itemize} 

\subsection{Research questions and methodology}
    In this section, the research questions are presented and described. The questions are as follows:

    \begin{enumerate}[label=RQ\arabic*:]
        \item \textbf{How is user data understood and utilized in previous research and in industrial applications?}
        \item \textbf{What kind of content is more engaging for users and draws the most attention in the Choicely platform?} The aim of this research question is to understand which type of voting contests tend to engage more audience. Secondly, the question targets to analyze the attributes of those contestants, who received high number of votes. In other words, the aim is to find common features of contestants, who are highly rated by public opinion.     
        \item \textbf{What are the behavioral characteristics of users by location and gender?} This research question targets to answer the question how users tend to use the platform. This covers the analysis of what kind of content users seek, how the votes are spent and what similarities can be observed in the data. To gain more specific understanding, the users will be grouped by their gender and location data.  
        \item \textbf{How is it possible to identify anomalies, such as peaks in or fraud usage from the data?}
        \item \textbf{How is it possible to recommend relevant content for users based on prior user data?} The question targets to investigate and develop methods which can help users to find new content in the platform relevant to their interests.
    \end{enumerate}  

    % literature review
    The research consists of theoretical and a practical part. The foundations for the theoretical basis are retrieved by reviewing relevant literature. The sources include scientific journals, articles as well as books related to the topic of this study. The two former sources are retrieved through online digital libraries, such as the ACM digital library, IEEE Xplore and Google Scholar. After finding the first papers, the snowballing technique was used to retrieve further papers in the field. Furthermore, various keywords were used to obtain related literature in the reserach area. Keywords included, but were not limited to "data mining", "social media", "user data", "user behavior" and "demographic characteristics". 
    
    % practical part
    In the practical part of the study, the data contained of the Choicely voting platform's databases is analyzed. The analysis is fundamentally focused on two topics: the content uploaded by contest organizers and the users' data during the usage of the platform. The methods chosen for performing the analysis are [Method1, Method2, Method3] %TODO
    The choice behind these techniques is the successful application of these in previous research. 

    % data collection if needed % TODO add stuff if needed
    The data for this research is provided by the case company. The chosen techniques are applied and the analysis is performed on historical data, which was gathered through contests and votes in the past by Choicely and its customers. The structure and the properties of the data at hand is explained in the Section \ref{section::data-mining-at-choicely} to follow.
    
    % \begin{enumerate}
    %     \item in case the currently available data is not sufficient or incomplete, data can be gathered by volunteers (e.g. students and staff from the university)
    %     \item the data collection can be done through the Choicely platform \url{http://choicely.com} online,
    %     \item if needed, multiple sessions can be organized in the Choicely office or Kumpula for the data collection,
    %     \item training material for the participants can be provided in written form or a training session can be organized if needed,  
    %     \item Choicely is ready to reward the participants in the study with movie tickets or some other small gifts and hence encourage more to participate.
    % \end{enumerate}

\subsection{Research scope and objectives}
    % what is the wide scope of the thesis? What will we begin with as a theoretical background?
    Initially, the study presents an overview on the field of user data analysis in different research settings as the theoretical of the study. Being a large field of science, the scope is focused only on the most important portion of Data Mining techniques and methods, that are utilized to answer the research questions stated above. Particulary, the study addresses the possibilities and challenges of analyses performed on user data more specifically in terms of various statistical methods and asssociation analysis techniques.    

    % in what context is this theoretical background relevant? Answer "Why?" % how does that relate to Choicely, particularly? 
    Building on top of the theoretical framework, the study is oriented towards applying techniques for data mining purposes at Choicely. Specific topics from the field of Data Science are chosen to obtain the answers to the research questions that are listed in the next section. At the beginning of this research project, the company does not utilize any advanced data analysis tools. This thesis work is motivated in the direction to establish the basis of a data mining framework at the company and hence increase the business value of the firm.  

    % what is the objective? 
    One of the objectives of this thesis work is to develop advanced data analysis tools in order to assist the company and its customers to gain a better understanding on the data at hand. In order to do that, the available data is presented and analyzed, the most interesting questions are stated and the information with more influential business value is identified. Afterwards, potential data analysis methods are discussed which are capable to retrieve such information from the given data set.  

    % what is the focus?
    More specifically, the research is oriented towards applying data mining techniques to gain a better understanding on the users and their level of engagement. Initially, the study is oriented towards discovering what kind of  computational methods were applied in related studies to analyze properties of the user data. In the second part of the study, data mining techniques are utilized to discover the attributes of the content, which tends to engage a wider range of users. Computer vision is utilized to classify the content of the images that are uploaded to the platform, while the transactional user data is investigated using association analysis techniques. % TODO finish when clarified

\subsection{Thesis structure}
    % how is the thesis structured? 
    This section presents the structure of this thesis work. First an overview on the theoretical background and on the related work in the field of audience engagement and user data analysis is presented. 

    % what is in each of the sections to follow? 
