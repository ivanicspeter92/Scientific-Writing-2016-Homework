\subsection{Background and motivation}
    % context, setting - B2C business gathers a lot of data
	Many software businesses collect enormous amount of data generated by end-users. End-user data incorporate essential information about how users interact with the system of discussion as well as tell a lot about the users themselves. Due to the size of the data, a significant part of the information might remain hidden and therefore the business-critical information remains unseen \cite{inmon2007tapping, wegener2010integrating, introtodatamining}. As a result, there is a growing need for all businesses to introduce data analysis processes in their daily activies with the aim of understanding users and their generated data better.  
	
	% what is the potential in analyzing user data? why is it important not only to collect but also analyze data?
	Data analysis tools and methods already exist to help businesses operate on their data. However, these applications often not utilized, because companies rather focus on developing their service package than understanding the previously gathered data \cite{inmon2007tapping}. As a result, essential knowledge concerning previously collected data is neglected, which would be key for the future development of the business. Careful data analysis may reveal interesting relationships and point out facts, what human eyes would never notice otherwise. Consequently, Knowledge Discovery in Databases on user data is important and is essential part of Business Intelligence applications \cite{zarsky2002mine}. 

    % how do mobile devices have an impact on the amount of data? 
    With the continous increase of mobile devices, the amount of data increases significantly. Due to the wide availability and commonness of smart phones and tablets, anybody can easily generate rich location, media or textual data. This fact further increases the call for research in the field of mobile and user data analysis, because it can help researchers to understand the society, human behavior, preferences and public opinions better.

    % what kind of results were derived in the past from end-user data analysis? 
    Depending on the portfolio of the business, analysis of end-user data can reveal various interesting findings. For example, movie databases often contain user reviews on movies, actors and producers of all sort. Such databases are open and are available for the public, and therefore the amount of data has grown huge over the past years. Unsurprisingly, databases like the Internet Movie Database (IMDB) drawn the interest of researchers \cite{saraee2004data, kabinsingha2012movie, sumathi2013performance}. The successful application of statistical methods and data mining techniques on such revealed interesting findings, such as that larger budget does not necessarily result in good ratings by the public, while actors have higher impact on the opinion of the audience \cite{saraee2004data}. On top of deriving such conclusions, machine learning techniques are emerging to predict future movie rating data, based on prior reviews of users \cite{saraee2004data} or the analysis of genre and other attributes of movies \cite{kabinsingha2012movie}.

    % how does researches on social media website data see the user data analysis?  
    Social media data analysis is another trending topic in discovering the secrets of user data. Various researches have applied advanced data mining techniques on Instagram data \cite{jang2015noreciprocity, bakhshi2014faces, hu2014we, jang2016teensengagemorewithfewerphotos, han2016teensarefrommars}, the attached tags and comments added to the images. 

    % why are these researches interesting? What do we learn from the society by performing analysis on user data? 

    % structured and unstructured data is being investigated by different researchers

    % what is the focus? 

    % what is the motivation behind the focus? 

\subsection{Research scope and objectives}


\subsection{Thesis structure}
    % how is the thesis structured? 

    % what is in each of the sections to follow? 
