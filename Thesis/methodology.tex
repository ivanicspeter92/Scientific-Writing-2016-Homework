% introduce the two parts of the thesis
The research consists of a theoretical and a practical part. First, an exploratory literature review is carried out to establish the basis, the relevance and the background of this study. As the research continues with the empirical part, the exploratory literature review forms the basis of understanding the background and assists to answer the research questions and the challenges at the case company. 

% literature review sources
The foundations for the theoretical basis are retrieved by reviewing relevant literature. The sources include scientific journals, articles as well as books related to the topic of this study. The two former sources are retrieved through online digital libraries, such as the ACM digital library, IEEE Xplore and Google Scholar. After finding the first papers, the snowballing technique was used to retrieve further papers in the field. Furthermore, various keywords were used to obtain related literature in the research area. Keywords included, but were not limited to "data mining", "social media", "user data", "user behavior" and "demographic characteristics". 

% why? 
This part of the research forms a sound basis on the understanding and possible utilization of user data. Furthermore, it establishes a common ground for the practical part of the study and helps to address the research gaps and questions. By reviewing related papers it also becomes clear, what kind of challenges are faced and techniques are utilized by other professionals in similar studies.

% practical part
In the practical part of the study, the data contained of the Choicely voting platform's databases is analyzed. The analysis is fundamentally focused on two topics: the content uploaded by contest organizers and the users' data during the usage of the platform. 

% The methods chosen for performing the analysis are [Method1, Method2, Method3] %TODO The choice behind these techniques is the successful application of these in previous research. 

% data collection if needed % TODO add stuff if needed
The data for this research is provided by the case company. The chosen techniques are applied and the analysis is performed on historical data, which was gathered through contests and votes in the past by Choicely and its customers. The structure and the properties of the data at hand is explained in the Chapter \ref{section::data-mining-at-choicely} to follow.

% \begin{enumerate}
%     \item in case the currently available data is not sufficient or incomplete, data can be gathered by volunteers (e.g. students and staff from the university)
%     \item the data collection can be done through the Choicely platform \url{http://choicely.com} online,
%     \item if needed, multiple sessions can be organized in the Choicely office or Kumpula for the data collection,
%     \item training material for the participants can be provided in written form or a training session can be organized if needed,  
%     \item Choicely is ready to reward the participants in the study with movie tickets or some other small gifts and hence encourage more to participate.
% \end{enumerate}

% Exploratory Data Analysis (EDA)
To grasp on the currently available data, Exploratory Data Analysis (EDA) is utilized as the first step of the research. A comprehensive overview on the data related to the most important entities in the Choicely platform is performed by calculating basic statistical measures and visualizing the data. Such entities are the user profiles, the contests, contest participant and the votes performed by users on the contest participants. The findings of the EDA are then utilized to determine how to prune the data such that the acquired sample is representative, does not contain redundant nor faulty data. The results of the EDA are explained in the next chapter. 

% Preprocessing and pruning
Based on the results from the EDA, a subset of contests is chosen for more careful analysis. The subset of the contests is limited to those, which have engaged the most users and gathered a higher enough number of votes and unique voters (users who have voted at least once in the contest). After applying the pruning rules, some part of the data is preprocessed so that the analyses can be performed easily. The pruning rules and preprocessing steps are explained in the next chapter. 

% Association Analysis
More importantly, Association Analysis is performed on the combined demographic, vote and the image label data in the chosen contests. Frequent Itemset Analysis is then performed on the data to identify behaviors and preferences of the different demographic groups. Association Rule Discovery is performed on the same data to understand which pair of traits can engage more of the (targeted) audience.  

% check \cite{socialdiversityongithub} -> Blau index / diversity index: reflects how many different types (such as species) there are in a dataset (a community), and simultaneously takes into account how evenly the basic entities (such as individuals) are distributed among those types.

% what are the decisions behind the choices? 
% what were the alternatives and why were they rejected? 
% performance comparison - why was the local cache added?