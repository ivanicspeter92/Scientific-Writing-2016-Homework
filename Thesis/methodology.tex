% introduce the two parts of the thesis
The research consists of a theoretical and a practical part. First, an exploratory literature review is carried out to establish the basis, the relevance and the background of this study. As the research continues with the empirical part, the exploratory literature review forms the basis of understanding the background and assists to answer the research questions and the challenges at the case company. 

% literature review sources
The foundations of the theoretical basis are retrieved by reviewing relevant literature. The sources include scientific journals, articles as well as books related to the topic of this study. The two former sources are retrieved through three online digital libraries: the ACM digital library, IEEE Xplore and Google Scholar. 

After finding the first papers, the snowballing technique was used to retrieve further papers in the field. Various keywords were used to obtain related literature in the research area. Keywords included, but were not limited to "data mining", "social media", "user data", "user behavior", "demographic characteristics" and "digital footprints". The search for the retrieved papers was performed between May 2017 and February 2018. The retrieved papers are then critically analyzed and the most interesting points from and across the studies are summarized in this short paper. The emphasis is placed on the research goals towards which other studies utilize demographic and user data as well as the computational methods, that were chosen by other researchers. 

% why? 
This part of the research forms a sound basis on the understanding and possible utilization of user data. Furthermore, it establishes a common ground for the practical part of the study and helps to address the research gaps and questions. By reviewing related papers it also becomes clear, what kind of challenges are faced and techniques are utilized by other professionals in similar studies.

% practical part
In the practical part of the study, the data contained of the Choicely voting platform's databases is analyzed. The analysis is fundamentally focused on two topics: the content uploaded by contest organizers and the users' data during the usage of the platform. 

The data for this research is provided by the case company. The chosen techniques are applied and the analysis is performed on historical data, which was gathered through contests and votes in the past by Choicely and its customers. The structure and the properties of the data at hand is explained in the Chapter \ref{section::data-mining-at-choicely} to follow. 

% Exploratory Data Analysis (EDA)
To grasp on the currently available data, Exploratory Data Analysis (hereinafter EDA) is utilized as the first step of the research. The underlying reasoning behind this decision is mainly that EDA performed on any dataset helps researchers to familiarize themselves with the dataset at hand, hence to choose the appropriate preprocessing and data analysis techniques \cite{introtodatamining}. 

Accordingly, comprehensive overview on the data related to the most important entities in the Choicely platform is performed by calculating basic statistical measures and visualizing the data. Such entities are the user profiles, the contests, contest participant and the votes performed by users on the contest participants. The findings of the EDA are then utilized to determine how to prune the data such that the acquired sample is representative, does not contain redundant nor faulty data. The results of the EDA are explained in the next chapter. 

% Preprocessing and pruning
Based on the results from the EDA, a subset of contests is chosen for more careful analysis. The subset of the contests is limited to those, which have engaged the most users and gathered a higher enough number of votes and unique voters (users who have voted at least once in the contest). After applying the pruning rules, some part of the data is preprocessed so that the analyses can be performed easily. Preprocessing in this case means joining the datasets together along the identifiers of contestants, images or users, respectively.  

% computer vision
To address the lack of meta data on the contestants images, Computer Vision is utilized. Being a widely used technique, researchers have successfully utilized this technique in similar studies for various purposes \cite{hu2014we, farseev2015harvestingmultiplesources, han2016teensarefrommars, bakhshi2014faces}. One of the areas where the technique was used with great success by Farseev et al. \cite{farseev2015harvestingmultiplesources} is the extraction of concepts from images. In this study, the field of Computer Vision is applied through the application of Google Vision to identify the content that is on the contestants' images in the Choicely platform. 

% Association Analysis
More importantly, it was studied how user behavior can be modelled and how similarities between groups' preferences can be measured. Several related studies have utilized different methods, such as Text/Natural Language Processing \cite{ottoni2013ladies, farseev2015harvestingmultiplesources, jang2016teensengagemorewithfewerphotos, kabinsingha2012movie, han2016teensarefrommars}, Supervised Machine Learning \cite{chinesemobilebankingusers, saraee2004data, kabinsingha2012movie, farseev2015harvestingmultiplesources, han2016teensarefrommars, jang2015no, bakhshi2014faces} or Unsupervised Machine Learning \cite{saraee2004data, hu2014we, jang2015no} to study similar areas, such as like activities, gender differences or clustering of users. 

Similarly to Ottoni et al. \cite{ottoni2013ladies}, this study utilizes Association Analysis is performed on the combined demographic, vote and the image label data in the chosen contests. Association Analysis was originally formulated and studied by Agrawal et al. as Associations \cite{database_mining_agrawal, mining_association_rules_agrawal}, which has grown to a large extend in the numerous research fields. Research areas in this field also include Frequent Itemset Generation and Association Rule Mining, originally introduced by Agrawal et al \cite{database_mining_agrawal, mining_association_rules_agrawal}. This study utilizes these on the given data to identify behaviors and preferences of the different demographic groups. The paragraphs to follow introduce this technique based on the textbook written by of Tan, Steinbach and Kumar \cite{introtodatamining}.

In particular, this studie generates frequent itemsets from the computer-vision recognized image labels in and users' voting data in the Choicely platform. In other words, the vote data generated through the usage of the platform is facilitated with the list of labels on the contestants' images. The list of labels is gathered for every vote transaction and is handled as a container (often called as "market basket" or "market data" \cite{Brin97dynamicitemset, Brin1997BeyondMB, Raeder2010MarketBA}) out of which frequently co-occurring items are extracted from. 

More formally, the support count of an itemset can be defined as

\begin{equation}
    \sigma (X) = |\{ t_i | X \subseteq t_i, t_i \in T \}|
\end{equation}

where $T = {t_1, t_2, ..., t_N}$ is the set of transactions, $X$ is the itemset of discussion and $| \cdot |$ donates the number of number of elements in an itemset. A further essential constraint on the itemsets is that they have to contain at least one item \cite{introtodatamining}. In simple language, this number tells how many transactions contained the itemset $X$ in the given population. The itemset that contains $k$ unique items is often referred to as a k-itemset. 

Deriving from this, the supports for each itemset can be calculated, that is the percentage of transactions in which the itemset is present. In other words, the support tells the probability of itemset $X$ to be present in a randomly chosen transaction. Formally, this can be stated as 

\begin{equation}
    s(X) = \frac{\sigma (X)}{N}
\end{equation}

where $N$ is the number of transactions. In this thesis work, vote transactions are filtered by demographic groups $G$ and the support $S(X|G)$ is calculated. For instance, by calculating $S(X|gender = male)$ one can obtain the itemset supports for males in a given list of vote transactions. The extraction of the itemset supports is achieved using the Apriori algorithm, which addresses the complexity growth of itemsets during their generation and analysis \cite{introtodatamining}. 

The itemset supports can be calculated for all demographic groups in a single contest or a list of contests (e.g. every contest in certain category). The frequent itemsets (which exceed the $minsup$ threshold) are organized into the rows of a matrix during the analysis, with each the columns representing a demographic group which is being analyzed. The cells in the matrix contain the support values of the itemsets for the given group of discussion. 

The calculated support values can be studied further such that the tendencies in voting among demographic groups can be compared (RQ3). To answer RQ3, the next task is to seek significant differences or interesting similarities in the itemset support values of demographic groups. There are two challenges with this approach: 

\begin{enumerate}
    \item the number of potential itemsets can be large, and
    \item there is not a clear way of knowing which of the resulting itemsets will be interesting to look at from the viewpoint of a data scientist.
\end{enumerate} 

The first challenge is addressed by the Apriori Principle, originally introduced by Agrawal et al. \cite{mining_sequential_patterns_agrawal}. The principle states that any subset of an infrequent itemset is inevitably infrequent subset as well \cite{mining_sequential_patterns_agrawal, introtodatamining}. This means that if the algorithm encounters an itemset under the desired minimal support threshold, its subsets can be safely dropped from any further analysis. This way less engaging content can be filtered out and the more interesting itemsets can be pinpointed. According to the Association Rule Discovery \cite{database_mining_agrawal, mining_association_rules_agrawal, introtodatamining}, all itemsets which have support values less than $minsup$ are considered as infrequent, hence not shown on the algorithm's output. As a consequence, any subsets of such itemsets are going to be infrequent as well. 

More interestingly, it is challenging to know which itemsets or demographic groups show similarities with eachother. After collecting the image labels and extracting the itemset supports, one may want to identify patterns or the underlying structure in the data, so that interesting findings can be extracted from it. To address this problem, the Co-Clustering (often called also as biclustering or two-mode clustering) is used, which is an approach towards identifying patterns simultaneously on columns and rows of a (large) matrix of numbers \cite{coclustering}. Like other clustering techniques, this is a Unsupervised Machine Learning technique as well, because the structure of the data is not known beforehand. 

The reason behind choosing the Co-Clustering approach is its relevance and applicability to this problem. In their structured study, Van Mechelen, Bock and De Boeck shows the relevance and wide applicability of this approach, also explaining the mathematical details behind the different approaches \cite{coclustering}. The researchers conclude, that the large number of models already had an impact on medical field, as it provides tools for DNA sequence analysis or classifying syndromes based on patients' symptoms \cite{coclustering}. Van Mechelen, Bock and De Boeck also argues, that despite its robustness, the two-mode clustering has not reached as many applications as its "simpler" version, one-way clustering. 

In this study, Co-Clustering is used on the matrix of extracted support values explained above. The aim of applying this method is to identify which demographic groups as well as itemsets show similarities to eachother. By using the Co-Clustering method, the rows and columns of the constructed matrix can be analyzed, its rows and columns are clustered at the same time. This way, the grouping of the similar entities (rows, columns and cells) can be achieved, which contributes to the analysis of the data.

% The combination of these methods is unique as none of the studied researches utilizes them in such combination with these kind of goals in mind. The data available at Choicely contains a great specimen of user data, in which a large part of the data is already available for analysis. The potential application of these methods are interesting for the case company, as their platform currently does not include any tool with this purpose, while the need from the customers' side continously rises.