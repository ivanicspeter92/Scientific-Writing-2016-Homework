% introduce the two parts of the thesis
The research consists of a theoretical and a practical part. First, an exploratory literature review is carried out to establish the basis, the relevance and the background of this study. As the research continues with the empirical part, the exploratory literature review forms the basis of understanding the background and assists to answer the research questions and the challenges at the case company. 

The sources for the literature review include scientific journals, articles as well as textbooks related to the topic of this study. The two former sources are retrieved through three online digital libraries: the ACM digital library, IEEE Xplore and Google Scholar. After finding the first papers, the snowballing technique was used to retrieve further papers in the field. Various keywords were used to obtain related literature in the research area. Keywords included, but were not limited to "data mining", "social media", "user data", "user behavior", "demographic characteristics" and "digital footprints". The search for the retrieved papers was performed between May 2017 and October 2017. The retrieved papers are then critically analyzed and the most interesting points from and across the studies are summarized in the corresponding chapter. The emphasis is placed on the research goals towards which other studies utilize demographic and user data as well as the computational methods chosen for addressing those goals. 

% why? 
This part of the research forms a sound basis on the understanding and possible utilization of user data. Furthermore, it establishes a common ground for the practical part of the study and helps to address the research gaps and questions. By reviewing related papers it also becomes clear, what kind of challenges are faced and techniques are utilized by other professionals in similar studies.

% practical part
In the practical part of the study, the data of the Choicely voting platform's databases is analyzed. The analysis is fundamentally focused on two topics: the content uploaded by contest organizers and the users' behavior while using the platform. 

The data for this research is provided by the case company. The chosen techniques are applied and the analysis is performed on historical data, which was gathered through contests and votes in the past by Choicely and its customers. The structure and the properties of the data at hand is explained in the Chapter \ref{section::data-mining-at-choicely} to follow. 

% Exploratory Data Analysis
To grasp on the currently available data, Exploratory Data Analysis (hereinafter EDA) is utilized as the first step of the research. The term of EDA originates from Tukey \cite{tukey77}, who intially proposed an informal study of data repositories. On top of introducing the term, Tukey's work proposes various approaches and best practices for data exploration. Over the years, the EDA concept has become widely accepted and utilized as a tool for understanding and getting an overview on data. 

The underlying reasoning behind performing EDA in this study is its wide usage among researchers to explore the dataset at hand. This way preliminary hypotheses can be made about the data as well as erroneous data can be removed before the beginning of the real data analysis. EDA in many cases also helps to choose the appropriate preprocessing and data analysis techniques for the rest of the study \cite{tukey77, introtodatamining}. 

Accordingly, comprehensive overview on the data related to the most important entities in the Choicely platform is performed by calculating basic statistical measures and visualizing the data. Such entities are the user profiles, the contests, contest participant and the votes performed by users on the contest participants. The findings of the EDA are then utilized to determine how to prune the data such that the acquired sample is representative, does not contain redundant nor faulty data. The results of the EDA are explained in the next chapter. 

% Preprocessing and pruning
Based on the results from the EDA, a subset of contests is chosen for more careful analysis. The subset of the contests is limited to those, which have engaged the most users and gathered a higher enough number of votes and unique voters (users who have voted at least once in the contest). After applying the pruning rules, some part of the data is preprocessed so that the analyses can be performed easily. Preprocessing in this case means joining the datasets together along the identifiers of contestants, images or user profiles, respectively.  

% computer vision
To address the lack of meta data on the contestants images, Computer Vision is utilized. Being a widely used technique, researchers have successfully utilized this technique in similar studies for various purposes \cite{hu2014we, farseev2015harvestingmultiplesources, han2016teensarefrommars, bakhshi2014faces}. One of the areas where the technique was used with great success by Farseev et al. \cite{farseev2015harvestingmultiplesources} is the extraction of concepts from images. In this study, the field of Computer Vision is applied through the application of Google Vision to identify the content that is on the contestants' images in the Choicely platform. 

% Association Analysis
Previous studies have looked into how user behavior can be modelled and how similarities between groups' preferences can be measured. Several related studies have utilized different methods, such as Text/Natural Language Processing \cite{ottoni2013ladies, farseev2015harvestingmultiplesources, jang2016teensengagemorewithfewerphotos, kabinsingha2012movie, han2016teensarefrommars}, Supervised Machine Learning \cite{chinesemobilebankingusers, saraee2004data, kabinsingha2012movie, farseev2015harvestingmultiplesources, han2016teensarefrommars, jang2015no, bakhshi2014faces} or Unsupervised Machine Learning \cite{saraee2004data, hu2014we, jang2015no} to study similar areas, such as like activities, gender differences or clustering of users. 

Similarly to Ottoni et al. \cite{ottoni2013ladies}, this study utilizes Association Analysis on the combined demographic, vote and the image label data in the chosen contests. Association Analysis was originally formulated and studied by Agrawal et al. as Associations \cite{database_mining_agrawal, mining_association_rules_agrawal}, which has grown to a large extend in the numerous research fields. Research areas in this field also include Frequent Itemset Generation and Association Rule Mining, originally introduced by Agrawal et al \cite{mining_association_rules_agrawal}. This study utilizes these on the given data to identify behaviors and preferences of the different demographic groups. The paragraphs to follow introduce this technique based on the textbook written by of Tan, Steinbach and Kumar \cite{introtodatamining}.

In particular, frequent itemsets are generated from the computer-vision recognized image labels in and users' voting data in the Choicely platform. In other words, the contestants' images in the platform are facilitated with the list of labels. The list of labels is gathered for every vote transaction and is handled as a container (often called as "market basket" or "market data" \cite{Brin97dynamicitemset, Brin1997BeyondMB, Raeder2010MarketBA}) out of which frequently voted items are extracted from. 

More formally, the support count of an itemset can be defined as

\begin{equation}
    \sigma (X) = |\{ t_i | X \subseteq t_i, t_i \in T \}|
\end{equation}

where $T = {t_1, t_2, ..., t_N}$ is the set of transactions, \emph{X} is the itemset of discussion and $| \cdot |$ donates the number of number of elements in an itemset \cite{introtodatamining}. A further essential constraint on the itemsets is that they have to contain at least one item \cite{introtodatamining}. In simple language, this number tells how many transactions contained the itemset \emph{X} in the given population. The itemset that contains \emph{k} unique items is often referred to as a \emph{k-itemset}. 

Deriving from this, the supports for each itemset can be calculated, that is the percentage of transactions in which the itemset is present. In other words, the support tells the probability of itemset \emph{X} to be present in a randomly chosen transaction. Formally, this can be stated as 

\begin{equation}
    s(X) = \frac{\sigma (X)}{N}
\end{equation}

where \emph{N} is the number of transactions. In this thesis work, vote transactions are filtered by demographic groups \emph{G} and the support \emph{S(X|G)} is calculated. For instance, by calculating \emph{S(X|gender = male)} one can obtain the itemset supports for males in a given list of vote transactions. Note that other researches might use the term confidence for studying association rules, but the present study consistenly uses the term of support with this meaning. The extraction of the itemset supports is achieved using the Apriori algorithm, which addresses the complexity growth of itemsets during their generation \cite{introtodatamining}. 

The calculated support values can be studied further such that the tendencies in voting among demographic groups can be compared (RQ3). To answer RQ3, the next task is to seek significant differences or interesting similarities in the itemset support values of demographic groups. There are two challenges with this goals: 

\begin{enumerate}
    \item the number of potential itemsets can be large, and
    \item there is not a clear way of knowing which of the resulting itemsets will be interesting to look at from the viewpoint of a data scientist.
\end{enumerate} 

The first challenge is addressed by the Apriori Principle, originally introduced by Agrawal et al. \cite{agrawal1994fast}. The principle states that any subset of an infrequent itemset is inevitably infrequent subset as well \cite{agrawal1994fast, introtodatamining}. This means that if the algorithm encounters an itemset under the desired minimal support threshold, its subsets can be safely dropped from any further analysis. For this reason, Association Rule Discovery algorithms often introduce a \emph{minsup} threshold \cite{database_mining_agrawal, mining_association_rules_agrawal, introtodatamining}, which controls the value below which itemsets are considered as infrequent. As a consequence, less popular itemsets can be filtered out and the more interesting itemsets can be pinpointed. 

Lift is another common measure in Association Analysis, which fits the purposes of this study very well. In his textbook Tuffery \cite{tuffery2011data} provides a great introduction to this measure and argues, why it complements association rule mining. In his terms lift also tells the improvement introduced by a rule \cite{tuffery2011data}, which can assist researchers to find out which rules are more (or less) useful for the analysis. The lift is hence a measure on the dependency between condition and the consequence of the association rule. Obtaining the lift value is a simple division between the support of the rule and the result's probability. Formally, it can be stated as

\begin{equation}
    L(X|G) = \frac{S(X|G)}{S(X)}
\end{equation}

The value of lift varies on the range of 0.0 to positive infinity. If the value of the lift is 1.0, the condition makes no impact on the result of the rule, hence is typically not considered very interesting. In case the value is less than 1.0, the rule is said to be useless, while over 1.0 is usually called as useful rule \cite{tuffery2011data}. 

The aim by calculating lift values in this research is to identify how different age groups interact with the itemsets. For instance, if the itemset \emph{X} is more appealing to males than how often it appears in the complete dataset, its lift value \emph{L(X|gender=male) > 1.0}. Similarly, if the opposite can be said for females, the \emph{L(X|gender=female) < 1.0}. In other words, the more engaging itemsets have higher lift values which can suggest more attraction from the audience and hence lead to further, more careful analysis.

Tackling the second challenge stated above is a bit less obvious. The itemset supports can be calculated for all demographic groups in a single contest or a list of contests (e.g. every contest in certain category or contests hosted by the same organizer). The frequent itemsets (which exceed the \emph{minsup} threshold) are organized into the rows of a matrix during the analysis, with each the columns representing a demographic group which is being analyzed. The cells in the matrix contain the support values of the itemsets for the given group of discussion. 

Knowing which itemsets or demographic groups show similarities to eachother could be extracted from such matrix. After collecting the image labels and extracting the itemset supports, one may want to identify patterns or the underlying structure in the data, so that interesting findings can be extracted from it. To address this problem, the Co-Clustering (often called also as biclustering or two-mode clustering \cite{coclustering}) is used, which was first introduced by Hartigan \cite{hartigan-direct-clustering-data-1972} on the historical election voting data between 1900-1968. The main goals in his study are to partition the given dataset such that the similarities between states and over years are extracted from the dataset \cite{hartigan-direct-clustering-data-1972}. 

The novelty of the Co-Clustering approach lies in identifying patterns simultaneously on columns and rows of a (large) matrix of numbers \cite{hartigan-direct-clustering-data-1972, coclustering}. Like other clustering techniques, this is a Unsupervised Machine Learning technique as well, because the structure of the data is not known beforehand. The key advantage of Co-Clustering compared to other methods on such data is that the clusters can be interpreted directly after its completion \cite{hartigan-direct-clustering-data-1972}. In other words, the output directly tells the boundaries of the clusters and matches items to the clusters they belong to.

The reason behind choosing the Co-Clustering approach is its relevance and applicability to this problem. In their structured study, Van Mechelen, Bock and De Boeck explain the mathematical details behind the different approaches in depth \cite{coclustering}, which can vary from simple division of rows and columns of the data matrix to overlapping and even nested clusters. The researchers conclude, that the large number of models already had an impact on medical field, as it provides tools for DNA sequence analysis or classifying syndromes based on patients' symptoms \cite{coclustering}. The widespread suitability of the approach is proven by other reserachers in various fields, such as modelling text documents and words \cite{coclustering_documents}, gene expression data, time course data and sensor network data \cite{cho2008co}. 

Van Mechelen, Bock and De Boeck also argues \cite{coclustering}, that despite its robustness, the two-mode clustering has not reached as many applications as its "simpler" version, one-way clustering. This is another motivation towards experimenting with this method and evaluating its applicability to the problem studied in this work. For simiplicity, this research utilizes data partitioning \cite{coclustering}, which can be seen as the simplest Co-Clustering method. Data partitioning in this respect means the identification of areas in the data, where items semantically belong together. 

Co-Clustering in the present research is performed on the support values of itemsets by demographic groups. That is, the input matrix lists the itemsets constructed from the labels of the contest participants' images, while its columns contain the demographic groups, for which the supports have been calculated. The cells of the matrix containt the actual support values, which range from 0.0 to 1.0 inclusive.  

The aim of applying this method is to identify which demographic groups as well as itemsets show similarities to eachother. By using the Co-Clustering method, the rows and columns of the constructed matrix can be analyzed, its rows and columns are clustered at the same time. This way, the grouping of the similar entities (rows, columns and cells) can be achieved, which contributes to identifying patterns and revealing the underlying structure of the data.

The combination of these methods is unique as none of the studied researches utilizes them in such combination with these kind of goals in mind. The data available at Choicely contains a great specimen of user data, in which a large part of the data is already available for analysis. The potential application of these methods are interesting for the case company, as their platform currently does not include any tool with this purpose, while the need from the customers' side continously rises.