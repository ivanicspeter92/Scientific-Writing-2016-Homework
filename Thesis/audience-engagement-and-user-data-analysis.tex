% catch up the thread from the introduction why user data is important
As outlined in the previous section, careful analysis on Big Data can enhance business domain understanding for any company. Studying Big Data has became a widely studied and interesting topic all around the Information Technology industry in the recent years. Unsurprisingly, significant amount of the data is somehow related to the users of the software at hand. Such data often incorporates essential data about the users themselves, such as their demographic data, preferences or how they interact with the software.      

% what kind of results were derived in the past from end-user data analysis? 
For example, movie databases often contain user reviews on movies, actors and producers of all sort. Such databases are open and are available for the public, and therefore the amount of data has grown huge over the past years. Unsurprisingly, databases like the Internet Movie Database (IMDB) has drawn the interest of researchers \cite{saraee2004data, kabinsingha2012movie, sumathi2013performance}. The successful application of statistical methods and data mining techniques have revealed interesting findings, such as that larger budget for movies does not necessarily result in good ratings by the public, while actors have higher impact on the opinion of the audience \cite{saraee2004data}. On top of deriving such conclusions, machine learning techniques are emerging to predict future movie rating data, based on prior reviews of users \cite{saraee2004data} or the analysis of genre and other attributes of movies \cite{kabinsingha2012movie}.

% demographics characteristics in banking
Demographic characteristics play a role in analyzing user adoption and behavior in the banking industry. The study conducted by Wang and Petrounias \cite{chinesemobilebankingusers} concludes that mobile banking in China is more popular among middle-aged males, while the younger generation has not adopted to the new trends yet. Utilizing Big Data analytics in the same study has revealed the group of citizens and products that could be targeted in upcoming marketing campaigns \cite{chinesemobilebankingusers}. Social diversity was also studied by other researchers in the context of software development growth \cite{socialdiversityongithub}. In their study, Aué et al. have clearly identified correlation between the success of open source projects and the contributors' gender and cultural diversity by utilizing well chosen statistical methods \cite{socialdiversityongithub}.

% how do researches on social media website data see the user data analysis?  
Unsurprisingly, social media data analysis is another trending topic in discovering the secrets of user data. Various researches have applied advanced data mining techniques on Instagram data \cite{jang2015noreciprocity, bakhshi2014faces, hu2014we, jang2016teensengagemorewithfewerphotos, han2016teensarefrommars}, more specifically on the tags and comments that are attached to the images. Similarly, like activities and user-generated content is studied by many scientists. It is revealed, that Instagram users can be divided into two groups based on their activities: specialists, who publish and seek content around a certain topic of interest; and generalists, who are interested in all kinds of genres in the social media site \cite{jang2015noreciprocity}. Data mining techniques also allowed researchers to conclude, that the teenager users of Instagram tend to be more active, faster to react and more open to communicate with other users on social media \cite{jang2016teensengagemorewithfewerphotos, han2016teensarefrommars}. Furthermore, it was discovered that media content with human faces are more engaging than other type of media \cite{bakhshi2014faces}. Finally, rich social media data allowed researchers to analyze behavior and user preferences among genders, age groups and locations \cite{farseev2015harvestingmultiplesources}.

% what does user data analysis reveal in a corporate social media environment? [refer to \cite{guy2016whatsyourorganizationlike}] 
Social media platforms in enterprise environment are studied similarly to regular social networks \cite{guy2016whatsyourorganizationlike}. Despite the fact that the two types of social media platforms share many features, analysis performed in enterprise environment can provide great insights about how employees interact and cooperate. Among many other findings, studies have shown that blogs posts in an enterprise social media site tend to be more engaging and contribute to form communities inside the organization \cite{guy2016whatsyourorganizationlike}. Such insights are essential for higher management, because it can be used for instance to identify departments that tend to be less interactive or engaged.  

% structured and unstructured data is being investigated by different researchers

% why are these researches interesting? What do we learn from the society by performing analysis on user data? 
The brief examples above demonstrate the capabilities of data mining in the field of user data and demographic analysis. The proper application of data mining methods allow us to learn more about the society as well as human behavior, which was not possible in the past. This information is essential for business operations, because it gives insights on the user groups, their preferences and what kind of content keeps the audience engaged. In the past these kind of insights were unavailable to managers and content providers. In the present time, access to this information can facilitate business processes, helps determining the future content, analyzing trends and understanding target groups of particular services better. In sum, modern data mining techniques created the potential to study human preferences and behavior from a different angle. 