% catch up the thread from the introduction why user data is important
Careful analysis on Big Data can enhance business domain understanding for any company. Accordingly, Big Data has became a widely studied and interesting topic all around the field Information Technology in the recent years, both in scientific and industrial research \cite{inmon2007tapping, introtodatamining}. Significant amount of the data is somehow related to the users of the software at hand. Such data often incorporates essential data about the users themselves, such as their demographic data, preferences or how they interact with the software 
\cite{jang2015noreciprocity, hu2014we, jang2016teensengagemorewithfewerphotos, han2016teensarefrommars, socialdiversityongithub}. 

% how is the term user data used? 
Despite its value and availability, this kind of data is often not looked at, however some interesting studies have been conducted in the field which promotes its analysis \cite{jang2016teensengagemorewithfewerphotos}. Although being widely present, there is no common agreement in the field what user data means. To address RQ1 of this research, user data is defined as follows. 

% what is user data? what does it include or exclude?
The term user data in this study covers two major kind of data that is related to users. On one hand it covers data, which is willingly uploaded and generated by users, on the other hand, it consists of data which generated through interactions while using software solutions (typically online).
% demographic characteristics
The first part of user data typically consists of the demographic characteristics of the users and the content which they have generated \cite{han2016teensarefrommars}. This kind of data often can be found on social media profiles where social demographic information, such as gender, age and nationality are shared.
% digital footprints
Secondly, data gathered during the usage of a software is an important source of information for scientific studies. Researchers use the term digital footprint \cite{youyou2015computer} to describe this kind of data, which composes the second type of user data in this research. Such digital footprints can vary a lot depending on the software of discussion, but typically the include like activities, user comments, photos, posts on social media, ratings in movie databases or browsing history and usage patterns for web browsers. Figure \ref{user_data_venn} demonstrates how user data is understood in this study.

\begin{figure}[h] 
  \begin{center}
    \includegraphics[width=3in]{Images/user_data_venn.png}
    \caption{User data in this research is the combination of users' demographical characteristics and digital footprints.}
    \label{user_data_venn}
  \end{center}
\end{figure}

\subsection{Related research}
% explain that these are often not studied together but rather separately 
The concepts of demographical characteristics and digital footprints are widely studied in the research field, but often only separately, focused on one certain area. Demographic characteristics play a role in analyzing user adoption and behavior, for instance in the banking industry. The study conducted by Wang and Petrounias \cite{chinesemobilebankingusers} reveals that mobile banking in China is more popular among middle-aged males, while the younger generation has not adopted to the new trends yet. By utilizing Big Data analytics the group of citizens and products for the upcoming marketing campaigns were revealed \cite{chinesemobilebankingusers}, which greatly enhances the marketing activities of financial organizations. Social diversity was also studied by other researchers in the context of software development growth \cite{socialdiversityongithub}. In their study, Aué et al. \cite{socialdiversityongithub} have clearly identified correlation between the success of open source projects and the contributors' gender and cultural diversity by utilizing well chosen statistical methods. 

Like activities performed on social media are widely studied concept \cite{bakhshi2014faces, jang2015noreciprocity, jang2016teensengagemorewithfewerphotos, ottoni2013ladies}. Comments, hashtags and content that is generated by users are also widely studied \cite{bakhshi2014faces, jang2016teensengagemorewithfewerphotos, hu2014we, bakhshi2014faces} and also contribute to this kind of data in this study. However, little research has been conducted in the field on how these usage patterns can be projected onto demographical data, which can greatly contribute to understand user behavior of certain target groups. Despite the fact the term user behavior is widely studied, researchers claim that its concept is used rather ambiguously \cite{waheed2017investigation}.
%Furthermore in case of social networks, the users' contribution to the available content is essential, which is often called as user-generated content (UCG) \cite{jang2016teensengagemorewithfewerphotos}. 

% what kind of results were derived in the past from end-user data analysis? 
Movie databases often contain user reviews on movies, actors and producers of all sort. Such databases are open and are available for the public, and therefore the amount of data has grown huge over the past years. Unsurprisingly, databases like the Internet Movie Database (IMDB) has drawn the interest of researchers \cite{saraee2004data, kabinsingha2012movie, sumathi2013performance}. The successful application of statistical methods and data mining techniques have revealed interesting findings, such as that larger budget for movies does not necessarily result in good ratings by the public, while actors have higher impact on the opinion of the audience \cite{saraee2004data}. On top of deriving such conclusions, machine learning techniques are emerging to predict future movie rating data, based on prior reviews of users \cite{saraee2004data} or the analysis of genre and other attributes of movies \cite{kabinsingha2012movie}.

% how do researches on social media website data see the user data analysis?  
Social Networking Sites (SNSs) are another trending source in discovering the secrets of user data as the number of scientific publications in the topic has increased significantly in the recent years \cite{waheed2017investigation}. Various researches have applied advanced data mining techniques on Instagram data \cite{jang2015noreciprocity, bakhshi2014faces, hu2014we, jang2016teensengagemorewithfewerphotos, han2016teensarefrommars}, more specifically on the tags and comments that are attached to the images. Similarly, like activities and user-generated content is studied by the scientists. It is revealed, that Instagram users can be divided into two groups based on their activities: specialists, who publish and seek content around a certain topic of interest; and generalists, who are interested in all kinds of genres in the social media site \cite{jang2015noreciprocity}. Data mining techniques also allowed researchers to conclude, that the teenager users of Instagram tend to be more active, faster to react and more open to communicate with other users on social media \cite{jang2016teensengagemorewithfewerphotos, han2016teensarefrommars}. Furthermore, it was discovered that media content with human faces are more engaging than other type of media \cite{bakhshi2014faces}. Finally, rich social media data allowed researchers to analyze behavior and user preferences among genders, age groups and locations \cite{farseev2015harvestingmultiplesources}.

% facebook reactions
Recently Facebook has introduced reactions among their features. Through reactions, users can not only "like" content, but also express other emotions, such as love, joy, amazement, anger or sadness \cite{shouldfacebookusereactions, howarenewspublishersreactingonfacebook}. This way users' emotional feelings about the content can be collected easily and efficiently. Shortly after its release, it was identified that the new feature is very popular and generally engages a wider audience than previous likes and comments \cite{shouldfacebookusereactions}. Study also shows, that reactions are a great way for publishers to get a feedback on the public's opinion \cite{howarenewspublishersreactingonfacebook}. On top of that, reactions offer an easy way for content providers to organize a by assigning one of the available reactions to the participants and asking the users to react on the content with their favorite's reaction \cite{shouldfacebookusereactions}. The limitation on such polls is that users have the possibility to choose only one of the reactions and hence only one of the participants as their answer to the voting.  

% where is all this research coming from?
Interestingly, most of the SNS-related studies are conducted in the United States of America and Asia \cite{waheed2017investigation}. Only a few studies were conducted in the European region, which allows to conclude that there is a great interest and development possibility in the area. However it is important to highlight, that due to the wide popularity of the international social networks (such as Facebook, Instagram or Pintrest), some part of the data may be derived from users in the European continent. It was also pointed out that some researches focus on sites, that are specific to a particular region or country \cite{waheed2017investigation}, which means the findings are strongly related to the cultural environment of the user base.    

% what does user data analysis reveal in a corporate social media environment? [refer to \cite{guy2016whatsyourorganizationlike}]
Social media platforms in enterprise environment are studied similarly to regular social networks \cite{guy2016whatsyourorganizationlike}. Despite the fact that the two types of social media platforms share many features, analysis performed in enterprise environment can provide great insights about how employees interact and cooperate. Among many other findings, studies have shown that blogs posts in an enterprise social media site tend to be more engaging and contribute to form communities inside the organization \cite{guy2016whatsyourorganizationlike}. Such insights are essential for higher management, because it can be used for instance to identify departments that tend to be less interactive or engaged.  

\subsection{Tools and methods}
Various methods were utilized in previous reserch to conclude the findings above. The major categories of the techniques and their purposes are explained in the paragraphs to follow. Table \ref{table_of_techniques} below lists the same findings.

% what methods are utilized? 
  % multiple data sources
   The combination of multiple data sources, such as Facebook, Instagram, Pintrest or other social media profiles via their connection is a common practice in researches \cite{farseev2015harvestingmultiplesources, ottoni2013ladies}. Researchers have managed to perform complete demographic profiling and concluded that the intergration of multiple data sources is indeed a great method of enhancing performance on user data analysis \cite{farseev2015harvestingmultiplesources}. 
  
  Utilizing additional tools, existing datasets can be further enhanced for data analysis purouses. A great specimen for this is computer vision, which is used in numerous studies to identify content of photos \cite{hu2014we, farseev2015harvestingmultiplesources}. In another study, similar techniques are utilized to identify faces on and predict ages of people from photos on social media sites \cite{han2016teensarefrommars, bakhshi2014faces}. Through these reserches, computer vision is proven to be a powerful tool to facilitate the information in the data and can provide researchers further data for their studies. 

  % text processing techniques
  Text processing techniques are potentially the most common way to discover the insights of user data. The Linguistic Inquiry and Word Count (LIWC) and Latent Dirichlet Allocation (LDA) methods are used by researchers \cite{ottoni2013ladies, farseev2015harvestingmultiplesources, jang2016teensengagemorewithfewerphotos} for extracting linguistic features of text, identifying keywords in comments and reviews as well as for topic modelling on user comments, reviews or hashtags on Instagram. Topic modelling was also applied for analyzing genres of movies by their title and description \cite{kabinsingha2012movie}. Applying natural language processing techniques also allowed researchers to infer age and gender of users \cite{han2016teensarefrommars} who have commented on content on Instagram. Utilizing these methods greatly contribute to the undertsanding of data gathered, which can lead to richer analysis and pattern recognition on the dataset at hand.
   
  % supervised ML techniques
  Supervised machine learning is often used for prediction tasks performed on user data. Decision trees are often utilized as prediction tools for the purposes of identifying audience of mobile banking software in China \cite{chinesemobilebankingusers}. Decision trees were also successfully used for movie rating prediction \cite{saraee2004data} and classificiation of movies \cite{kabinsingha2012movie}. Bayesian networks in the same study \cite{kabinsingha2012movie} are also investigated and are found as the most suitable method for predictions in the area of movie ratings. Ensemble Modeling, which is another supervised learning method, is successfully used for user profile learning on social media sites \cite{farseev2015harvestingmultiplesources}. Similarly, Support Vector Machines and Logistic Regression was utilized for predicting age group of users based on their social media activities \cite{han2016teensarefrommars}. Last, but not least Negative Binomial Regression is put in use to model like activities in researches \cite{jang2015no, bakhshi2014faces}.

  % unsupervised learning - less popular but still possible
  Unsupervised machine learning techniques are present in the literature. As an example, clustering is used by Saraee White and Eccleston \cite{saraee2004data} for detecting relationships between the ratings given on a movie and the year it was published. In another study \cite{hu2014we}, the k-means algorithm was used successfully to create five clusters of Instagram users based on the type of content they have uploaded to Instagram. These results share similarities with another study \cite{jang2015no}, where generalist and specialits groups were identified based on their like activities based on natural language processing techniques. 

% frequent itemset
Frequent Itemset Analysis is utilized by Ottoni et al. \cite{ottoni2013ladies} for user portfolio analysis and detecting differences among genders in their data. By utilizing this technique, the researchers have identified significant differences between the two genders' preferences in terms of online content \cite{ottoni2013ladies}. 

\begin{table}[!t]
  \centering
    \begin{tabular}{c||c}
      Technique & Number of studies \\ 
      \hline
      Computer vision & 3  \\
      Text processing & 5 \\
      Unsupervised machine learning & 2 \\
      Supervised machine learning & 5 \\
      Frequent Itemset Analysis & 1
    \end{tabular}
    \caption{The summary of utilized data mining techniques on user data in other researches.}
    \label{table_of_techniques}
\end{table}