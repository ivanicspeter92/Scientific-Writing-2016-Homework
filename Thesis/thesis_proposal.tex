\documentclass[english]{tktltiki}
\usepackage[pdftex]{graphicx}
\usepackage{subfigure}
\usepackage{url}
\begin{document}
%\doublespacing
%\singlespacing
\onehalfspacing

\title{Master thesis work proposal}
\author{P\'eter Ivanics}

\date{\today}

\maketitle

\numberofpagesinformation{\numberofpages\ pages + \numberofappendixpages\ appendices}
%\keywords{}
%
%\begin{abstract}
%
%\end{abstract}
%
%\mytableofcontents

\section{Topic}
	% context, setting - B2C business gathers a lot of data
	Many software businesses collect enormous amount of data generated by end-users. End-user data incorporate essential information about how users interact with the system of discussion as well as tell a lot about the users themselves. Due to the size of the data, a significant part of the information might remain hidden and therefore the business-critical information remains unseen. 
	
	% what is the potential in analyzing user data? why is it important not only to collect but also analyze data?
	Data analysis tools and methods already exist to help businesses operate on their data. However, these applications often not utilized, because companies rather focus on developing their service package than understanding the previously gathered data \cite{inmon2007tapping}. As a result, essential knowledge concerning previously collected data is neglected, which would be key for the future development of the business. Careful data analysis may reveal interesting relationships and point out facts, what human eyes would never notice otherwise. Consequently, Knowledge Discovery in Databases on user data is important and is essential part of Business Intelligence applications. 

\section{Goals and methods}
    % goals
	The concepts and principles of Business Intelligence (BI) and Knowledge Discovery in Databases (KDD) are discussed briefly \cite{data_mining_and_knowledge_discovery}. This research work explains how KDD and BI tools can help businesses to understand their domain better. This research analyzes what data mining methods and techniques have proven to be the useful in the setting of BI and KDD \cite{bose2001business}. The work aims to analyze practices and previous case studies how it is possible to discover hidden patterns in databases to understand more knowledge about users and their data. 
	
	The major part of the work will explain the development of data analysis tools at Choicely as a case company. The nature of the data at the company is analyzed from the point of view of data analysis. The findings will be put in relation to the business domain of the firm. The business cases for which data mining techniques can be used at the company are introduced, such as:  
	
	\begin{itemize}
		\item customer engagement analysis,
		\item correlation analysis between user demographics and expressed opinion, 
		\item prediction of future data,
		\item identifying fraudulent usage, 
		\item visualization of the findings.
	\end{itemize}
	
	Choicely is developing a voting platform for media publishers and advertisers, which is used by thousands of users worldwide. The analyzed data consists of basic user profiles, contests/sweepstakes (for example talent shows, beauty contests, sport events), the votes spent on the contests and purchases made in the platform. 
		
	An expected outcome of the thesis work is a software, which can automatically detect and track hidden patterns and changes in the data. The software would be able to visualize the data in a way it is easy to interpret and therefore helps managers understanding the composition of their targeted audience. 
		
	For example, an interesting development area would be to find correlation between the contestants' images, the amount of received votes and the voters' demographic composition. In case the research area seems to be too wide, the analysis could be reduced to location areas, brands or contest categories in Choicely's platform. 
	
    % methods
    The basis of the theoretical framework will be retrieved through review of relevant literature. The practical side of the research will include comparison of multiple approaches to solve the tasks, description of the development process, evaluation of the results and analysis of the findings. 
        
    The research is oriented towards topics in computer science and data mining over business-related topics. References include material on practices in the field of data mining \cite{witten2016data, introtodatamining, inmon2007tapping, zarsky2002mine, bose2001business, data_mining_and_knowledge_discovery, monetisingusergeneratedcontent, datamininginbusinessprocesses, data_mining_in_educational_science}, but some references from social science-related research may be included as well \cite{kahneman1982psychology, pang2008opinion}.
    
\nocite{*}
\bibliographystyle{tktl}
\bibliography{references}

\lastpage

\appendices

\pagestyle{empty}

%\internalappendix{1}{Model ABC}
%
%The appendices here are just models of the table of contents and the presentation. Each appendix 
%usually starts on its own page, with the name and number of the appendix at the top. Each appendix is paginated separately.
%
%In addition to complementing the main document, each appendix is also its own, independent entity. 
%This means that an appendix cannot be just an image or a piece of programming, but the appendix must explain its contents and meaning.

\end{document}


