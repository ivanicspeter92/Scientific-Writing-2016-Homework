\documentclass[english]{tktltiki}
\usepackage[pdftex]{graphicx}
\usepackage{subfigure}
\usepackage{booktabs}
\usepackage{amsthm,amssymb}
\usepackage{amsmath}
\usepackage{url}
\usepackage{paralist}
\usepackage{enumitem}
\usepackage{totcount}

\newtotcounter{citnum} %From the package documentation
\def\oldbibitem{} \let\oldbibitem=\bibitem
\def\bibitem{\stepcounter{citnum}\oldbibitem}

\begin{document}
\onehalfspacing

\title{Analysis of audience engagement and expressed opinion with Data Mining}
% \title{User data analysis with Data Mining techniques}
% \title{Discovering the public's engagement and opinion with Data Mining}
% \title{User data and audience engagement analysis with Data Mining - the case of the Choicely voting platform}
\author{P\'eter Ivanics}

\date{\today}

\maketitle

\numberofpagesinformation{\numberofpages\ pages + \numberofappendixpages\ appendices}
\keywords{data mining, association analysis, user data}

\begin{abstract}
%	1.  What is the problem? <Context>
Many software businesses collect enormous amount of data generated by end-users. End-user data incorporates essential information about how users interact with the system of discussion as well as information about the users themselves. For instance, such digital footprints can explain the preferences of users, what kind of content they like, how they interact with the system and eachother or how frequent their activity in the platform is.
%	2.  Why is this problem important? <Objectives>
The goal of this research is to study understand the potential use of user data in business and research environments in the present time with the application of Data Mining methods. Since there is no common understanding on its concept, user data in this research consists of data willingly uploaded by users and generated through their interactions while using the software. The research aims to study how content on the Internet is observed by the users, who access the content. Secondly, it is studied what kind of feedback is received from the engaged audience in terms of like activities and comments to the provided content. Thirdly, the tendencies among the demographic characteristics between users are studied in relation to the content which gets them engaged online.
%	3.  What does your solution achieve? <Methods and objectives>
The foundations for the theoretical basis are retrieved by reviewing relevant literature. The sources include scientific journals, articles as well as books related to the topic of this study. The two former sources are retrieved through online digital libraries, such as the ACM digital library, IEEE Xplore and Google Scholar. After finding the first papers, the snowballing technique was used to retrieve further papers in the field. Furthermore, various keywords were used to obtain related literature in the research area.
%	4.  What follows from your solution? <Results>
The research papers in the field show various ways how user data can be used. Social networking sites and movie databases commonly utilize various statistical and data analysis tools on user data. Results show for instance, that Instagram users typically can be divided into two groups -  generalists and specialists - based on the variety of the content they like in the software. Studies about social media sites in enterprise environments have given insights to the organization on how their employees cooperate and interact by utilizing a set of data mining tools. 
The related studies in the research field demonstrate the relevance of studying user data, both in commercial as well as research environments. By utilizing data mining techniques and statistical measures, researchers can learn the insights of the user data and look at human behavior from a different, novel angle.  
\end{abstract}

\mytableofcontents

	\section{Introduction}
	\label{section::introduction}
	\subsection{Background and motivation}
    % context, setting - B2C business gathers a lot of data
	Many software businesses collect enormous amount of data generated by end-users. End-user data incorporate essential information about how users interact with the system of discussion as well as tell a lot about the users themselves. Due to the size of the data, a significant part of the information might remain hidden and therefore the business-critical information remains unseen \cite{inmon2007tapping, wegener2010integrating, introtodatamining}. As a result, there is a growing need for all businesses to introduce data analysis processes in their daily activies with the aim of understanding users and their generated data better.  
	
	% what is the potential in analyzing user data? why is it important not only to collect but also analyze data?
	Data analysis tools and methods already exist to help businesses operate on their data. However, these applications often not utilized, because companies rather focus on developing their service package than understanding the previously gathered data \cite{inmon2007tapping}. As a result, essential knowledge concerning previously collected data is neglected, which would be key for the future development of the business. Careful data analysis may reveal interesting relationships and point out facts, what human eyes would never notice otherwise. Consequently, Knowledge Discovery in Databases on user data is important and is essential part of Business Intelligence applications \cite{zarsky2002mine}. 

    % how do mobile devices have an impact on the amount of data? 
    With the continuous increase of mobile devices, the amount of data increases significantly. Due to the wide availability and commonness of smart phones and tablets, anybody can easily generate rich location, media or textual data. This fact further increases the call for research in the field of mobile and user data analysis, because it can help researchers to understand the society, human behavior, preferences and public opinions better.

    % what kind of results were derived in the past from end-user data analysis? 
    Depending on the portfolio of the business, analysis of end-user data can reveal various interesting findings. For example, movie databases often contain user reviews on movies, actors and producers of all sort. Such databases are open and are available for the public, and therefore the amount of data has grown huge over the past years. Unsurprisingly, databases like the Internet Movie Database (IMDB) drawn the interest of researchers \cite{saraee2004data, kabinsingha2012movie, sumathi2013performance}. The successful application of statistical methods and data mining techniques on such revealed interesting findings, such as that larger budget does not necessarily result in good ratings by the public, while actors have higher impact on the opinion of the audience \cite{saraee2004data}. On top of deriving such conclusions, machine learning techniques are emerging to predict future movie rating data, based on prior reviews of users \cite{saraee2004data} or the analysis of genre and other attributes of movies \cite{kabinsingha2012movie}.

    % how do researches on social media website data see the user data analysis?  
    Social media data analysis is another trending topic in discovering the secrets of user data. Various researches have applied advanced data mining techniques on Instagram data \cite{jang2015noreciprocity, bakhshi2014faces, hu2014we, jang2016teensengagemorewithfewerphotos, han2016teensarefrommars}, more specifically on the tags and comments that are attached to the images. Similarly, like activities and user-generated content is studied by many scientists. It is revealed, that Instagram users can be divided into two groups based on their activities: specialists, who publish and seek content around a certain topic of interest; and generalists, who are interested in all kinds of genres in the social media site \cite{jang2015noreciprocity}. Data mining techniques also allowed researchers to conclude, that the teenager users of Instagram tend to be more active, faster to react and more open to communicate with other users on social media \cite{jang2016teensengagemorewithfewerphotos, han2016teensarefrommars}. Furthermore, it was discovered that media content with human faces are more engaging than other type of media \cite{bakhshi2014faces}. Finally, rich social media data allowed researchers to analyze behavior and user preferences among genders, age groups and locations \cite{farseev2015harvestingmultiplesources}.

    % what does user data analysis reveal in a corporate social media environment? [refer to \cite{guy2016whatsyourorganizationlike}] 
    Social media platforms in enterprise environment are studied similarly to regular social networks \cite{guy2016whatsyourorganizationlike}. Despite the fact that the two types of social media platforms share many features, analysis performed in enterprise environment can provide great insights about how employees interact and cooperate. Among many other findings, studies have shown that blogs posts in an enterprise social media site tend to be more engaging and contribute to form communities inside the organization \cite{guy2016whatsyourorganizationlike}. Such insights are essential for higher management, because it can be used for instance to identify departments that tend to be less interactive or engaged.

    % structured and unstructured data is being investigated by different researchers

    % why are these researches interesting? What do we learn from the society by performing analysis on user data? 
    The brief examples above demonstrate the capabilities of data mining in the field of user data analysis. The proper application of data mining methods allow us to learn more about the society as well as human psychology, which was not possible in the past. This information is essential for business operations, because it gives insights on the user groups, their preferences and what kind of content keeps the audience engaged. In the past these kind of insights were unavailable to managers and content providers. In the present time, access to such information can facilitate business processes, help determining the future content, analyze trends and understand the target group of particular services better. As a result, modern data mining techniques created the potential to study human preferences and behavior from a different angle, which was not possible in the past. 

    % what is the focus? 
    This thesis work focuses on...

    % what is the motivation behind the focus? 

\subsection{Research scope and objectives}


\subsection{Thesis structure}
    % how is the thesis structured? 

    % what is in each of the sections to follow? 

	
	\section{Audience engagement and user data analysis}
	\label{section::audience-engagement-and-user-data-analysis}
	% catch up the thread from the introduction why user data is important
As outlined in the previous section, careful analysis on Big Data can enhance business domain understanding for any company. Studying Big Data has became a widely studied and interesting topic all around the Information Technology industry in the recent years. Unsurprisingly, significant amount of the data is somehow related to the users of the software at hand. Such data often incorporates essential data about the users themselves, such as their demographic data, preferences or how they interact with the software.      

% what kind of results were derived in the past from end-user data analysis? 
For example, movie databases often contain user reviews on movies, actors and producers of all sort. Such databases are open and are available for the public, and therefore the amount of data has grown huge over the past years. Unsurprisingly, databases like the Internet Movie Database (IMDB) has drawn the interest of researchers \cite{saraee2004data, kabinsingha2012movie, sumathi2013performance}. The successful application of statistical methods and data mining techniques have revealed interesting findings, such as that larger budget for movies does not necessarily result in good ratings by the public, while actors have higher impact on the opinion of the audience \cite{saraee2004data}. On top of deriving such conclusions, machine learning techniques are emerging to predict future movie rating data, based on prior reviews of users \cite{saraee2004data} or the analysis of genre and other attributes of movies \cite{kabinsingha2012movie}.

% how do researches on social media website data see the user data analysis?  
Unsurprisingly, social media data analysis is another trending topic in discovering the secrets of user data. Various researches have applied advanced data mining techniques on Instagram data \cite{jang2015noreciprocity, bakhshi2014faces, hu2014we, jang2016teensengagemorewithfewerphotos, han2016teensarefrommars}, more specifically on the tags and comments that are attached to the images. Similarly, like activities and user-generated content is studied by many scientists. It is revealed, that Instagram users can be divided into two groups based on their activities: specialists, who publish and seek content around a certain topic of interest; and generalists, who are interested in all kinds of genres in the social media site \cite{jang2015noreciprocity}. Data mining techniques also allowed researchers to conclude, that the teenager users of Instagram tend to be more active, faster to react and more open to communicate with other users on social media \cite{jang2016teensengagemorewithfewerphotos, han2016teensarefrommars}. Furthermore, it was discovered that media content with human faces are more engaging than other type of media \cite{bakhshi2014faces}. Finally, rich social media data allowed researchers to analyze behavior and user preferences among genders, age groups and locations \cite{farseev2015harvestingmultiplesources}.

% what does user data analysis reveal in a corporate social media environment? [refer to \cite{guy2016whatsyourorganizationlike}] 
Social media platforms in enterprise environment are studied similarly to regular social networks \cite{guy2016whatsyourorganizationlike}. Despite the fact that the two types of social media platforms share many features, analysis performed in enterprise environment can provide great insights about how employees interact and cooperate. Among many other findings, studies have shown that blogs posts in an enterprise social media site tend to be more engaging and contribute to form communities inside the organization \cite{guy2016whatsyourorganizationlike}. Such insights are essential for higher management, because it can be used for instance to identify departments that tend to be less interactive or engaged.  

% structured and unstructured data is being investigated by different researchers

% why are these researches interesting? What do we learn from the society by performing analysis on user data? 
The brief examples above demonstrate the capabilities of data mining in the field of user data analysis. The proper application of data mining methods allow us to learn more about the society as well as human behavior, which was not possible in the past. This information is essential for business operations, because it gives insights on the user groups, their preferences and what kind of content keeps the audience engaged. In the past these kind of insights were unavailable to managers and content providers. In the present time, access to this information can facilitate business processes, helps determining the future content, analyzing trends and understanding target groups of particular services better. In sum, modern data mining techniques created the potential to study human preferences and behavior from a different angle. 

Clearly, there is a growing need for data mining applications that operate on user data. Social media sites are living their golden age in the present time, which closes the distance between businesses and their customers by creating a new channel of communication. Many software solutions are integrating authentication of user profiles via social media sites to their service packages. From the users' perspective this integration is convenient, because they can register and authenticate themselves with their social media credentials. From the viewpoint of the businesses gathering user data was never easier, because such action allows easy access to the public information of users.	

	% \section{Data mining}
	% \input{data-mining}
	
	\section{Data mining at Choicely}
	\label{section::data-mining-at-choicely}
	\subsection{Introduction to the Choicely voting platform}
    % what does the company do? 
    Choicely is a voting platform that... The platform has already hosted numerous contests in various fields, such as beauty pageants, public polls, design contests, sport events among many others. The customer base of the firm consist of mainly Finnish broadcasters, publishers and advertisers. The customer and user base is constantly growing with many international companies from all around the world.   
    
    % what are the contests like, what kind of configuration settings are available?
    The contests in the platform are...

    % what kind of data is generated?
    The available data is two-folded: each user has a user profile, which contains basic demographical information about the user (gender and location), while contests have a number of participants with arbitrary number of votes that the users have spent on them. 

    % how does the data look like and how can it be retrieved?
    The data is structured in nature and is stored in a... The data can be accessed via...

    %  why is it important to analyze this kind of data and what can we learn from it? 
    There is a need to analyze this data, because...

    % why is the data analysis relevant from scientific research point of view?
    Performing scientific research on such data is interesting for multiple reasons...

\subsection{Research setting}


\subsection{Results}


\subsection{Evaluation}

	
	\section{Discussion}
	\label{section::discussion}
	% variance in user data (especially in the digital footprints)
Based on the reviewed literature, that there is no common understanding on the term of user data among researchers. While demographic data is very well understood, digital footprints (data gathered during the usage of a software) are less commonly studied in researches. It is clearly identified that the careful utilization of user demographic data and digital footprints software service providers and researchers can retrieve previously unknown information about users' behavioral characteristics. 

% no common understanding 
Users' demographic data and digital footprints are not usually brought under the same hood, are often vaguely defined and sometimes are studied separately. By introducing the concept of user data, a common ground of understanding is set for researchers. As a scientist in this area, it is essential to understand the variety of user data in terms of availability and format. Depending on the software at hand, the data generated by digital footprints varies, while user demographic data is more commonly understood.

Access to demographic data in the present time seems to be getting easier for researchers. Social networks already have standard ways of helping users to authenticate themselves while using other services (in other words use their social network credentials to use other services), businesses can get access to rich demographic data of their user base easily. This fact creates a research space for novel studies, that have not been possible in the past. 

% why are these researches interesting? What do we learn from the society by performing analysis on user data? 
The reviewed research papers demonstrate how others have utilized Data Mining and Machine Learning methods to reveal patterns or behavioral characteristics of users. The proper application these methods allow us to learn more about the society as well as human behavior, which was not possible previously. This information is essential for business operations, because it gives insights on the user groups, their preferences and what kind of content keeps the audience engaged. In conclusion, the interest and relevance of applying computational methods in this area is proven.

The set reviewed studies show, that Natural Language Processing and Supervised Machine Learning techniques are the most common approaches for conducting research on user data. The encouraging results achieved by other researchers prove the relevance of computational techniques applied on user data in a wide range of development areas, such as banking, social media, online movie databases, user portfolio analysis, to name a few. This suggests that computational methods are versatile enough to be applied in various domains. However, as every software and dataset is unique in some sense, there is not a single method that could be consdered as a silver bullet to solve problems. 

In the past such insights were unavailable for researchers and content providers. Audience engagement was not possible hardly possible and knowing the interests user groups was challenging. In the present time, access to this information can facilitate research, business processes, helps determining the future content, analyzing trends and understanding target groups of particular services better. In sum, modern Data Mining techniques created the potential to study human preferences and behavior from a different angle.

The results from studying the Choicely platform demonstrate an actual realization of the user data concept. The platform incorporates demographic data of its users as well as digital footprints of the software's usage. The nature of the platform is interesting, because users' engagement, voting trends and favor towards contest participants can be extracted from the data. Using the developed data analysis framework, contest organizers can analyze the engagement summary after a contest is over in a new way. 

The developed tools can not only pinpoint the content that is more appealing to users, but also reveal the tendencies shown by different groups of users. This provides the potential for developing customized services and content for the groups, which then leads to better experience. For example, if younger users show more interest in sports, while elders in traveling, customized content recommendations can be made to them, either in the Choicely platform or in external sites. Furthermore, this knowledge can also contribute to better product design and marketing, because the quantified results can confirm preliminary hypotheses concerning users' personal desires. 

% Evaluation of the chosen research methods
The chosen methods for the practical part of this study help the company to understand the data at hand better, as well as to provide better services to their customers. The methods used in this study are relevant tools towards answering the research questions, however their limitations were also demonstrated in the study. Hence it is important to know the pros and cons of these methods and think critically about their applicability and relevance to problems. Therefore, the chosen methods are retrospectively reviewed in the scope of this study in the paragraphs to follow.

EDA has provided great access to statistical measures as well as visualization tools to explore the data. Through this study it can be seen great technique for getting a grasp on what the data at hand looks like and what it contains. This method provides the possibility of laying down initial hypotheses and conceptualizing the "big picture". Looking at the data in general is a good idea not only to explore, but to understand how observations relate to eachother and what the connections between them are. Despite being a useful technique, EDA is limited to this only purpose and cannot answer complex questions, nor it can identify underlying structure or patterns in many cases. Last but not least, EDA can reveal also the potential biases in the data with the choice of proper visualization and statistical measures. In this study EDA for instance pinpointed the fact that many contests are considerably small in terms of voters and hence helped enhancing the results in later stages. Furthermore, the technique also revealed which type of contests in the platform have more data to work with, which led to better and valid results. 

Association Analysis... % TODO

Co-Clustering.... % TODO

% Privacy and ethical concerns % TODO
Privacy and ethical concerns...

\begin{enumerate}
    \item how ethical is it to collect this kind of data?
    \item privacy concerns
    \item how to communicate to users what their data is used for - how to give a succinct consent?
    \item personal data
    \item refer to GDPR 
    \item ask Hannu's and Aleksi's opinion
\end{enumerate}

% Threads to validity % TODO
Threads to validity? 


	\section{Conclusions}
	\label{section::conclusions}
	% sum up the main points in a paragraph
% what are the main messages and takeaways? 
% future work    

\pagebreak

% TODO remove the next line when the thesis is completed
[Overall \total{citnum}\ references]

\nocite{*}
\bibliographystyle{tktl}
\bibliography{references}

\lastpage
\appendices
\pagestyle{empty}

%\internalappendix{1}{Model ABC}
%
%The appendices here are just models of the table of contents and the presentation. Each appendix 
%usually starts on its own page, with the name and number of the appendix at the top. Each appendix is paginated separately.
%
%In addition to complementing the main document, each appendix is also its own, independent entity. 
%This means that an appendix cannot be just an image or a piece of programming, but the appendix must explain its contents and meaning.

\end{document}