This chapter presents the results achieved via applying the chosen methods introduced in Chapter \ref{section::methodology}. The results obtained via each method are listed under the following subchapters respectively. This chapter provides answers to RQ2 and RQ3 (Chapter \ref{section::introduction}).

\subsection{Exploratory Data Analysis}
\label{section::exploratory-data-analysis}
    % unique voters
    By the end of 2017, there is a total number of 573 contests in the platform. To identify what kind of content is more engaging (RQ2), first let us look at the amount of unique voters over the number of contests. Table \ref{user_engagement_in_contests} lists the key figures of the number of unique voters over contests in the platform. 

    It can be easily seen that most of the contests engage very small amount of users, as the median of the unique voter count for all contests is 3. This means that at least half of the contests have had only 3 users who actually voted for any of the participants. One of the reasons behind this is that the company did not establish a large user base yet. Therefore there are many users who created only one contest but never used the platform on the long run. Many of the contests serve only testing purposes, hence engage only a few users. Such records can create bias in the upcoming analyses, because their data does not represent realistic scenarios. 

    \begin{table}[H]
        \centering
        \begin{tabular}{l|c}
            \textbf{Measure} & \textbf{Value} \\
            \hline
            Mean & 447 \\
            Standard deviation & 2992 \\
            Min & 0 \\
            25th percentile & 1 \\
            Median (50th percentile) & 3 \\
            75th percentile & 22 \\
            Max & 54684
        \end{tabular}
        \caption{The basic statistical measures of unique voters over contests.}
        \label{user_engagement_in_contests}
    \end{table}
    
    For this reason, contests with less than 100 unique voters are excluded in the remainder of the analysis, because such observations are not representative. This dataset contains 166808 vote transactions by 145000 users over 81 contests, 1113 contest participants and 432 labels on the images recognized by Google Vision. For the remainder of the EDA, this filtered dataset is used.

    Figure \ref{user_engagement_in_contests-pruned} displays the same distribution for the filtered set of contests. In this figure contests with more voters are more apparent. The highest number of unique voters is close to 55 000 in one of the contests, the mean value ($\mu$ = 2525) and the standard deviation ($\sigma$ = 6804). These numbers mean that there is a large variance in the amount of engaged users in contests. From the data it cannot be clearly said which traits make a contest more attractive to users. Presumably the marketing activities done by the contest's organizers have strong impact on the size of the engaged audience. For instance news agencies have an established customer base already, who was supposedly targeted by these contests via the web widget provided by Choicely (Chapter \ref{section::introduction-to-the-choicely-voting-platform}). 
    
    The boxplot on the right side of the figure uses the 95 percentile (around 5200 unique voters), above which the outliers can be seen. It can be also seen that the most of the contests engage 260-2600 unique voters (as described by the first and the third quartiles). There are 6 large contests, from which the biggest have engaged 54684 voters. 

    \begin{figure}[h] 
        \begin{center}
            \includegraphics[width=1\textwidth]{Images/user_engagement_in_contests-pruned.png}
            \caption{The number of unique voters over contests after filtering out contests with less than 100 unique voters.}
            \label{user_engagement_in_contests-pruned}
        \end{center}
    \end{figure}
    
    The six large contests are worth investigating a bit more closely. Four\footnote{\url{https://choicely.com/contest/5ca98554-0f7d-11e7-9f0c-6f102a54d68d}}\footnote{\url{https://choicely.com/contest/fb112461-9000-11e6-9e28-87ebd7a21d0d}}\footnote{\url{https://choicely.com/contest/7425566e-8c8e-11e6-b8ce-2147b021362f}}\footnote{\url{https://choicely.com/contest/164f52c7-9df8-11e7-b3c9-d1a0f88250ad}}
    out of the six large contests were beauty pageants, labeled with the categories of "beauty", "fashion" and "entertainment". The two other contests\footnote{\url{https://choicely.com/contest/50819173-f838-11e6-b171-b949f18a4d21}}\footnote{\url{https://choicely.com/contest/4257ea9c-3e21-11e7-84ec-5f5a9bcfd190}}
    are listed only in the "other" category, which is certainly a mistake. By looking at the latter two contests, it can be easily seen that they would better belong to the "entertainment" and "sports" category. 
    
    In all six of the large contests, the contest participants were people: either sportsmen, celebrities or beauty queens/kings. It is interesting that none of the large contests have had objects, places or other intangibles as contest participants, although the platform has seen many of such participants previously. 

    In the next step, let us look at the distribution of contest categories in the filtered set of contests. It can be seen from the histogram on Figure \ref{contests_over_categories}, that the amount of "beauty", "entertainment", "sport" and "fashion" contests is considerably high compared to the rest of the categories. This finding is well aligned with the case company's profile at the point of conducting this study. As pointed out in Chapter \ref{section::introduction-to-the-choicely-voting-platform}, most of the company's customer base consits of Finnish broadcasters and advertisers. Hence there is no surprise in the distribution of the categories except for the "other" group. 
    
    \begin{figure}[h] 
        \begin{center}
            \includegraphics[width=.75\textwidth]{Images/contests_over_categories.png}
            \caption{The number of contests in each category.}
            \label{contests_over_categories}
        \end{center}
    \end{figure}
    
    By manually looking at contests in the "other" category it can be seen, that many (28 out of 34) of these contests are actually a sport-related. To name a few examples, there are contests with titles such as "Best player poll" ("Paras pelejaa aanestys")\footnote{\url{https://choicely.com/contest/5f8f8470-914d-11e6-bd5b-e571d894172f}}, "Who is the hottest driver?" ("Kuka on kuumin kuski?")\footnote{\url{https://choicely.com/contest/93d5f89c-5676-11e7-8cf7-0759c198269e}}, "Fastest driver of the race" ("Kisan nopein kuski")\footnote{\url{https://choicely.com/contest/bb7db707-3683-11e7-9f48-cbe34704f83e}}. This finding is not a suprise knowing the firm's customer base, but it also suggests the popularity of sports contests. If these contests were labeled correctly, sports contest would top the charts (Figure \ref{contests_over_categories}) with the highest bin around 50. Another interesting fact is, that in all of the sports contests, participants are athletes, hence the images contain human beings in these contests as well. 
    
    Three of the contests were related to the "travel" category, one to "fashion" and "beauty" and one to "entertainment". The error in this case is not as high as with sport contests, however correcting these category labels would facilitate data analysis in the future. For this reason, it is suggested to the company to review such issues, correct them manually and potentially prevent them happening in the future. 
    
    To answer the question of most engaging contest categories, the number of unique voters over contest categories is studied. Simple statistical measures (sum, mean, median and standard deviation) are calculated for the number of unique voters for each contest category. Table \ref{user_engagement_over_categories} displays the results.
    
    It can be seen, that "entertainment" and "beauty" contests cover the majority the amount of unique voters ($\approx 66.60 \%$ of the total). The values of these categories together are similar with the  "fashion" category. In these categories the mean and the median of the voters is also considerably high, which suggests their attractiveness. However, the high values in the standard deviation of the unique voters indicate that values are widely spread around the mean. The relevance of this observation is, that not all contests in these categories engage a large audience necessarily. Thus it can be concluded, that these categories tend to appear together and also attract a larger audience compared to the rest in general. 

    \begin{table}[]
        \centering
        \begin{adjustbox}{width=1\textwidth}
            \begin{tabular}{l|c|c|c|c|c}
                \textbf{Contest category} & \textbf{Sum of unique voters} & \textbf{Mean of unique voters} & \textbf{Median of unique voters} & \textbf{Standard deviation of unique voters} \\
                \hline
                beauty & 135866 & 3996 & 1482 & 9854 \\
                other & 61455 & 1807 & 1240 & 1831 \\
                entertainment & 161233 & 5374 & 1728 & 11772 \\
                sports & 15220 & 634 & 299 & 757 \\
                fashion & 75872 & 4742 & 874 & 12972 \\
                travel & 4451 & 1483 & 409 & 1719 \\
                humor & 1367 & 683 & 683 & 274 \\
                food & 767 & 767 & 767 & 0 \\
                danger & 958 & 958 & 958 & 0 \\
                games & 164 & 164 & 164 & 0
            \end{tabular}
        \end{adjustbox}
        \caption{The basic statistical measures of unique voters for each contest category.}
        \label{user_engagement_over_categories}
    \end{table}
    
    The categories "travel", "humor", "food", "danger" and "games" have hosted only a considerably low number of contests (Figure \ref{contests_over_categories}). Due to the small amount of data, it is difficult to derive any relevant results about the attractiveness of these categories at this point. 
    
    % It is suggested for the company to host and advertise more of these contests so that the public's opinion and engagement can be evaluated in these areas as well.

    % As the last part of the EDA, the attributes of highly rated contestants are investigated. In other words it is studied, what kind of traits make a contest participant more attractive to users. To answer this question, the podium finishers of the six large contests explained above are taken as examples. % TODO consider if needed

    % deriving results
    The above results contribute towards answering RQ2. The results allow to derive the following conclusions: 

    \begin{itemize}
        \item many of the contests are not labeled and hence belong only to the "other" category - fixing these labels manually could contribute towards better results,
        \item the "fashion", "beauty" and "entertainment" categories often appear together in contests,
        \item contests in the "beauty" and "entertainment" categories appear to be engaging to large audiences,
        \item contests where participants are human beings appear to be attractive to users,
        % \item there is no correlation between the unique voter count and the number of participants,
        \item the platform has hosted only a few contests in some of the categories and hence it is not possible to derive significant findings about the attractiveness of those contest at the moment.
    \end{itemize}

\subsection{Association Analysis}
The results of the Association Analysis are presented in this chapter are limited to the 1-itemsets and 2-itemsets in order to keep the findings easily understandable. The following figures and paragraphs display and explain the results. The itemsets in the figures are ordered by the variance of the values over the bins (however the variances are not displayed in the figures). This way the itemset with the highest variance is on the top, while the itemset with the smallest variance is on the bottom of the figure. All of figures in this chapter follow this convention of ordering. First the chosen Miss Suomi 2017 contest is analyzed, followed by the system-level analysis.

% ---------------------------- BEGIN OF GENDER ANALYSIS ---------------------------- 
% \subsubsection{Single-item itemsets}
Figure \ref{itemset_supports-gender-Miss_Helsinki-1_itemset} displays the 1-itemset supports and lifts for genders. The first finding to note is the \emph{\{"beauty"\}} and \emph{\{"dress"\}} itemsets with support and lift values 1.0 for all genders on the bottom of the figures. These values suggest that every single vote transaction in this contest has contained these itemsets. The reason behind this observation is that all of the participants' images had these labels. Therefore, inevitably all of the vote transactions picked up these itemsets and every vote in the contest has full support towards them. Such values do not show any significant meaning and therefore are excluded from the rest of the figures.

\begin{figure}[]
    \begin{center}
        \includegraphics[width=1.2\textwidth,center]{Images/itemset_supports-gender-Miss_Helsinki-1_itemset.png} 
        \includegraphics[width=1.2\textwidth,center]{Images/itemset_lifts-gender-Miss_Suomi-1_itemsets.png}
        \caption{The 1-itemset supports (top) and lifts (bottom) in the Miss Suomi 2017 contest by genders (all itemsets are displayed ordered by the variance of the values).}
        \label{itemset_supports-gender-Miss_Helsinki-1_itemset}
    \end{center}
\end{figure}

It is interesting to investigate, why all of the images have these two labels. As the participants' images were shot in a similar environment, with considerably similar content (e.g. the white dresses, the background, the hair styles, only one model in the image etc.), the Google Vision API assigned the same labels to the images. This gives a proof that computer vision in this case is more targeted towards getting an overall idea about the content of an image, rather than identifying its specific traits and attributes. 

Figure \ref{itemset_supports-age_group-Miss_Helsinki-1_itemset} provides a good possibility also to compare gender differences. For instance, the support \emph{S(X=\{"model"\}|gender=female)=0.45} is considerably higher than its male counterpart \emph{S(X=\{"model"\}|gender=male) = 0.26}. The lift values in this instance are \emph{L(X=\{"model"\}|gender=female) = 2.25} and \emph{L(X=\{"model"\}|gender = male) = 1.30}. Similar observation can be made for the \emph{\{"fashion model"\}} itemset, which suggests correlation between these two labels. This suggests two findings: the label "(fashion) model" appears to be more attractive to female voters compared to males and the presence of model in an image makes participants more appealing to all genders, but especially females. 

Similarly, the \emph{\{"photo shoot"\}, \{"sea"\}, \{"gown"\}} and \emph{\{"wedding"\}} itemsets have higher support in case of female voters. This may suggest that females are more likely to vote on images, where the object (the model and the dress) is put more into focus. The lift values however reveal, that the interest for \emph{\{"sea"\}, \{"gown"\}} and \emph{\{"photo shoot"\}} are higher for all groups, which proposes that the number of votes increases when sea is in the background. 

Another curious thing to note is the lift values of the \emph{\{"summer", "cloud", "beach"\}} itemsets, which are lower than 1.0 for both males and females. This observarion may suggest that having these in the background of the image is less attractive compared to the sea as background. Furthermore, \emph{\{"wedding"\}} has lift value less than 1.0 for genders other than females, which suggests less interest for this topic.  

Interestingly, the support and the lift for the \emph{\{"cloud"\}} and \emph{\{"summer"\}} itemsets are 0.0 for females, while the values of the male group are slightly higher. This means that no female users have voted on images with these labels (unless they belong to the \emph{not\_chosen} gender). This allows the conclusion that males might prefer images with more light and more colorful background in their votes. The lift and the support for \emph{\{"bride"\}} and \emph{\{"lady"\}} itemsets for males are also higher, which can mean that the model being a girl has more impact on their votes than her dress or the surroundings. However, as the lift value is not topping the charts for these itemsets, which suggests no major impact in users' behavior.

% ---------------------------- END OF GENDER ANALYSIS ---------------------------- 
% ---------------------------- BEGIN OF AGE GROUP ANALYSIS ---------------------------- 

Next, the 1-itemsets are calculated similarly in the same contest for age groups. Figure \ref{itemset_supports-age_group-Miss_Helsinki-1_itemset} displays results. It is interesting to notice that age groups 45-54 and 65+ have considerably higher support than the rest of the groups in case of the \emph{\{"shoulder"\}, \{"bridal clothing"\}} and the \emph{\{"wedding dress"\}} itemsets. The lift values for these itemsets are also higher for these groups which suggests their attractiveness. It seems, that the open-sholder wedding dresses displayed on the pictures raise the engagement of these groups. This may indicate also that dresses become influential in favoring a model in this kind of a contest for these groups. In other words, dress may be a more important aspect than the model herself.

\begin{figure}[] 
    \begin{center}
        \includegraphics[width=1.2\textwidth,center]{Images/itemset_supports-age_group-Miss_Helsinki-1_itemset.png}
        \includegraphics[width=1.2\textwidth,center]{Images/itemset_lifts-age_group-Miss_Suomi-1_itemsets.png}
        \caption{The 1-itemset supports (top) and lifts (bottom) in the Miss Suomi 2017 contest by age groups (only the first 15 itemsets are displayed ordered by the variance of the values).}
        \label{itemset_supports-age_group-Miss_Helsinki-1_itemset}
    \end{center}
\end{figure}

Another interesting observation is that the values for some itemsets, such as \emph{\{"body of water"\}, \{"beach"\}, \{"cloud"\}, \{"vacation"\}, \{"beach"\}} show 0.0 support and lift for the groups above age of 45. At the same time, these itemsets show somewhat higher support and lift by the rest of the age groups (0-17, 18-24, 25-34, 35-44). This finding may indicate that activities such as having vacation on a beach in a warm country is more in the interest of young people, which can be valuable information for travel agencies or marketing specialists. With such valuable information at hand, tourism offices could provide more customized advisory service for customers with different background. 

Strong difference can be observed in the lift values of \emph{\{"sea"\}} for the 65+ age group. The lift value is 0.0 in this case, while all the other groups have values over 1.0. This means that sea did not engage the elderly whatsoever, while triggered the interest the rest of the groups. It can be hence said in general, that sea appears to be an attractive trait in the background for most of the age groups but does not interest voters above 65 years of age. Another reason behind this observation could be that the amount of gathered votes from this age group is low, hence create bias in the results.  

% ---------------------------- END OF AGE GROUP ANALYSIS ---------------------------- 
% ---------------------------- BEGIN OF COUNTRY ANALYSIS ---------------------------- 

Finally, the 1-itemsets supports in the contest are analyzed by the location (country) of the users (Figure \ref{itemset_supports-country-Miss_Helsinki-1_itemset}). The first interesting observation is, that the support and lift of the USA group are 1.0 on some itemsets, while 0.0 on the others and there are no values in between. 

\begin{figure}[] 
    \begin{center}
        \includegraphics[width=1.2\textwidth,center]{Images/itemset_supports-country-Miss_Helsinki-1_itemset.png}
        \includegraphics[width=1.2\textwidth,center]{Images/itemset_lifts-country-Miss_Suomi-1_itemsets.png}
        \caption{The 1-itemset supports (top) and lifts (bottom) in the Miss Suomi 2017 contest by countries.}
        \label{itemset_supports-country-Miss_Helsinki-1_itemset}
    \end{center}
\end{figure}

By looking at the data the reason becomes obvious: the amount of transactions, where the users' country information is set as USA, Ukraine and Hungary are considerably low (below 5). When looking at the lift values this bias comes even more appealing for users in USA and Ukraine. This may not be a big surprise knowing that this contest was mainly advertised in Finland and not in other countries. As a matter of fact, there were 62 Finnish voters and 21 voters from the Philippines, which are still considerably low compared to the total number of voters. 

This finding suggests that many of the voters in this contest did not indicate their country information and hence do not display on Figure \ref{itemset_supports-country-Miss_Helsinki-1_itemset}. In other words, the amount of users, whose location information is filled is too low. Hence, the results of these groups are only the upper and lower extremes, which means bias in the data. It can be also said that this particular contest did not engage users across borders on a broad scale and therefore is not a good case for this kind of analysis.

Encouraging the users in the future to provide more information would allow to derive more relevant results. At the present time, no strong conclusions can be drawn from the location data of the voters in the Miss Suomi 2017 contest. To ensure the relevance of the support and lift calculations, it would be hence a good move to put a lower limit (e.g. 100 observations) on the number of voters, similarly as it was done in the EDA phase for number of unique voters in contests. For all of the above reasons, no more analysis on the country-based support and lift values is performed in the remainder of this study.

% ---------------------------- END OF COUNTRY ANALYSIS ---------------------------- 
% ---------------------------- BEGIN OF 2-ITEMSETS ---------------------------- 
\pagebreak
% \subsubsection{Multi-item itemsets}

The next step of the Association Analysis covers the investigation on k-itemsets, where the \emph{k}-value is higher than 1. Figure \ref{itemset_supports-gender-Miss_Helsinki-2_itemset} displays the support values for the 2-itemsets by genders. To enhance the readability of the figure, only the first 15 values are shown. 

\begin{figure}[h] 
    \begin{center}
        \includegraphics[width=1.2\textwidth,center]{Images/itemset_supports-gender-Miss_Helsinki-2_itemset.png}
        \caption{The 2-itemset supports in the Miss Suomi 2017 contest by genders (only the first 15 itemsets are displayed, ordered by the variance of the support values).}
        \label{itemset_supports-gender-Miss_Helsinki-2_itemset}
    \end{center}
\end{figure}

% closed frequent itemsets
An important observation to make is the identicality of the 2-itemsets, where \emph{"photograph"} is present. This is due to the fact that all of the pairing itemset \emph{Y} (i.e. \emph{\{"beauty"\}, \{"sea"\}, \{"photo shoot"\}} etc.) are always present on images, where \emph{"photograph"} is present. In other words, the confidence \emph{C("photograph"|Y) = 1.0}, where \emph{Y} is the itemset next to \emph{"photograph"}. When discussing k-itemsets, where $k \geq 2$, this issue may arise and provide identical support values for some combinations. When the complete list of 2-itemsets is plotted, the same observation can be made for many other itemsets, such as \emph{\{"bride"\}}, \emph{\{"shoulder"\}}, \emph{\{"body of water"\}}, etc. 

% address the problem of closed itemsets
To address this issue, such itemsets are combined and analyzed together. In the above example (Figure \ref{itemset_supports-gender-Miss_Helsinki-2_itemset}, itemsets with \emph{\{"photograph"\}}), the 2-itemsets with identical support values are merged into a single itemset, such that it contains all 1-itemsets respectively, i.e. \emph{\{"photograph", "beauty", "photo shoot", "gown", ..., "woman"\}} in this case. This way the combined itemset in this case has 10 items in overall. This is done repeatedly for all itemsets, which show the same support values. The lift values are then calculated for the resulting itemsets. For the sake of interestingness, the support values are not displayed here and only the lift values are analyzed. Figure \ref{itemset_lifts-gender-Miss_Suomi-multi_itemsets} displays the lift values for genders. Similar results for age groups are listed in Appendix 3, however they are not analyzed here.

\begin{figure}[]
    \begin{center}
        \includegraphics[width=1.2\textwidth,center]{Images/itemset_lifts-gender-Miss_Suomi-multi_itemsets+markup.png}
        \caption{The multi-item itemset lifts in the Miss Suomi 2017 contest by genders (all itemsets are displayed ordered by the variance of the lift values).}
        \label{itemset_lifts-gender-Miss_Suomi-multi_itemsets}
    \end{center}
\end{figure}

The results show that the lift values for females tend to be higher than the other groups in many cases. These suggests high interest towards the topics that appear in the itemsets compared to the rest of the groups. Such itemsets are marked with a blue \textbf{*} mark on the left side of the figure. As the number of items in the itemsets varies, these often share similarities, but they also describe certain topics. For instance, it can be seen that the items in the sets are telling about a lady wearing some sort of wedding/bridal dress. In other words, the items which explain the dress as such are more dominant and appear accross multiple itemsets. 

Other itemsets show more attraction from male voters. These are marked with an orange \textbf{\^} on the left side of the figure. In contrast, these itemsets describe not the dress as much as the model on the image and the background (sea, sky). It can be hence speculated that female voters might relate to the dress and the model (maybe think about themselves as being the models wearing the dress), while male voters might focus on the setting and the model in the image more.

Another interesting observation is that male voters' lift rarely goes to extremes, but usually is around the value of 1.0 in most cases. In contrast, the lift for the female group goes to extremes in both end for many more itemsets. This may suggest that females follow more of a specialist approach than males, who are more generalists. Another explanation behind this observation can be the fact that the male voters dominate the mean. This assumption is confirmed when the distribution of the genders is looked at: 66 \% (n=95) of the voters in this contest were males, while 33 \% (n=40) are females and 33 \% (n=41) did not provide gender information. Based on these results it seems, that females are strongly engaged towards certain topics in this contest while males are engaged accross multiple topics which appear in the contestants' images. 

% Another observation to be made on Figure \ref{itemset_supports-gender-Miss_Helsinki-2_itemset} is the behavior of the 1-itemsets, which had support of 1.0 on their own (\emph{\{"beauty"\}} and \emph{\{"dress"\}}, Figure \ref{itemset_supports-gender-Miss_Helsinki-1_itemset}). When these itemsets are paired with other itemsets, the support of the union is guaranteed to be the support of the set alongside. This implies also that any 1-itemset combined together with such itemsets has no impact on the support value. Therefore, 2-itemsets which include such label (that appears on all of the images) are non-informative and should be hidden. Another possibility is to study the lift instead of the support value, which in such cases can tell more about the value of such combinations.

% There is a total number of 150 2-itemsets in the image labels of the Miss Suomi contest, which is hard to analyze for the human eye. This number grows further with number of items in the set. To tackle this problem, itemsets with more than 1 item are not investigated the same way. Instead, the Co-Clustering approach is applied on the data to identify itemsets, that appear similar to eachother in their support. The details on this analysis and the results are explained in the next chapter. 

% ---------------------------- BEGIN OF SYSTEM-LEVEL ---------------------------- 
The final part of the Association Analysis has taken a look at itemset supports and lifts on the system level. That is, all of the vote transactions from contests with 100 unique voters were extracted from the system. This is the same dataset, which was used for the major part of the EDA in Chapter \ref{section::exploratory-data-analysis}, containing a total of 166808 votes over 81 contests. 

The biases in the data become appearent when the votes are studied on the system level. Most of the itemsets reflect on the great majority of "beauty" and "fashion" category contests, as itemsets such as \emph{\{"model"\}, \{"beauty"\}, \{"photo shoot"\}, \{"shoulder"\}} etc. From the itemset supports it seems that all of the itemsets were extracted from contests of these two category, as they clearly describe concepts that appear in such contests. This observation also supports the previous finding, that mainly contests in these categories were hosted in the platform with great success.

While studying the results it was identified, that the \emph{not\_chosen} group has considerably higher support over males and females in many cases. This can be explained by the fact that many users (49.82 \%) did not provide their gender information on their profile, which yields in a much higher number of observations (vote transactions) for users whose gender is unknown. 46.4 \% (77416 observations) from the the vote transactions were received from users with unknown gender. Likewise, only 27 \% (44918 observations) of the votes were received from females and 26.6 \% (44407 observations) from males. Similar observations can be made for the age group values, where 67 \% of the voters have not filled their age. 

It can be therefore seen that the sample size of the \emph{not\_chosen} gender and age group is dominant in the system-wide data. This creates a bias in the support values as the data for these group is more representative than for others. While this problem did not emerge on a single-contest level, when looking at all transactions, it becomes apparent. It could be assumed, that the distribution of the two genders is equal in this group, however there is no proof on this claim. For these reasons, strong claims and concrete findings are harder to derive from this data due to the bias explained above. The system-level data is therefore not analyzed, however some of the extracted figures are displayed in the appendices (Appendix 1 and Appendix 2). 


\subsection{Statistical significance}
An important aspect of evaluating the results is the concern of statistical significance. Calculating the support and lift values in a single contest provides an idea on the engagement measures, however the differences among groups in these measures on their own do not indicate statistical relevance. Stated differently, even if the lift and support values differ for two groups of users, that may not mean that the difference is statistically representative.

For this reason, the paragraphs to follow outline an approach towards evaluating the statistical significance in the comparison of the groups. In other words the following paragraphs present a way which can be used for the hypothesis testing on the behavioral differences of demographic groups in Choicely for itemsets. Some non-trivial challenges are also outlined as part of the discussion on this issue. 

The testing approach is introduced thorugh an example from the results of the Association Analysis. The \emph{\{"model"\}} itemset is taken as an example. The values to be presented are all obtained using the data gathered in this Miss Suomi 2017 contest. The reason behind choosing this itemset as an example is that it was discussed already in the preceeding paragraphs and the demographic groups have shown interesting differences in this particular case.

First the terms used and the null hypothesis are formulated. Let \emph{f(\{"model"\})} be the relative frequnecy of "model" and \emph{f(\{"model" | gender = g \})} is the relative frequency given that the voter's gender is \emph{g}. Let \emph{n} be the total number of votes, \emph{n(gender=g)} be the number of votes performed by voters in the \emph{g} gender. The number of "model" given that gender is male is binomially distributed under the null hypothesis. Using the gathered data, the above values are as follows:

\begin{center}
    $n=176$ \\
    $n(gender=male)=95$, \\
    $n(gender=female)=40$, \\
    $n(gender=not\_chosen)=41$, \\
    $f(\{"model"\})=\frac{52}{176}=0.3$, \\ % \frac{13}{44}=0.30$, \\
    $f(\{"model"\}|gender=male)=\frac{25}{95}=0.26$, \\ % =\frac{5}{19}=0.26$, \\
    $f(\{"model"\}|gender=female)=\frac{18}{40}=0.45$, \\ % \frac{9}{20}=0.45$, \\
    $f(\{"model"\}|gender=not\_chosen)=\frac{9}{41}=0.21$ % =\frac{9}{41}=0.21$
\end{center}

Let the null hypothesis \emph{$H_0$} be that gender \emph{g} does not differ from others genders in terms of voting behavior, i.e. 

\begin{center}
    \emph{$H_0$: f(\{"model" | gender = g \}) = f(\{"model"\})}. 
\end{center}

Let the alternative hypothesis \emph{$H_1$} be that \emph{g} differs from other genders, i.e. 

\begin{center}
    \emph{$H_1$: f(\{"model" | gender = g \}) $\neq$ f(\{"model"\})}. 
\end{center}

Using the values above and the binomial testing method, a two-sided test is performed. The chosen significance level is set to \emph{$\alpha$ = 0.05}. The obtained p-values for all genders are as follows:

\begin{center}
    $p_{male}=0.50$, \\
    $p_{female}=0.055$, \\
    $p_{not\_chosen}=0.30$
\end{center}

As a conclusion it can be said, that none of the obtained p-values indicate strong enough proof to reject the null hypothesis in every case. The studied groups in this particular case hence do not show statistically significant differences. The female group is however fairly close to the significance region. To provide a visual proof on this observation, Figure \ref{likelihood_distribution_females} displays the p-values over the number of votes on this itemset for females. It can be seen that the p-value at \emph{$n_{female}$=18} is just above the chosen level of significance. 

\begin{figure}[h]
    \begin{center}
        \includegraphics[width=0.75\textwidth]{Images/likelihood_distribution_females.png}
        \caption{The p-values over the number of successes (votes) for the \emph{\{"model"\}} itemset in the Miss Suomi 2017 contest for the female group.}
        \label{likelihood_distribution_females}
    \end{center}
\end{figure}

The statistical significance evaluation faces further challenges and possibilities. It would be interesting to compare for example results of multiple contests. Due to the fact that contests may have different voting rules, this is rather difficult to compare contests to one another. On top of that, the issue of synonyms in the image labels may rise if votes over multiple contests are analyzed simultaneously. For example, a model can not be only a fashion model/beauty peagant, but also a 3-dimensional model of a building. The same labels therefore may apper over contests with different meaning, which is again not sensible to compare. Last but not least, the sample size is still too small to perform statistically significant analysis over multiple contests.

\subsection{Co-Clustering}
The chosen Co-Clustering algorithm (introduced in Chapter \ref{section::methodology}) was first executed on a smaller set of 1-itemsets in the Miss Suomi 2017 contest. The clustering is performed for the supports calculated for 22 itemsets for both genders and age groups. The results are presented in the pragraphs to follow only by age groups. The analysis on the results by genders are not presented because there were no significant findings identified. This is mainly because the amount of data is smaller (\emph{22*3=66} values in the matrix) compared to the results by age groups (\emph{22*8=176} values in the matrix). 

Figure \ref{coclustering_miss-suomi-age_groups-1-itemsets} displays the matrix of itemset supports after the Co-Clustering for 3 clusters. Choosing the number of clusters was essentialy done by looking at the results and analyzing which k-value produces the most sensible results in this particular case. The colors correspond to the support of the itemset: the lighter values indicate smaller, the darker colors indicate higher support values. The clusters are circulated with different colors and displayed on the side and the bottom of the chart. 

\begin{figure}[h] 
    \begin{center}
        \includegraphics[width=0.75\textwidth]{Images/coclustering_miss-suomi-age_groups-1-itemsets-3_clusters_cropped.png}
        \caption{The results of the Co-Clustering in the Miss Suomi 2017 contest for 1-itemsets with 3 clusters by age groups.}
        \label{coclustering_miss-suomi-age_groups-1-itemsets}
    \end{center}
\end{figure}

As it can be seen on the figure, the clustering algoritm organizes the rows of the matrix based on the patterns their values show. For instance, Cluster 3 contains the itemsets, which have high support values for most of the age groups. This cluster in particular contains the more engaging labels, which supports the findings of the Association Analysis. The itemsets are clustered together, because they have similar support values accross age groups. It can be also seen that these itemsets have similar contextual meaning, for instance lady, girl and bride. Therefore it is not much of a surprise, that their support values are similar and hence are in the same cluster.

Cluster 2 in this case contains only 2 itemsets (\emph{\{"wedding"\}} and \emph{\{"model"\}}), which seemed to be the interest of the 45-54 age group. Similarly, Cluster 1 contains a list of itemsets, which show differences for the 55-64 group. These distinctions has became even more apparent when the algorithm was executed with 4 clusters: in this case the 65+ age group was grouped under its own cluster alone. Strong conclusions about these findigns cannot be made, however it seems that the clustering approach has the capability to reliably distinguish between the groups and itemsets, which show differences. The clustering may not be entirely correct and human revision might be needed, for instance the \emph{\{"bride"\}} itemset looks considerably different from the rest of the rows in Cluster 3.

Figure \ref{coclustering_miss-suomi-age_groups-multi-itemsets-4_clusters} displays the results for the multi-item itemsets. These are the same itemsets, which were analyzed during Association Analysis (created by combining Frequent Closed Itemsets, Figure \ref{itemset_lifts-gender-Miss_Suomi-multi_itemsets}). For the same reasons as above, the Co-Clustering is performed on age groups rather than genders. The matrix of support values is clustered to 4 clusters. 

\begin{figure}[] 
    \begin{center}
        \includegraphics[width=1\textwidth]{Images/coclustering_miss-suomi-age_groups-multi-itemsets-4_clusters_cropped.png}
        \caption{The results of the Co-Clustering in the Miss Suomi 2017 contest for multi-item itemsets with 4 clusters by age groups.}
        \label{coclustering_miss-suomi-age_groups-multi-itemsets-4_clusters}
    \end{center}
\end{figure}

Cluster 2 contains most of the itemsets that are supported by all of the groups. The rest of the age groups are also assigned under cluster 2, where itemsets are generally have higher supports for all of the groups. It can be hence said that the itemsets in Cluster 2 are in the interest of generally all of the groups, while itemsets outside this are more specific to the interest of only some groups. This can be useful information to marketers, whose aim is to target one of these groups with certain content. Another useful aspect in this visualization is, that the itemsets with various supports can be distinguished easily based on the color's depth. Identifying itemsets that belong together is considerably easier on the right side of the figure as the Co-Clustering algorithm puts these close to eachother. 

One of the most interesting observations in this case is that the age groups 35-44, 45-54 and 65+ are separated from the rest of the groups. This indicates that the interests of these groups are somewhat unique compared to the others. It can be also seen that the support values in groups 35-44 and 55-64 are considerably similar, not only in Cluster 3 but also outside of the clusters. It could be hence concluded, that some demographic groups follow a generalist, some others a specialist approach. In this particular case, users between the age of 0-34 are more specialists towards the itemsets in Cluster 2. Likewise, users from groups 55-64 and 65+ are unique, because some content (in clusters 1, 3 and 4) is engaging only to them. 

It is interesting to see that some cells do not belong to any of the clusters according to the algorithm's outcome. This particular co-clustering method assumes a block-diagonal structure, which might not be the desired outcome in this case. Approximately half of the cells were unclustered in both cases above, which may not be the optimal solution in all cases. After seeing the results above it could be said, that this structure might not be the best solution to the problem of identifying the structure in the dataset. Some interpretations can be made intuitively, however a more sophisticated co-clustering approach could reveal more insights of the data.

% It is interesting to see that there are many white "spots" in the matrix. In other words these are observations, where the support of the itemset for the particular age group is close to 0.0. For instance, there is quite many of such records in the 65+ age group. 

% ----- BEGIN SYSTEM LEVEL -----
Finally, the Co-Clustering is executed on the system level. The \emph{minsup} value is set to 0.05 for this analysis and the itemset size is not limited to any numbers. When performing the analysis for genders, the same observation can be made as during the Association Analysis, namely that the data from the \emph{not\_chosen} age and gender groups dominates the data, because of which this group has higher support values for most itemsets. Inevitably, this fact also has an impact on the behavior of the Co-Clustering algorithm. 

With 4 clusters and 166808 vote transactions, the results are as follows. All three genders (male, female and not chosen) are assigned to their own clusters with a subset of the itemsets in the platform. This findings suggests that the different genders tend to behave differently on the system level and there is a difference in which itemsets engage them. 

The majority of the itemsets (977 out of 1626) are assigned to the "not\_chosen" group, which is the consequence of the observation stated in the previous paragraph. Interestingly, the number of itemsets for females (148) is considerably lower than for males (482). This suggests that males have wider range of interests than females, if the data is analyzed on the system-wide level. There are 19 itemsets that are not assigned into the same cluster with any of the demographic groups, which may mean that these are on the same level of interest for all of the groups. 

When the system-wide analysis is performed for age groups with the same settings, the results are as follows. Similarly to genders, the users whose age is unknown are grouped under their own specific cluster with most of the itemsets (888 out of 2332). However, it is interesting to observe that age groups 0-17, 18-24, 25-34 and 45-54 are grouped under the same cluster with 314 itemsets. Age group 35-44 has its own cluster with 578 items and the 55-65 and 65+ age groups are again clustered together with 551 itemsets. 

It can be seen that there is a clear distinction between the younger and the older generation in the clustering. This finding suggests that the engaging content is different for these groups, while the middle-aged (34-54) users are somehow between them. The algorithm was executed with various number of clusters, but this behavior was always present. This observation further enhances the validity of this finding and suggests further investigation on the reasons in this area.