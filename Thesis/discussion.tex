% variance in user data (especially in the digital footprints)
Based on the reviewed literature, that there is no common understanding on the term of user data among researchers. While demographic data is very well understood, digital footprints (data gathered during the usage of a software) are less commonly studied in researches. It is clearly identified that the careful utilization of user demographic data and digital footprints software service providers and researchers can retrieve previously unknown information about users' behavioral characteristics. 

% no common understanding 
Users' demographic data and digital footprints are not usually brought under the same hood, are often vaguely defined and sometimes are studied separately. By introducing the concept of user data, a common ground of understanding is set for researchers. As a scientist in this area, it is essential to understand the variety of user data in terms of availability and format. Depending on the software at hand, the data generated by digital footprints varies, while user demographic data is more commonly understood.

Access to demographic data in the present time seems to be getting easier for researchers. Social networks already have standard ways of helping users to authenticate themselves while using other services (in other words use their social network credentials to use other services), businesses can get access to rich demographic data of their user base easily. This fact creates a research space for novel studies, that have not been possible in the past. 

% why are these researches interesting? What do we learn from the society by performing analysis on user data? 
The reviewed research papers demonstrate how others have utilized Data Mining and Machine Learning methods to reveal patterns or behavioral characteristics of users. The proper application these methods allow us to learn more about the society as well as human behavior, which was not possible previously. This information is essential for business operations, because it gives insights on the user groups, their preferences and what kind of content keeps the audience engaged. In conclusion, the interest and relevance of applying computational methods in this area is proven.

The set reviewed studies show, that Natural Language Processing and Supervised Machine Learning techniques are the most common approaches for conducting research on user data. The encouraging results achieved by other researchers prove the relevance of computational techniques applied on user data in a wide range of development areas, such as banking, social media, online movie databases, user portfolio analysis, to name a few. This suggests that computational methods are versatile enough to be applied in various domains. However, as every software and dataset is unique in some sense, there is not a single method that could be consdered as a silver bullet to solve problems. 

In the past such insights were unavailable for researchers and content providers. Audience engagement was not possible hardly possible and knowing the interests user groups was challenging. In the present time, access to this information can facilitate research, business processes, helps determining the future content, analyzing trends and understanding target groups of particular services better. In sum, modern Data Mining techniques created the potential to study human preferences and behavior from a different angle.

The results from studying the Choicely platform demonstrate an actual realization of the user data concept. The platform incorporates demographic data of its users as well as digital footprints of the software's usage. The nature of the platform is interesting, because users' engagement, voting trends and favor towards contest participants can be extracted from the data. Using the developed data analysis framework, contest organizers can analyze the engagement summary after a contest is over in a new way. 

The developed tools can not only pinpoint the content that is more appealing to users, but also reveal the tendencies shown by different groups of users. This provides the potential for developing customized services and content for the groups, which then leads to better experience. For example, if younger users show more interest in sports, while elders in traveling, customized content recommendations can be made to them, either in the Choicely platform or in external sites. Furthermore, this knowledge can also contribute to better product design and marketing, because the quantified results can confirm preliminary hypotheses concerning users' personal desires. 

% Evaluation of the chosen research methods
The chosen methods for the practical part of this study help the company to understand the data at hand better, as well as to provide better services to their customers. The methods used in this study are relevant tools towards answering the research questions, however their limitations were also demonstrated in the study. Hence it is important to know the pros and cons of these methods and think critically about their applicability and relevance to problems. Therefore, the chosen methods are retrospectively reviewed in the scope of this study in the paragraphs to follow.

EDA has provided great access to statistical measures as well as visualization tools to explore the data. Through this study it can be seen great technique for getting a grasp on what the data at hand looks like and what it contains. This method provides the possibility of laying down initial hypotheses and conceptualizing the "big picture". Looking at the data in general is a good idea not only to explore, but to understand how observations relate to eachother and what the connections between them are. Despite being a useful technique, EDA is limited to this only purpose and cannot answer complex questions, nor it can identify underlying structure or patterns in many cases. Last but not least, EDA can reveal also the potential biases in the data with the choice of proper visualization and statistical measures. In this study EDA for instance pinpointed the fact that many contests are considerably small in terms of voters and hence helped enhancing the results in later stages. Furthermore, the technique also revealed which type of contests in the platform have more data to work with, which led to better and valid results. 

Association Analysis... % TODO

Co-Clustering.... % TODO

% Privacy and ethical concerns % TODO
Privacy and ethical concerns...

\begin{enumerate}
    \item how ethical is it to collect this kind of data?
    \item privacy concerns
    \item how to communicate to users what their data is used for - how to give a succinct consent?
    \item personal data
    \item refer to GDPR 
    \item ask Hannu's and Aleksi's opinion
\end{enumerate}

% Threads to validity % TODO
Threads to validity? 
