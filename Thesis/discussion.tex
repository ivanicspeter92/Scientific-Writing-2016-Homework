% variance in user data (especially in the digital footprints)
Based on the reviewed literature, there is no common understanding on the term of user data among researchers. While demographic data is very well understood, digital footprints (data gathered during the usage of a software) are less commonly studied in the field. It is clearly identified that the careful utilization of user demographic data and digital footprints software service providers and researchers can retrieve previously unknown information about users' behavioral characteristics. 

% no common understanding 
Users' demographic data and digital footprints are not usually brought under the same hood, are often vaguely defined and sometimes are studied separately. By introducing the concept of user data, a common ground of understanding is set for researchers of the future. As a scientist in this area, it is essential to understand the variety of user data in terms of availability and format. Depending on the software at hand, the data generated by digital footprints varies, while user demographic data is more uniformed.

Access to demographic data in the present time seems to be getting easier for researchers. Social networks already have standard ways of helping users to authenticate themselves while using other services (in other words use their social network credentials to use other services). Taking advantage of this, businesses can get access to rich demographic data of their user base easily. This fact creates a research space for studies, that have not been possible in the past. 

% why are these researches interesting? What do we learn from the society by performing analysis on user data? 
The reviewed research papers demonstrate how others have utilized Data Mining and Machine Learning methods to reveal patterns or behavioral characteristics of users. The proper application data analysis methods allows us researchers, to learn more about the society as well as human behavior, which was not possible previously. This information is essential for business operations, because it gives insights on the user groups, their preferences and what kind of content keeps the audience engaged. In conclusion, the interest and relevance of applying computational methods in this area is proven.

The set reviewed studies show, that Natural Language Processing and Supervised Machine Learning techniques are the most common approaches for conducting research on user data. The encouraging results achieved by other researchers prove the relevance of computational techniques applied on user data in a wide range of development areas, such as banking, social media, online movie databases, user portfolio analysis, to name a few. This suggests that computational methods are versatile enough to be applied in various domains. However, as every software and dataset is unique in some sense, there is not a single method that could be consdered as a silver bullet to solve problems. 

In the past audience engagement was hardly possible and knowing the interests demographic user groups was challenging. In the present time, access to this information can facilitate research, business processes, help determining the future content, analyze trends and understanding target groups of particular services better. In sum, modern Data Mining techniques created the potential to study human preferences and behavior from a different angle.

The results from studying the Choicely platform demonstrate an actual realization of the user data concept. The platform incorporates demographic data of its users as well as digital footprints of the software's usage. The nature of the platform is interesting, because users' engagement, voting trends and favor towards contest participants can be extracted from the data. Using the developed data analysis framework, contest organizers can analyze the engagement summary after a contest is over in a new way. 

The developed tools can not only pinpoint the content that is more appealing to users, but also reveal the tendencies shown by different demographic groups. This provides the potential for developing customized services and content for the groups, which then leads to better user experience while interacting with the platform. For example, if younger users show more interest in sports, while elders in traveling, customized content recommendations can be made to them, either in the Choicely platform or in external sites. Furthermore, this knowledge can also contribute to better product design and marketing, because the quantified results can confirm preliminary hypotheses concerning users' personal desires. 

% Evaluation of the chosen research methods
The chosen methods for the practical part of this study help the company to understand the data at hand better, as well as to provide better services to their customers. The methods used in this study are relevant tools towards answering the research questions, however their limitations were also demonstrated in the study. Hence it is important to know the pros and cons of these methods and think critically about their applicability and relevance to problems. Therefore, the chosen methods are retrospectively reviewed in the scope of this study in the paragraphs to follow.

EDA has provided great access to statistical measures as well as visualization tools to explore the data. Through this study it can be seen great technique for getting a grasp on what the data at hand looks like and what it contains. This method provides the possibility of laying down initial hypotheses and conceptualizing the "big picture". Looking at the data in general is a good idea not only to explore, but to understand how observations relate to eachother and what the connections between them are. Despite being a useful technique, EDA is limited to this only purpose and cannot answer complex questions, nor it can identify underlying structure or patterns in many cases. Last but not least, EDA can reveal also the potential biases in the data with the choice of proper visualization and statistical measures. In this study EDA for instance pinpointed the fact that many contests are considerably small in terms of voters and hence helped enhancing the results in later stages. The finding that some contests did not have proper meta-data and labeling was also obtained through this research method. Last but not least, the technique also revealed which type of contests in the platform have more data to work with, which led to better and valid results. 

By performing Association Analysis some preliminary assumptions in the EDA phase have been confirmed. More importantly, the calculation of the itemset supports have contributed towards answering the research questions in more details. Based on the results it can be concluded, that contests in the "beauty" and "fashion" categories create higher engagement than others, which may be explained by the history and the customer base of the firm. Nevertheless, it seems that contests where participants are human beings are more attractive to voters, whereas abstract objects or places appear less interesting to the audience. 

Itemset supports on their own do not necessarily tell all of the interesting information about the engagement of the audience. The lifts of the itemsets are more interesting to look at and can enhance the provided information with more interesting and meaningful insights. 

It was identified, that studying the generated itemsets in a single contest is possible and can assist the organizer to retrospectively analyze the content which has engaged the audience. Using this data, different demographic groups and their preferences can be compared and analyzed. Due to the fact that the biases in the data enlarge on the system level, it is difficult to derive valid results using the current methods and data on a larger scale. It would be hence important to enhance the quality of the data over the contest or to use an even subset of data for such purposes.

This method also contributed to identify, that the labels assigned to images by Computer Vision are very similar to eachother. This led to similar support values in some cases and somewhat biased results. The reason behind this is the fact that chosen Computer Vision tool recognized only the main objects on the images and omitted the details. It would be interesting to enhance the image labels with more specific tags or other meta data (either manually or automatically), that tell unique features of the participants, then perform the same analysis and compare the results. This approach could point out more details concerning which detailed features of participants lead to more attention and engagement from users.

The large number of generated itemsets via this technique also makes the readability of the results challenging. When the generated itemsets grow over certain size (typically 2-3 items), the number of output items is too high to analyze manually. The Co-Clustering technique addresses this challenge well by clustering the matrix of itemset supports and the demographic groups. Through this technique, the itemsets and demographic groups whose structures are similar can be identified and analyzed more carefully together. The results of this approach also can make suggestions for demographic groups, who share similarities in terms of their voting behavior.

One of the challenges with the Co-Clustering method is to choose the number of clusters. For the time being, there was not a single good way of suggesting the best value for this value, because the best match for the number of clusters depends on the size of the dataset (in terms of itemsets and demographic groups) and the underlying structure of the data. Nevertheless, it seems that trial and error works considerably well in this case, but interpreting the results is often not evident. It would be interesting to develop the solution towards a direction, where this challenge is addressed.

The assignment of the cluster labels in the current implementation could be questioned. The fact of assigning one cluster to every row and colum puts a limitation to the outcome. A large number of cells do not belong to any clusters, because the algorithm assumes a block-diagonal structure, which might not be the case in this kind of dataset. In order to improve on that, it would be a good idea to apply more complex and flexible Co-Clustering algorithms for the analysis of the underlying structure. Such algorithm could consider cell-wise comparisons and cluster assignments, which would be more robust in identifying homogeneity of items inside clusters. 

Through the results it can be seen how differences in the behavior of demographic groups are revealed. For instance, through the application of the methods it is speculated that users in the Choicely platform can be seen as specialists and generalists. Furthermore, the data suggests distinctions between the younger end elder generation as well. 

Biases in the data limit the conclusions and the viability of the results and raise threads to their validity. To address this issue, there is a need for a more careful evaluation on the statistical significance of the results. The dataset could be enhanced with more reliable and validated data. One could carry out a structured data collection from a chosen sample of users to ensure the validity of the collected data. Another angle on this topic is that the Choicely platform is rather unique in nature compared to social media sites for instance. Therefore the results obtained in this study may be difficult to reproduce in other online software platforms.

% Privacy and ethical concerns
Privacy and ethical concerns are interesting for many reasons to discuss through the scope of this study. Undoubtedly, the recent technical evolvement has brought many challenges with respect to human rights and data protection, hence there is a need to lay down a common standard in the European Economic Area (EEA). The General Data Protection Regulation (hereinafter GDPR) \cite{gdpr} is being issued on 25th May 2018, which regulates the business and data processing activities by protecting personal data through its novel standards all over the world. According to the regulations, personal data means any kind of information through which the natural person (or data subject, who has generated that data) can be identified \cite{gdpr}. 

The data analysis activities performed in this study are operating on personal and sensitive data, however the results do not reveal any piece of that data. Based on the results of the audience engagement presented in the present research, there is no possible way to identify individual data subjects. The results only show the extraction and statistically significant information about the collected data, but fully hide the individual's personal data. In terms of the GDPR, these activities are called "profiling" \cite{gdpr}. As by Article \cite{gdpr}, the data subject has to be clearly informed about the purposes and goals how his or her generated data is being used. Many other important aspects, such as portability, erasure or automated decision making are addressed over multiple articles (12-23) concerging the rights of the data subjects and the responsibilities of the processor \cite{gdpr}.

In terms of the case company this means, that the purposes of any data processing tasks should be clearly stated and communicated to users. A standatd and convenient channel to forward this information to the platform's users is the company's privacy policy\footnote{\url{https://choicely.com/about/privacy}}, which already contains a list of points in this topic. Nevertheless, as the results and methods of this study are being added to the platform, this list should be extended by clearly expressing the purposes and goals of data collection and analysis.

% self criticism, what would I do differently? 