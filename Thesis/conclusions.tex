% what are the main messages and takeaways? 
User data is widely collected and commonly used for research purposes as well increasing business value of services. The willingness to give this data from users' perspective is present, as many social media sites create a reliable and convenient hub for such repositories. As a consequence, businesses and researchers can easily get access to rich demographic and software usage data through the interfaces of social media platforms. Furthermore, these platforms serve the purpose of convenient authentication and integration to other services, which is highly appreciated by the users. On top of that, the software service providers get easy, quick and reliable access to demographic information about their users. 

Despite its wide availability, the concept of user data is not commonly used in the literature. Nevertheless, numerous studies were conducted in the past - mainly via social media websites - to analyze engagement and trends on how users behave in an online environment. The research topics mainly resolve around what kind of content is attractive to them or how they communicate with eachother in the online software. One of the challenges in this area is, that every software platform is unique, therefore there is not a single way of studying these aspects. Secondly, the data at hand can differ from platform to platform, nevertheless it is commonly too large to analyze using human resources. For this reason, various Data Mining and Machine Learning techniques were utilized by the scientific community to enhance and analyze the data at hand. 

Through this research it was revealed, that while user demographic data is often well established, digital footprints are very domain-dependent in nature. It is concluded that the both sides of user data is necessary to analyze user characteristics and behavior. The combination of computational methods and such data has been already utilized in various areas, such as banking, studying and predicting movie ratings, social media studies and enterprise social networks. This observation proves the relevance and the wide applicability of this research area. Depending on the domain, various tools and methods, such as computer vision, machine learning, association analysis and natural language processing techniques are available, but none of them is a silver bullet for every dataset. 

The Choicely voting/audience engagement platform is an excellent specimen for a software service, where user data is utilized. The aim of this research was to study the user data that is collected by the Choicely platform. The study analyzes the type of content that appears more engaging to users and investigates the voting behaviors of different demographic groups. 

The user data at Choicely is two folded. Users from all over the world log in to the platform, typically using one of their social media credentials to authenticate their identity. As a result, their personal profiles are created in the Choicely platform with their demographic data pre-populated, which establishes one half of the data. The second half of user data is generated by users while using the software. In case of the Choicely platform this means vote transactions on the participants of the contests. 

Each contest participant can be connected to a single image in the platform. The contents on the images varies and generally speaking, there is no standard of labeling the images with meta data for analysis purposes. As a result, analysis on the vote transactions is difficult, because it is not possible to know the type of content users find engaging.

To tackle this problem, Computer Vision (using Google Vision) was integrated to the platform. This technology can reliably assign meta data to the images concerning thier content automatically, whenever they are uploaded to the platform. The vote transactions hence can contain not only on who voted on which participant, but also what the content on that participant's image is. The combination of this data together with the users' demographic data opens up the possibility for a more sophisticated data analysis on the data, in order to study engaging content and users' behavior in the platform. 

To enhance the firm's business package, the basis of a data analysis framework were established. Association Analysis is utilized to extract the support values of the itemsets that appear in the vote transactions. The vote transactions can be studied from multiple angles, however this study is limited to standalone contests and system-wide analysis. By studying transactions of a single contests organizers can evaluate, which labels tend to be more engaging to different demographic in the contest. Furthermore, this methods provides room for analyzing differences between groups as well as itemsets that appear in the contest. The system-wide analysis can reveal the same on a larger, global scale from a richer dataset. 

To computationally identify patterns and the underlying structure of the itemset supports, the Co-Clustering is used. The Co-Clustering is performed on matrix of itemset supports by demographic groups. This approach provides a reliable way to identify which demographic groups and itemsets tend to behave similarly in the data. As a result, engaging content that are specific to groups can be pinpointed, differences in the voting behavior between groups of users can be revealed. 

The results show that there are certain contest categories, which have hosted many more contests, hence have got more engagement than others. The Exploratory Data Analysis revealed, that there is a high number of very small contests. Deriving from this, only a fraction of the whole dataset is actually ready for complex data analysis, because the rest simply does not have sufficient amount of data to analyze. 

The Association Analysis on the itemsets recognized by Computer Vision reveals some interesting insights on the kind of content, which is more engaging to users. One of these findings is, that the contests where participants are actual people appeal to voters more than objects or other abstract concepts. Secondly, objects that appear in the foreground of the images might have more impact on the voting trends than the background or the surroundings, but there is no clear support for this claim. The results also reveal that Computer Vision in this research is useful to identify overall concepts of the images, however it fails to recognize smaller details on the pictures. 

The proposed evaluation on the statistical significance establishes the basis of comparing multiple groups of users once a contest is over. Further development on the automatic execution of this method is needed, which could then easily pinpoint differences in users' behavior. On top of that, the statistical comparison of results over multiple contests is an interesting topic for further studies.

Last but not least, the Co-Clustering approach was successfully used to pinpoint similar demographic groups and itemsets in the vote transactions. This technique provides a robust way to analyze behavior of demographic groups of users and pieces of content. However, this method has not proven useful on the system level, because the data in the platform is often incomplete. More particularly, the demographic data is often missing for many of the users and the assigned labels to the images are often share many similarities in contests. 

% future work
Future work could address on enhancing the quality of the data. The validity of the results in this study could be elevated applying these techniques on a more carefully selected and tested dataset. Future work could also build on top of the results of this study, for example a content recommendation system could be built using the results of Association Analysis and Co-Clustering. It would be also interesting to see the chosen methods performance in similar software platforms, particularly in social media applications. 

Due to the fact that user data involves a large set of personal data, there is an emerging need in studying and establishing standards for the research community. Future work could elaborate also more in-depth on the ethical concerns when conducting research with sensitive user data. Furthermore, it would be a interesting to approach this area not from only computer science, but human behavior point of view. Finally, the application of Data Mining and other computational techniques to facilitate studies focusing on psychology and human behavior would be another stimulating direction for researchers. Conducting more studies in this area would further deepen our understanding on human behavior, preferences as well as gender and cultural differences. 