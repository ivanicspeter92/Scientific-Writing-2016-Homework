\documentclass[english]{tktltiki}
\usepackage[pdftex]{graphicx}
\usepackage{subfigure}
\usepackage{url}
\usepackage{enumerate}
\begin{document}
%\doublespacing
%\singlespacing
\onehalfspacing

\title{Location Awareness - Week 2}
\author{P�ter Ivanics}
\date{\today}

\maketitle

\numberofpagesinformation{\numberofpages\ pages + \numberofappendixpages\ appendices}
\keywords{}

\mytableofcontents

\section{Coordinate Systems and Projections}

\begin{enumerate}[a)]
	\item The reference ellipsoid and 
	
	To determine ones position, it is essential to know the actual altitude of the object. The sea level is a good reference for the altitude, however due to the gravitational effects, the actual sea level might differ around the Earth. 
	
	The geoid is a hypothetical figure that approximates the mean sea level around the Earth as well as its shape. This yields in a mathematically indescribable shape and therefore cannot be used for determining ones position.
	
	The reference ellipsoid is the model of the geoid, which is utilized for mathematical calculations to determine ones position. This means that the shape is not totally accurate, it is only a fit to the geoid. If the mean sea level determined with this technique naturally includes some errors but afterall gives a good enough fit estimation on the position. This means, we can determine the position of an object with a tolerable error rate using an approximate surface of the Earth.
	
	\item 
	\item 
\end{enumerate}

\section{KML Visualization}

\begin{enumerate}[a)]
	\item 
	\item 
	\item 
\end{enumerate}

\section{Fingerprint-based positioning}

\begin{enumerate}[a)]
	\item 
	\item 
\end{enumerate}

\section{Positioning implementation}

\nocite{*}
\bibliographystyle{tktl}
\bibliography{lahteet}

\lastpage

\appendices

\pagestyle{empty}

%\internalappendix{1}{Model ABC}
%
%The appendices here are just models of the table of contents and the presentation. Each appendix 
%usually starts on its own page, with the name and number of the appendix at the top. Each appendix is paginated separately.
%
%In addition to complementing the main document, each appendix is also its own, independent entity. 
%This means that an appendix cannot be just an image or a piece of programming, but the appendix must explain its contents and meaning.

\end{document}


