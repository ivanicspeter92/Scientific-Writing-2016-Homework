\documentclass[english]{tktltiki}
\usepackage[pdftex]{graphicx}
\usepackage{subfigure}
\usepackage{url}
\usepackage{enumerate}
\usepackage{amsmath}
\usepackage{hyperref}
\begin{document}
\onehalfspacing

\title{Location Awareness - Week 5}
\author{P�ter Ivanics}
\date{\today}

\maketitle

\numberofpagesinformation{\numberofpages\ pages + \numberofappendixpages\ appendices}
\keywords{}

\mytableofcontents

\section{Concepts}
	\begin{enumerate}[a)]
		\item R-tree is a special tree data structure which is used for indexing data in spatial databases. R-trees are designed specifically for and widely used in spatial indexing. 
		
		The high level idea of this data structure is to group objects together with their minimum bounding rectangle level-by-level in the tree. The leafs of the tree are single objects, while elements on higher levels embed more and more items and groups. This creates a hierarchical structure, which greatly facilitates location-based algorithms and entity representation. Searching algorithms, such as intersection, containment and K nearest neighbors can be executed efficiently and stored using relatively small amount of resources. Other variations are R+ tree, R* tree and Hilbert R-tree. 
		
		\item After fitting the third order Hilbert curve to the given figure, the Hilbert values of the boxes are 8, 9 and 14 as shown on Figure \ref{hilbert_curve}. The Hilbert values in an R-tree serve the purpose to create and maintain the hierarchical structure of the indexes. 
		
	\begin{figure}[h] 
		\begin{center}
			\includegraphics[]{images/hilbert_value.png}
			\caption{The third order Hilbert curve fit to the given bounding boxes.}
			\label{hilbert_curve}
		\end{center}
	\end{figure}		
		
		\item The attached $trace.R$ script estimates the number of segments. The calculation is done by estimating the distance between the consecutive measurements. In case the consecutive headings are the same, the segments are considered to be individual. This way there are a total number of 18 segments identified in the data.
		\item 
	\end{enumerate}
\section{Trajectory analysis}
	\begin{enumerate}[a)]
		\item The solution for both tasks can be found in the attached $trajectory\_analysis.R$ file. 
		
		\item The $douglasPeucker()$ function simulates the Douglas-Peucker algorithm on the provided trajectory. With the error bound of 0.005 units, the algorithm senses two points as shown on Figure \ref{douglas_peucker}. This yields in a reduction rate of $749 / 2 = 374.5$.
		
		\begin{figure}[h] 
		\begin{center}
			\includegraphics[width=0.5\textwidth]{images/douglas-peucker-1.png}
			\includegraphics[width=0.5\textwidth]{images/douglas-peucker-2.png}
			\includegraphics[width=0.5\textwidth]{images/douglas-peucker-3.png}
			\caption{The three steps of the Douglas-Peucker algorithm on the given input. As shown, sensing two points was sufficient enough to match the 500 meter error rate.}
			\label{douglas_peucker}
		\end{center}
	\end{figure}		
	\end{enumerate}

\section{Transportation modes}
	\begin{enumerate}[a)]
		\item The values are calculated by the $transportation\_modes.R$ script. The values for the given dataset using the given thresholds are as follows. As the assignment suggested, 1 km was used as a unit distance for the calculations. 
		
		\begin{eqnarray*}
			d = 327.1344 \\
			\\
		 	|P_c| = 8 \rightarrow HCR= |P_c| / d = 24.45478 \\
		 	|P_s| = 2 \rightarrow SR = |P_s| / d = 6.113695 \\
		 	|P_v| = 39 \rightarrow VCR = |P_v| / d = 119.2171
		\end{eqnarray*}
		
		\item To tell the motion of the measurements, the mean variance and intensity was calculated. The findings suggest that the data in $mode1.csv$ was recorded while the user was standing still and the data in $mode2.csv$ was recorded on the tram while moving. Table \ref{Intensity-and-variance} summarizes the findings.
		
		\begin{table}[]
			\centering
			\caption{The mean intensity and variance of the provided measurements.} 
			\label{Intensity-and-variance}
			\begin{tabular}{llll}
			\\
			\hline
          & Mean intensity & Mean variance & Conclusion     \\ \hline
			mode1.csv & 1884.4         & 11.90095      & Standing still \\
			mode2.csv & 1830.4         & 14.21143      & Moving         \\ \hline
			\end{tabular}
		\end{table}
		
	\end{enumerate}

\section{Temporal modeling}
	\begin{enumerate}[a)]
		\item The solution is implemented in the attached $temporal\_modelling.R$ script file. More specifically, the $discretizeData()$ function performs the calculation of seconds since midnight for each timestamp and performs the bin-categorization accordingly based on 15-minute slots. 
		\item 
		\item
	\end{enumerate}
	
\nocite{*}
\bibliographystyle{tktl}
\bibliography{lahteet}

\lastpage

\pagestyle{empty}

\end{document}