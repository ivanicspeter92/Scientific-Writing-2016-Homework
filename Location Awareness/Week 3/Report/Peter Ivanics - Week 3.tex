\documentclass[english]{tktltiki}
\usepackage[pdftex]{graphicx}
\usepackage{subfigure}
\usepackage{url}
\usepackage{enumerate}
\begin{document}
\onehalfspacing

\title{Location Awareness - Week 3}
\author{P�ter Ivanics}
\date{\today}

\maketitle

\numberofpagesinformation{\numberofpages\ pages + \numberofappendixpages\ appendices}
\keywords{}

\mytableofcontents

\section{Location systems}
	\begin{enumerate}[a)]
		\item Beacon Scan Request is a periodic broadcast request generated by mobile devices based on Bluetooth Low Energy or WiFi signals. It allows to identify being in the proximity of beacons, which may store valuable information for the user. On top of that, since beacons can approximate the distance of the devices based on the broadcast signal strength, positioning algorithms may be applied to determine the relative position of the users. 
	\item Range and Pseudorange (biased range) are terms used in Global Positioning System (GPS), which refer to the distance between a satellite and a device calculated based on the satellite's broadcast message.
	
	The range is estimated based on the orbital position and the system time of the satellite, which is embedded in its broadcast message. The challenge is that the calculations may not precise, because the receiver's and the satellite clock times are typically not synchronized.
	
	 Pseudorange estimates the position based on the receiver's clock offset as an unknown variable. In order to determine one's position correctly, the pseudorange for 4 different satellites has to be calculated at least. As a result, the pseudorange calculation is more precise in comparison to regular range calculations. 
	
	\item Table \ref{positioning-errors} summarizes the accuracy for different percentiles while Figure \ref{positioning-errors-plot} visualizes the errors of the given systems. From this data we can see that System 1 is more accurate on average, as the its mean is only $4.65$ while System 2 has a mean error of $5.39$. Furthermore, System 1 is more consistent as it produces less errors for all percentiles. 
 	
		\begin{table}[]
		\centering
		\caption{The accuracy of the given systems in the $positioningErrors.csv$. }
		\label{positioning-errors}
		\begin{tabular}{lll}
			& System 1 & System 2 \\ \hline
			Mean error & 4.654359 & 5.390046 \\
			100-percentile & 18.935   & 28.837   \\
			99-percentile  & 13.568   & 15.442   \\
			95-percentile  & 11.067   & 12.817   \\
			50-percentile  & 4.0117   & 4.6446  
		\end{tabular}
		\end{table}
				
			\begin{figure}[h] 
			\begin{center}
				\includegraphics[width=0.75\textwidth]{images/positioningErrorsCDF.png}
				\caption{The CDF plot of the positioning errors.}
				\label{positioning-errors-plot}
			\end{center}
		\end{figure}		
		
	\end{enumerate}
\section{GPS}

\section{Particle Filter}

\section{Kalman Filter}

\nocite{*}
\bibliographystyle{tktl}
\bibliography{lahteet}

\lastpage

\appendices

\pagestyle{empty}

\end{document}