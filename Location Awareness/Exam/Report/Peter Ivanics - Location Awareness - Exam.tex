\documentclass[english]{tktltiki}
\usepackage[pdftex]{graphicx}
\usepackage{subfigure}
\usepackage{enumerate}
\usepackage[table,xcdraw]{xcolor}
\usepackage{url}
\usepackage{mathtools}  
\mathtoolsset{showonlyrefs}  
\begin{document}
\onehalfspacing

\title{Location Awareness - Home Exam}
\author{P�ter Ivanics}
\date{\today}

\maketitle

\numberofpagesinformation{\numberofpages\ pages + \numberofappendixpages\ appendices}
\keywords{}

\mytableofcontents

\section{Concepts}
	\begin{enumerate}[a)]
		\item Let us donate the length of the semi-major axis with $a = 6378137$ and the inverse flattening $1/f \approx 298.26$. Calculating the actual flattening we get $f = \frac{1}{298.26} \approx 3.35 * 10^{-3}$.  Then we use the equation of flattening to calculate $b$ as follows: 
		\begin{eqnarray*}
			f = \frac{a - b}{a} \\ \\
			3.35 * 10^{-3} = \frac{6378137 - b}{6378137} \\ \\
			3.35 * 10^{-3} * 6378137 = 6378137 - b \\ \\ 
			- 3.35 * 10^{-3} * 6378137 + 6378137 = b \\ \\ 
			b \approx 6356753
		\end{eqnarray*}
		
	\item To obtain the answer we shall calculate the Helsinki-Santiago and the Helsinki-Auckland distance. We use the Haversine formula \footnote{\url{https://en.wikipedia.org/wiki/Haversine_formula}} 
	\begin{gather}
		Dist([\phi_1, \lambda_1], [\phi_2, \lambda_2]) = 2 r \arcsin \sqrt{\sin^2 (\frac{\phi_2 - \phi_1}{2}) + \cos (\phi_1) * \cos (\phi_2) * \sin^2 (\frac{\lambda_2 - \lambda_1}{2})}
	\end{gather}
	to calculate the distances as follows (in kilometers) using $r = 6371$ kilometers and coordinates retrieved from Google:
	\begin{eqnarray*}
		Dist(Hel, San) = Dist([60.1699� N, 24.9384� E], [33.4489� S, 70.6693� W]) = 13480 \\
		Dist(Hel, Auc) = Dist([60.1699� N, 24.9384� E], [36.8485� S, 174.7633� E]) = 16660
	\end{eqnarray*}
	
	If the flying range of the plane is $r = 15190$ kilometers, it can reach Santiago without refueling and still have fuel for approximately $15190 - 13480 = 1710$ kilometers. However, it cannot reach Auckland without refueling as it would run out of fuel around $|15190 - 16660| = 1470$ kilometers before reaching the destination. 
	
	\item The attached $k-anonimity.R$ file contains an implementation for retrieving the answer for item c) and d). The calculated Euclidean distances between the given points are displayed in the table below. The $4^{th}$ smallest distance in the table is $d = 0.08331873$ which means to ensure k-anonymity of $k=4$, this value should be used as threshold. 
	
	\begin{table}[ht]
		\centering
		\begin{tabular}{rrrrrrrrrrr}
 		 \hline
 		& 1 & 2 & 3 & 4 & 5 & 6 & 7 & 8 & 9 & 10 \\ 
  		\hline
		1 & 0.00 & 0.26 & 0.27 & 0.11 & 0.17 & 0.18 & 0.15 & 0.21 & 0.16 & 0.24 \\ 
		 2 & & 0.00 & 0.28 & 0.26 & 0.18 & 0.34 & 0.27 & 0.13 & 0.16 & 0.04 \\ 
		 3 & & & 0.00 & 0.23 & 0.14 & 0.21 & 0.14 & 0.16 & 0.22 & 0.27 \\ 
		 4 & & & & 0.00 & 0.12 & 0.23 & 0.17 & 0.20 & 0.10 & 0.22 \\ 
		 5 & & & & & 0.00 & 0.21 & 0.13 & 0.08 & 0.08 & 0.16 \\ 
		 6 & & & & & & 0.00 & 0.08 & 0.24 & 0.27 & 0.33 \\ 
 		 7 & & & & & & & 0.00 & 0.17 & 0.19 & 0.26 \\ 
 		 8 & & & & & & & & 0.00 & 0.13 & 0.12 \\ 
 		 9 & & & & & & & & & 0.00 & 0.13 \\ 
  		10 & & & & & & & & & & 0.00 \\ 
   		\hline
		\end{tabular}
	\end{table}
	
	\item Using the given parameter the program outputs the following: \textit{"Parameters: eps = 0.15, minPts = 4. The clustering contains 1 cluster(s) and 0 noise points"}. This means that all points are grouped under the same cluster and all of them have dense neighborhoods. 
	\end{enumerate}
\section{Fingerprinting}
\section{Position Estimation}
\section{Spatial Analysis}
\section{Trajectory Preprocessing and Analysis}
\section{Map Matching}

\nocite{*}
\bibliographystyle{tktl}
\bibliography{lahteet}

\lastpage

\pagestyle{empty}

\end{document}