\documentclass[english]{tktltiki}
\usepackage[pdftex]{graphicx}
\usepackage{subfigure}
\usepackage{url}
\begin{document}
%\doublespacing
%\singlespacing
\onehalfspacing

\title{Assignment 1: User research}
\author{P�ter Ivanics (pair: Julian Schulte)}
\date{\today}

\maketitle

\numberofpagesinformation{\numberofpages\ pages + \numberofappendixpages\ appendices}
\keywords{}

\mytableofcontents

\section{Introduction}
	This short report is a summary of a user research concerning the usage of tablet devices for reading and learning purposes. The report is rolled out as one of the assignment of the Human-Computer Interaction course and serves the purpose to gain hands-on experience with the course material. 
	
	The motivation behind choosing this topic mainly my own experience and interest, but also the common applicability of the problem. Personally, I never liked to read literature (novels, stories, poems etc.) from screen and preferred to use books instead. The main reason behind this is the fact that it felt more natural and less tiring, nevertheless, individuals may have subjective opinions in this manner. For this reason, it would be interesting if others share the same feelings, what are the tendencies and reasons why individuals prefer one or another way of reading. 
	
	On top of that, there are plenty of researches ongoing in the present time about the potential replacement of textbooks with electronic devices in education. For example, several elementary and grammar schools started to hand over tablets to pupils instead of printed books in classroom environments. This has several positive results, such as children do not have to carry heavy books with themselves day-by-day, teachers can present audiovisual material as part of the lessons to facilitate different styles of learning. Personally, I tried to read study material from, add markup and notes to my textbooks and lecture notes, which I did not find convenient.
	
	This short research aims to discover the opinion of users concerning reading literature and study material from an iPad tablet. The research plan below describes the main guidelines and aspects that are utilized to conduct the research. Further, the findings are of a single person's interview session is described. Finally, some conclusions are made about my learning progress and gained experiences through this assignment both as interviewee and interviewer. 

\section{Research plan}
	As the first step of the research plan, the research question and its subquestions are formulated. The main research questions is \textit{How does reading experience differ on a tabled compared to paper-based books?}  Due to the fact that this research question can have many aspects and may be too wide for this assignment, the scope is reduced to answer the following subquestions (and their methodological dimensions): 
	
	\begin{itemize}
		\item \textit{How does the reading experience of literature and coursebooks differ on tablets compared to paper-based books based on users' impressions?} (open-ended, interventionist, qualitative, subjective)
		\item \textit{Does it feel natural for readers to read from a tablet?} (close-ended, observational, quantitative, subjective	)
		\item \textit{Do readers prefer books over tablets as sources of reading resources?} (close-ended, interventionist, quantitative, subjective)
		\item \textit{Do users find it difficult to use a tablet for studying (e.g. making markups, notes, bookmarks in a coursebook) more difficult compared to a regular coursebook?} (close-ended, interventionist, quantitative, subjective)
	\end{itemize}
	
	An in-depth research may include more aspects, such as efficiency/speed of reading, long-term reading and how it affects the reader eye's tiredness, focus and attention. As the assignment was targeting to practice the interview and observation methods, these are given for the study. Nevertheless, it is important to point out that some of the questions above can be answered through other qualitative and quantitative measures, e.g. through surveys. In other words, this means that different methods may be better choice to answer some of the questions above in a similar study. 
	
	To carry out the research, the following semi-structured interview plan is designed with the participants. The interview is structured as follows. To begin with, a 5-minute opening/introductory step is performed, where general information is told to the participant. The aim of this is to make him/her comfortable and sure that he/she is motivated to participate in the study. These questions will also assist to establish trust throughout the interview.
	
	Second, some background information is retrieved about the participant. It is important to ask simple and clear questions as below and let him/her express feelings and experiences. The collected information here is targeted to learn more about the participants, to discover their reading habits and technical skills. This will make the participant relaxed as well as researchers will understand more of the background during the analysis. 
	
	In the next 20-25 minutes, the participant is asked to perform some simple tasks. The shown performance serves as the basis of the evaluation. Participants should not be hasted and put under pressure - they should be given time to perform the tasks and guidance, in case it is needed. The tasks are explained in more details in the paragraphs to follow.
	
	Finally, the last five minutes is dedicated to close the interview. The participant is asked of his/her feelings, impressions and feedback. The interviewer should conclude the session and say thanks for the participation. 
	
	% where the interview is conducted
	In order to achieve the best possible outcome, the interview should be conducted in a quiet environment. As reading in general is a focused activity and the interviewee will be asked several questions during the session, such noises and disturbances should be eliminated by carefully choosing the location. On the other hand, a laboratory room may not be a proper choice either, because it would make the participants feel uncomfortable. The best match probably would be a "living room"-like environment where the participant feels comfortable. 
	 
	% preparations/preconditions of the interview
	Prerequisites and preparation equipment for this interview includes the following:
	\begin{itemize}
		\item the tablet (e.g. iPad) which is going to be evaluated (make sure the battery is charged),
		\item a printed copy of literature (e.g. a novel),
		\item the PDF/ePub version of the same book downloaded to the device,
		\item any kind of university lecture slides (PDF/PPT) and a coursebook downloaded to the device,
		\item audio recorder (or camera) for recording the session,
		\item pen and paper for taking notes.
	\end{itemize}
	
	The next page lists the step-by-step task and schedule for the interview. The listed points are more guidelines than strict points to follow. Respect the time and leave points out if necessary. Let the participants finish their thoughts and do not interrupt their speech unless the discussion goes into deep details!
	
	\pagebreak
	\singlespacing
	\footnotesize
	
	\begin{itemize}
		\item \textbf{5 minutes: Opening}
		\begin{itemize}
			\item open the discussion,
			\item introduce the participant the content of the interview (what will happen),
			\item remind the participant that the interview will take approximately 40-45 minutes,
			\item ask the participant for permission about recording the session and ensure him/her that it will remain confidential material,
			\item ask the participant to introduce him/herself in couple of sentences,
			\item ask the participant how he/she feels,
			\item ask the participant if he/she has any questions before beginning the interview.
		\end{itemize}
		
		\item \textbf{5 minutes: Background information}
		\begin{itemize}
			\item ask how often does he/she read books,
			\item ask what kind of books are in his/her interest,
			\item ask if he/she prefers paper-based books rather than reading from electronic devices, 
			\item ask how familiar he/she is with mobile devices and tablets, 
			\item if the participant is not familiar with the device, give a short introduction about the usage.
		\end{itemize}
		
		\item \textbf{20-25 minutes: Tasks}
		\begin{enumerate}
			\item Literature reading (10 minutes)
			\begin{enumerate}
				\item open the book reader software (e.g. iBooks) and open a book of choice (a novel, a roman or similar literature) through the software,
			\item read 1-2 pages and express feelings (e.g. how does it feel to read from the screen, is it more difficult than from paper, does it feel natural to hold the device in hand in landscape and portrait orientation, etc.)
			\item give the same paper-based copy of a book to the participant; ask him/her to read 1-2 pages and express feelings in comparison to the previous experience (does this feel more natural, is the text easier or harder to read, etc.), 
			\end{enumerate}
			
			\item Study-related material (10 minutes)
			\begin{enumerate}
				\item open the book reader software (e.g. iBooks) and open a study related material (a coursebook or PPT lecture slides) through the software,
				\item browse through the pages and express feelings (how does it feel to read from the screen, is it more difficult than from paper, does it feel natural to hold the device in hand in landscape and portrait orientation, etc.)
				\item add couple of bookmarks to certain pages and navigate between these, 
				\item add markups (own notes) to the contents of the material.
			\end{enumerate}
		\end{enumerate}
		
		\item \textbf{5 minutes: Closing, conclusions}
		\begin{itemize}
			\item close the discussion,
			\item say thanks for the participation, 
			\item ask if there are any questions
		\end{itemize}
	\end{itemize}
	
	\pagebreak
	\normalsize
	\onehalfspacing

\section{Findings}

\section{Conclusions}
	
\pagebreak
\nocite{*}
\bibliographystyle{tktl}
\bibliography{bibliography}

\lastpage

\end{document}


