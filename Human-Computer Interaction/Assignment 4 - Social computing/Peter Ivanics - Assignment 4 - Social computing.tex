\documentclass[english]{tktltiki}
\usepackage[pdftex]{graphicx}
\usepackage{subfigure}
\usepackage{url}
\usepackage{tabu}
\begin{document}
\onehalfspacing

\title{Assignment 4: Social computing}
\author{P�ter Ivanics}
\date{\today}

\maketitle

\numberofpagesinformation{\numberofpages\ pages + \numberofappendixpages\ appendices}
\keywords{}

\mytableofcontents

\section{Introduction}
	This report compares the nature of users' communication in the official Formula 1 Facebook \cite{F1F} and Instagram \cite{F1I} pages. The report is rolled out as one of the assignment of the Human-Computer Interaction course and serves the purpose to gain hands-on experience with the course material. 
	
	The reason behind choosing the channels of Formula 1 are the popularity of the sport, the recent ending of the 2016 season, as well as my personal interest. The main basis of comparison are the dimensions of computer-mediated communication. The report is carried out from the point of some of the listed dimensions in the original assignment. The analyzed dimensions are introduced in the chapter to follow. 
	
\section{Discussion}
	Before beginning the analysis, let us introduce the official Formula 1 Facebook \cite{F1F} and Instagram \cite{F1I} pages. The two channels are operated by officials and their content is updated fairly regularly during the Formula 1 seasons. In particular, on race weekends this means multiple posts on both channels which often share the same content. Each page has more than 1.9 million followers. These channels are good candidate for this assignment as they represent a sport, are certified, publicly available and have many followers all around the world. 
	
	The 2016 Formula 1 world champion title was decided on the last race of the season last week. This fact makes it interesting to take a look at the activity on the channels of the sport now as a retrospective analysis. Typically the end of the season brings excitement not only in the champion's title, but also the retiring and freshly signed drivers, who will enter their rookie year in the upcoming season. For example, this year two "veteran" drivers are retiring from the sport which certainly creates discussions in the social media on top of the title battle. Last, but not least there may be other discussion topics, such as next year's new rules, tires, regulations, budget, teams and so on.

	To begin the analysis with, the dimensions to focus on are chosen. The dimensions used for the analysis and their descriptions are, as follows. The classification of the channels based on the above dimensions are displayed in Table \ref{classification-table}.
	
	\begin{enumerate}
		\item Identity: Communication is anonymous / based on avatar or nickname / with participant?s true identity
		\item Messages vs. streams: Communication is message-based vs. a continuous stream
		\item Symmetricity: Communication is the symmetric or asymmetric (i.e., participants do not have equal communication possibilities, e.g. question-and-answer role)
		\item Format: Communication is pictorial/textual/based on some other format
		\item Length: System allows short or long messages
	\end{enumerate}
	
	\begin{table}
	\centering
	\caption{The classification of the channels based on the selected dimensions. } 
	\label{classification-table}
	\begin{tabu} to \textwidth {XXX}
			\\
           & \textbf{Facebook page} & \textbf{Instagram page} \\ \hline
			\textbf{Identity} & Known - users have their real names and profile pictures displayed when commenting. & Not necessarily known - users typically have alias/nicknames rather than real, full names.\\ \hline
			\textbf{Messages vs. streams} & Message-based. Audiovisual material may be attached to the original post and to comments by followers. & Message-based. Audiovisual material may be attached to the original post. \\ \hline
			\textbf{Symmetricity} & Asymmetric & Asymmetric \\ \hline
			\textbf{Format} & Mainly textual but may include audiovisual material & Textual \\ \hline
			\textbf{Length} & No limitation on message length & No limitation on message length \\ \hline
			\textbf{Moderation} & Free communication with potential post-moderation & Free communication with potential post-moderation \\ \hline
		\end{tabu}
	\end{table}

\section{Conclusions}
	
\pagebreak
\nocite{*}
\bibliographystyle{tktl}
\bibliography{bibliography}

\lastpage

\end{document}