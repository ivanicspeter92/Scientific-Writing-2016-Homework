\documentclass[english]{tktltiki}
\usepackage[pdftex]{graphicx}
\usepackage{subfigure}
\usepackage{url}
\begin{document}
%\doublespacing
%\singlespacing
\onehalfspacing

\title{Assignment 1: Improvement of the interaction of a website}
\author{P�ter Ivanics}
\date{\today}

\maketitle

\numberofpagesinformation{\numberofpages\ pages + \numberofappendixpages\ appendices}
\keywords{}

\mytableofcontents

\section{Heuristic evaluation}

% [description of the usability characteristics used, and UI improvements planned based on them]

\section{Implementation}

% [description that helps the lecturer(s) to understand your solution; does not need to be long]

\section{Unimplemented features}

% [describe here the solutions that your web site does not have but which would be useful and important]

\nocite{*}
\bibliographystyle{tktl}
\bibliography{lahteet}

\lastpage

\appendices

\pagestyle{empty}

%\internalappendix{1}{Model ABC}
%
%The appendices here are just models of the table of contents and the presentation. Each appendix 
%usually starts on its own page, with the name and number of the appendix at the top. Each appendix is paginated separately.
%
%In addition to complementing the main document, each appendix is also its own, independent entity. 
%This means that an appendix cannot be just an image or a piece of programming, but the appendix must explain its contents and meaning.

\end{document}


