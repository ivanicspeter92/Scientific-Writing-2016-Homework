\documentclass[english]{tktltiki}
\usepackage[pdftex]{graphicx}
\usepackage{subfigure}
\usepackage{url}
\begin{document}
%\doublespacing
%\singlespacing
\onehalfspacing

\title{Assignment 1: Improvement of the interaction of a website}
\author{P�ter Ivanics}
\date{\today}

\maketitle

\numberofpagesinformation{\numberofpages\ pages + \numberofappendixpages\ appendices}
\keywords{}

\mytableofcontents

\section{Introduction}
	This short report contains a heuristic evaluation and improvement of a simple one-page website. The report is rolled out as one of the assignment of the Human-Computer Interaction course and serves the purpose to gain hands-on experience with the course material. The given webpage of discussion is displayed on Figure \ref{original_webpage}. 
	
	\begin{figure}[h] 
		\begin{center}
			\includegraphics[height=0.6\textwidth]{images/original_webpage.png}
			\caption{The webpage of discussion in its original state, that is going to be analyzed and improved in this report.}
			\label{original_webpage}
		\end{center}
	\end{figure}

	As shown, the webpage includes a very simple web form with the purpose of user registration for a service .The page has a title of "Online shop -- Registration", a couple of labels and textboxes and a button in the bottom which is supposed to validate and submit the data if all fields are filled correctly by the user. The next chapter of this report describes the heuristic evaluation of the website, which means listing the proposed suggestions, improvement possibilities, design ideas and features for this simple form. The evaluation is mainly based on the course material and the contents of the book Don't Make me Think by Steve Krug \cite{SK14}. Further, a short description about the implementation is given which is based on the former evaluation. Finally, the last chapter concludes the report and describes further improvement possibilities for the website. 

\section{Heuristic evaluation}

% [description of the usability characteristics used, and UI improvements planned based on them]
	To set the context, the term of heuristic evaluation is defined shortly. As discussed on the lecture, heuristic evaluation is one of the possible methods for performing usability testing of a website or system. The heuristic method means performing the evaluation without end-user involvement. This is done on the given single page with the utilization of the discussed usability characteristics and explanation on User Centered Design (UCD) during the lecture.
	
	 To begin the analysis with, the original form displayed on Figure \ref{original_webpage} is as minimalistic as possible. In general this should not be considered as a problem, but rather as a possibility, because this observation gives a lot of room for improvements. Taking the first look at the original webpage, one can immediately see what it is about - the user is expected to provide certain information about him/herself to sign up for a service. Nevertheless, one notices soon that this page does not correspond to today's standards, the page is not optimized for mobile devices and usability can be greatly enhanced by applying some simple modifications. Due to the simplicity of the page, too many use-case scenarios cannot be investigated and therefore only the inspection of this screen is explained. The suggestions to follow are explained in importance order (from the most important to the least important). 
	 
	 The most important principle to point out during the improvement of a web page is the "Don't make me think" principle \cite{SK14}. To summarize in one sentence, the meaning is to keep the webpage as self-evident and obvious as possible. As suggestions are made for improving the look \& feel of the site, the aspects below should be kept in mind all the time. When a user is presented this page, they should immediately understand that 
	 
	\begin{enumerate}
		\item they are on a registration form, 
		\item what are they signing up for,
		\item what information are they required to give,
		\item how do they submit that information,
		\item what happens after the information is submitted/where the data is going to be go,
		\item what subset of the information is mandatory (i.e. which are the mandatory fields and which ones can be skipped).
	\end{enumerate}
	
	The next suggestion would be to ensure users will know how to use the form. Naturally, this strongly correlates with the previous "Don't make me think" principle \cite{SK14}. To be more specific, the form should have standard controls, use common conventions, metaphors and keep up the clarity and consistency. This would mean maintaining the same elements size of the elements on the screen, usage of the same colors, font size and type consistently. The content should be organized in a logical manner, for instance the items in the form should be in a reasonable order and the submit button should stay on the bottom of the form.
	
	To increase user attention and have a good first impression, the page could have some lively colors while keeping up an aesthetic and minimalistic design. This also means that the text should remain legible, the more important content should be emphasized using larger font size and there should be contrast between the text and the background's color. 
	
	Next up, as the user starts filling the form, he/she should get immediate feedback about the correctness of the information. For instance, when one of the textboxes looses focus, the borders of the control could turn green, if the given information is correct or red, if the entered value is invalid. This will give immediate feedback to the user about the input and help to avoid errors. An additional aim is to avoid potential errors as much as possible, for instance using dropdown selection or checkboxes rather than open text fields whenever possible \cite{SK14}. In case open text fields expect to have an input of a pre-defined format, the interface should intuitively help the user as suggested by Kim \cite{KIM15}. In case one of the values is incorrect, helpful error messages should indicate the cause of the problem and help the user to make proper corrections.
	
	Last but not least, to keep up to today's standards, the page should support mobile devices. Supporting multiple screen sizes and browsers would add value to the site and greatly enhance user experience along various platforms and therefore should be kept in mind during the implementation. Of course, in a real scenario this may not be necessary depending on the targeted audience, however it is good to be on the save side in this matter. Fortunately, there are many popular and ready-made solutions that provide wide support for all screen sizes. 
	
	To summarize the suggestions in preference order, they are listed in Table  \ref{usability_characteristics_table} below. During the implementation these aspects are going to be kept in mind as indicated with their priority order. 
	
	\begin{table}[]
		\centering
		\caption{The most important usability aspects for the improvement of the webpage.}
		\label{usability_characteristics_table}
		\begin{tabular}{ll}
			Priority & Usability aspect                           \\
			1        & "Don't make me think" / being self-evident \\
			2        & Consistency and standards                  \\
			3        & Visibility and logical layout              \\
			4        & Aesthetic and minimalistic design          \\
			5        & Legibility and harmonized colors           \\
			6        & Feedback                                   \\
			7        & Error prevention and recovery              \\
			8        & Support for mobile                        
		\end{tabular}
	\end{table}
	
\section{Implementation}
% [description that helps the lecturer(s) to understand your solution; does not need to be long]

\section{Unimplemented features}

% [describe here the solutions that your web site does not have but which would be useful and important]

\nocite{*}
\bibliographystyle{tktl}
\bibliography{bibliography}

\lastpage

\appendices

\pagestyle{empty}

%\internalappendix{1}{Model ABC}
%
%The appendices here are just models of the table of contents and the presentation. Each appendix 
%usually starts on its own page, with the name and number of the appendix at the top. Each appendix is paginated separately.
%
%In addition to complementing the main document, each appendix is also its own, independent entity. 
%This means that an appendix cannot be just an image or a piece of programming, but the appendix must explain its contents and meaning.

\end{document}


