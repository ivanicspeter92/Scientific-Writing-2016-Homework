\documentclass[conference]{IEEEtran}

\usepackage[utf8]{inputenc}
\usepackage{cite}
\ifCLASSINFOpdf
  \usepackage[pdftex]{graphicx}
\else
  \usepackage[dvips]{graphicx}
\fi
\usepackage{amsmath}
\usepackage{array}
\usepackage{url}
\usepackage{paralist}
\usepackage{enumitem}
% correct bad hyphenation here
% \hyphenation{op-tical net-works semi-conduc-tor}

\begin{document}
% paper title
% Titles are generally capitalized except for words such as a, an, and, as,
% at, but, by, for, in, nor, of, on, or, the, to and up, which are usually
% not capitalized unless they are the first or last word of the title.
% Linebreaks \\ can be used within to get better formatting as desired.
% Do not put math or special symbols in the title.
\title{Audience engagement and user data analysis with Data Mining}

\author{\IEEEauthorblockN{Péter Ivanics}
\IEEEauthorblockA{Department of Computer Science \\
University of Helsinki \\
Email: peter.ivanics@helsinki.fi \\
\url{http://pivanics.users.cs.helsinki.fi/portfolio/}}}

\maketitle

\begin{abstract}
%<Context>
Many software businesses collect enormous amount of data generated by end-users. End-user data incorporates essential information about how users interact with the system of discussion as well as information about the users themselves. For instance, such digital footprints can explain the preferences of users, what kind of content they like, how they interact with the system and eachother or how frequent their activity in the platform is.

%<Objectives>
The goal of this research is to study understand the potential use of user data in business and research environments in the present time with the application of Data Mining methods. Since there is no common understanding on its concept, user data in this research consists of data willingly uploaded by users and generated through their interactions while using the software. The research aims to study how content on the Internet is observed by the users, who access the content. Secondly, it is studied what kind of feedback is received from the engaged audience in terms of like activities and comments to the provided content. Thirdly, the tendencies among the demographic characteristics between users are studied in relation to the content which gets them engaged online.

%<Methods>
The foundations for the theoretical basis are retrieved by reviewing relevant literature. The sources include scientific journals, articles as well as books related to the topic of this study.
 %The two former sources are retrieved through online digital libraries, such as the ACM digital library, IEEE Xplore and Google Scholar. 
The snowballing technique was used to retrieve further papers in the field. Furthermore, various keywords were used to obtain related literature in the research area.

%<Results>
The research papers in the field show various ways how user data can be used. Social networking sites and movie databases commonly utilize various statistical and data analysis tools on user data. Results show for instance, that Instagram users typically can be divided into two groups -  generalists and specialists - based on the variety of the content they like in the software. Studies about social media sites in enterprise environments have given insights to the organization on how their employees cooperate and interact by utilizing a set of data mining tools. As a result, researchers can learn the insights of the user data and look at human behaviour from a different, novel angle.  

\end{abstract}

\begin{IEEEkeywords}
user data, data mining
\end{IEEEkeywords}

\section{Introduction}

Many software businesses collect enormous amount of data generated by end-users \cite{chinesemobilebankingusers, bigdatamanagementrevolution, inmon2007tapping}. End-user data incorporates essential information about how users interact with the system of discussion as well as tells a lot about the users themselves. For instance, such data can explain the preferences of users, what kind of content they like, how they interact with the system and eachother or how frequent their activity in the platform is.

 Depending on the portfolio of the business, analysis of end-user data can reveal various interesting findings. For instance, in banking industry Big Data tools are often used to analyze demographic characteristics to maintain and establish new client relationships \cite{chinesemobilebankingusers, bigdatamanagementrevolution}. Extracting the information from the data is challenging: most businesses are to collect more data than what is humanly possible to process and to analyze \cite{inmon2007tapping, wegener2010integrating}. As a result, significant part of the knowledge might remain hidden and hence business-critical information remains unseen \cite{inmon2007tapping, wegener2010integrating, introtodatamining, chinesemobilebankingusers}. As a result, there is a growing need for all businesses to introduce data analysis processes in their daily activities with the aim of understanding users and data generated by them better.

 With the continuous growth of mobile devices in numbers, the amount of data increases significantly. Due to the wide availability and commonness of smartphones and tablets, anybody can easily generate rich location, media or textual data. Complementary to the popularity of mobile devices, social media sites have grown a lot in the past years \cite{ottoni2013ladies, hu2014we, bakhshi2014faces}, allowing users rich ways of interaction. These facts further increases the call for research in the field of mobile and user data analysis, because it can help researchers to understand the society, human behavior, preferences and public opinions better. 

 Data analysis tools and methods already exist to help businesses to process user data. However, these applications often not utilized, because companies rather focus on developing their service package than understanding the previously gathered data \cite{inmon2007tapping, bigdatamanagementrevolution}. As a result, essential knowledge concerning previously collected data is neglected, which would be key for the future development of the business. Careful data analysis may reveal interesting relationships and point out facts, what human eyes would never notice otherwise. Consequently, Knowledge Discovery in Databases is important and is essential part of Business Intelligence applications as well \cite{zarsky2002mine, bigdatamanagementrevolution}.

 The goal of this research is to study understand the potential use of user data in business and research environments in the present time with the application of Data Mining methods. The research aims to study how content on the Internet is observed by the users, who access the content. Secondly, it is studied what kind of feedback is received from the engaged audience in terms of like activities and comments to the provided content. Thirdly, the tendencies among the demographic characteristics between users are studied in relation to the content which gets them engaged online. 

 One reason behind conducting this research are the stimulating challenges about studying user data and the wide possibilities of the information that may lie around in databases. Previous research have proven the relevance of statistical analysis on data, such as like activities \cite{jang2015noreciprocity, jang2016teensengagemorewithfewerphotos, ottoni2013ladies, guy2016whatsyourorganizationlike, jang2015no}, user comments \cite{jang2016teensengagemorewithfewerphotos}, tags \cite{jang2016teensengagemorewithfewerphotos} and image content \cite{hu2014we, bakhshi2014faces} by revealing interesting findings about user behavior. Moreover, as service providers often get access to user demographics-related data through social network sites in the present time, new possibilities become available to seek correlation between user segments. Studies conducted in this field are also interesting from the point of view of human behavior research, which is another motivation towards conducting studies in this area.

% research questions of this paper
This paper aims to seek answers to the following research questions by discussing relevant literature in the topic: 

\setdefaultleftmargin{40pt}{}{}{}{}{}
\begin{enumerate}[label=RQ\arabic*:]
	\item How is user data understood in previous research and in industrial applications?
	\item Which data mining methods are typically utilized for user data analysis in scientific researches and the industry?
\end{enumerate}  

The rest of this paper is organized as follows: 

\section{Section 2}

\section{Discussion}

\section{Conclusions} 

\bibliographystyle{IEEEtran}
\bibliography{references}

\end{document}
