\documentclass[conference]{IEEEtran}

\usepackage[utf8]{inputenc}
\usepackage{cite}
\ifCLASSINFOpdf
  \usepackage[pdftex]{graphicx}
\else
  \usepackage[dvips]{graphicx}
\fi
\usepackage{amsmath}
\usepackage{array}
\usepackage{url}
\usepackage{paralist}
\usepackage{enumitem}
% correct bad hyphenation here
% \hyphenation{op-tical net-works semi-conduc-tor}

\begin{document}
% paper title
% Titles are generally capitalized except for words such as a, an, and, as,
% at, but, by, for, in, nor, of, on, or, the, to and up, which are usually
% not capitalized unless they are the first or last word of the title.
% Linebreaks \\ can be used within to get better formatting as desired.
% Do not put math or special symbols in the title.
\title{Audience engagement and user data analysis with Data Mining}

\author{\IEEEauthorblockN{Péter Ivanics}
\IEEEauthorblockA{Department of Computer Science \\
University of Helsinki \\
Email: peter.ivanics@helsinki.fi \\
\url{http://pivanics.users.cs.helsinki.fi/portfolio/}}}

\maketitle

\begin{abstract}
%<Context>
Many software businesses collect enormous amount of data generated by end-users. End-user data incorporates essential information about how users interact with the system of discussion as well as information about the users themselves. Such digital footprints can explain the preferences of users, what kind of content they like, how frequent their activity in the platform is or how they interact with the system and eachother.

%<Objectives>
The goal of this research is to study understand the potential use of user data in business and research environments in the present time with the application of Data Mining methods. Since there is no common understanding on its concept, the concept is user data is defined. Secondly, it is studied how user data is utilized in related studies and what kind of results can be derived from such dataset. For example, user data of social media sites can reveal what kind of feedback is received from the engaged audience in terms of like activities and comments to the provided content. Thirdly, the tendencies among the demographic characteristics between users in relation to the engaging content studied.

%<Methods>
The findings are retrieved by reviewing relevant literature. The sources include scientific journals, articles as well as books related to the topic of this study. The snowballing technique was used to retrieve further papers in the field. Furthermore, various keywords were used to obtain papers in this research area.

%<Results>
The research papers in the field show various ways how user data can be used. Social networking sites and movie databases commonly utilize various statistical and data analysis tools on user data. Results show for instance, that Instagram users typically can be divided into two groups - generalists and specialists - based on the variety of the content they like in the software. Studies conducted on social media sites in enterprise environments have given insights to the organization on how their employees cooperate and interact by utilizing a set of data mining tools. As a result, researchers can learn the insights of the user data and look at human behaviour from a different, novel angle.  

\end{abstract}

\begin{IEEEkeywords}
user data, data mining, user behaviour
\end{IEEEkeywords}

\section{Introduction}
Many software businesses collect enormous amount of data generated by end-users \cite{chinesemobilebankingusers, bigdatamanagementrevolution, inmon2007tapping}. End-user data incorporates essential information about how users interact with the system of discussion as well as tells a lot about the users themselves \cite{jang2015noreciprocity, hu2014we, jang2016teensengagemorewithfewerphotos, han2016teensarefrommars, socialdiversityongithub}. For instance, such data can explain the preferences of users, what kind of content they like, how they interact with the system and eachother or how frequently they use the software that has generated the data \cite{youyou2015computer, ottoni2013ladies}.

 Depending on the portfolio of the business, analysis of end-user data can reveal various interesting findings. For instance, in banking industry Big Data tools are often used to analyze demographic characteristics to maintain and establish new client relationships \cite{chinesemobilebankingusers, bigdatamanagementrevolution}. Extracting the information from the data is challenging: most businesses are to collect more data than what is humanly possible to process and to analyze \cite{inmon2007tapping, wegener2010integrating}. As a result, significant part of the knowledge might remain hidden and hence business-critical information remains unseen \cite{inmon2007tapping, wegener2010integrating, introtodatamining, chinesemobilebankingusers}. The computational methods of today's world enable us not only to achieve previously impossible tasks, but also to create tools that may outperform humans \cite{youyou2015computer}. Accordingly, there is a growing need for all businesses to introduce data analysis processes in their daily activities with the aim of understanding users and data generated by them better.

 With the continuous growth of mobile devices in numbers, the amount of data increases significantly. Due to the wide availability and commonness of smartphones and tablets, anybody can easily generate rich location, media or textual data \cite{jang2016teensengagemorewithfewerphotos}. Complementary to the popularity of mobile devices, social media sites have grown a lot in the past years \cite{ottoni2013ladies, hu2014we, bakhshi2014faces}, allowing users rich ways of interaction. Due to the combination of these two trends, users leave digital footprints in form of location, media, numerical or textual data in numerous places around the Internet \cite{youyou2015computer}. People often express their opinion by sharing, liking or commenting on data over social networks. Based on this information, algorithms can predict their personality traits more reliably than other humans would do \cite{youyou2015computer}. These facts further increases the call for research in the field of mobile and user data analysis, because it can help researchers to understand the society, human behavior, preferences and public opinions better.

 Data analysis tools and methods already exist to help businesses to process user data. However, these applications often not utilized, because companies rather focus on developing their service package than understanding the previously gathered data \cite{inmon2007tapping, bigdatamanagementrevolution}. As a result, essential knowledge concerning previously collected data may remain undiscovered, which would be key for the future development of the business. Careful analysis of Big Data may reveal interesting relationships and point out facts, what human eyes would never notice otherwise \cite{bigdatamanagementrevolution}. Consequently, Knowledge Discovery in Databases is important and is essential part of Business Intelligence applications as well \cite{zarsky2002mine, bigdatamanagementrevolution}.

 The goal of this research is to study understand the potential use of user data in business and research environments in the present time with the application of Data Mining methods. The research aims to study how content on the Internet is observed by the users, who access the content. Secondly, it is studied what kind of feedback is received from the engaged audience in terms of like activities and comments to the provided content. Thirdly, the tendencies among the demographic characteristics between users are studied in relation to the content which gets them engaged online. 

 One reason behind conducting this research are the stimulating challenges about studying user data and the wide possibilities of the information that may lie around in databases. Previous research have proven the relevance of statistical analysis on Big Data, such as like activities \cite{jang2015noreciprocity, jang2016teensengagemorewithfewerphotos, ottoni2013ladies, guy2016whatsyourorganizationlike, jang2015no}, user comments \cite{jang2016teensengagemorewithfewerphotos}, tags \cite{jang2016teensengagemorewithfewerphotos}, image content \cite{hu2014we, bakhshi2014faces} and movie ratings \cite{saraee2004data, kabinsingha2012movie} by revealing interesting findings about user behavior. Moreover, as service providers often get access to user demographics-related data through social network sites in the present time, new possibilities become available to seek correlation between user segments. Studies conducted in this field are also interesting from the point of view of human behavior research, which is another motivation towards conducting studies in this area.

% research questions of this paper
This paper aims to seek answers to the following research questions by discussing relevant literature in the topic: 

\setdefaultleftmargin{40pt}{}{}{}{}{}
\begin{enumerate}[label=RQ\arabic*:]
	\item How is user data understood in previous research and in industrial applications?
	\item Which data mining methods are typically utilized for user data analysis in scientific researches and the industry?
\end{enumerate}  

% how is the rest of this paper structured? 
The rest of this paper is organized as follows. The following section presents the selected research methods for this literature review. Section 3 introduces the concept of user data and explains, how is it understood and used in related studies. Section 4 explains the related studies where user data is studied and presents the most commonly used techniques for its analysis. Section 4 discusses the previously presented results by analyzing and comparing the findings of other researchers in the field. Finally, Section 5 concludes the research and pinpoints directions for further studies. 

\section{Research methodology}
The foundations for the theoretical basis are retrieved by reviewing relevant literature. The sources include scientific journals, articles as well as books related to the topic of this study. The two former sources are retrieved through online digital libraries, such as the ACM digital library, IEEE Xplore and Google Scholar. 

After finding the first papers, the snowballing technique was used to retrieve further papers in the field. Furthermore, various keywords were used to obtain related literature in the research area. Keywords included, but were not limited to "data mining", "social media", "user data", "user behavior" and "demographic characteristics". 

\section{The concept of user data}
% catch up the thread from the introduction why user data is important
Careful analysis on Big Data can enhance business domain understanding for any company. Accordingly, Big Data has became a widely studied and interesting topic all around the field Information Technology in the recent years, both in research and industrial environments \cite{inmon2007tapping, introtodatamining}. Significant amount of the data is somehow related to the users of the software at hand. Such data often incorporates essential data about the users themselves, such as their demographic data, preferences or how they interact with the software. Despite its value and availability, this kind of data is often not looked at, however some interesting studies have been conducted in the field which promotes its analysis. On top of that, there is no common agreement in the field what user data means and how it could be used for research purposes.

% what is user data? what does it include or exclude?
User data in this study is understood as data, which is willingly uploaded by users and generated through their interactions while using the software. The research aims to study how content on the Internet is observed by the users, who access the content.

% demographics characteristics in banking
As part of user data analysis, demographic characteristics play a role in analyzing user adoption and behavior in the banking industry. The study conducted by Wang and Petrounias \cite{chinesemobilebankingusers} reveals that mobile banking in China is more popular among middle-aged males, while the younger generation has not adopted to the new trends yet. By utilizing Big Data analytics the group of citizens and products for the upcoming marketing campaigns were revealed \cite{chinesemobilebankingusers}, which greatly enhances the marketing activities of financial organizations. Social diversity was also studied by other researchers in the context of software development growth \cite{socialdiversityongithub}. In their study, Aué et al. have clearly identified correlation between the success of open source projects and the contributors' gender and cultural diversity by utilizing well chosen statistical methods \cite{socialdiversityongithub}.

\section{User data and audience engagement analysis}
% what kind of results were derived in the past from end-user data analysis? 
Movie databases often contain user reviews on movies, actors and producers of all sort. Such databases are open and are available for the public, and therefore the amount of data has grown huge over the past years. Unsurprisingly, databases like the Internet Movie Database (IMDB) has drawn the interest of researchers \cite{saraee2004data, kabinsingha2012movie, sumathi2013performance}. The successful application of statistical methods and data mining techniques have revealed interesting findings, such as that larger budget for movies does not necessarily result in good ratings by the public, while actors have higher impact on the opinion of the audience \cite{saraee2004data}. On top of deriving such conclusions, machine learning techniques are emerging to predict future movie rating data, based on prior reviews of users \cite{saraee2004data} or the analysis of genre and other attributes of movies \cite{kabinsingha2012movie}.

% how do researches on social media website data see the user data analysis?  
Social Networking Sites (SNSs) are another trending source in discovering the secrets of user data as the number of scientific publications in the topic has increased significantly in the recent years \cite{waheed2017investigation}. Various researches have applied advanced data mining techniques on Instagram data \cite{jang2015noreciprocity, bakhshi2014faces, hu2014we, jang2016teensengagemorewithfewerphotos, han2016teensarefrommars}, more specifically on the tags and comments that are attached to the images. Similarly, like activities and user-generated content is studied by the scientists. It is revealed, that Instagram users can be divided into two groups based on their activities: specialists, who publish and seek content around a certain topic of interest; and generalists, who are interested in all kinds of genres in the social media site \cite{jang2015noreciprocity}. Data mining techniques also allowed researchers to conclude, that the teenager users of Instagram tend to be more active, faster to react and more open to communicate with other users on social media \cite{jang2016teensengagemorewithfewerphotos, han2016teensarefrommars}. Furthermore, it was discovered that media content with human faces are more engaging than other type of media \cite{bakhshi2014faces}. Finally, rich social media data allowed researchers to analyze behavior and user preferences among genders, age groups and locations \cite{farseev2015harvestingmultiplesources}.

% facebook reactions
Recently Facebook has introduced reactions among their features. Through reactions, users can not only "like" content, but also express other emotions, such as love, joy, amazement, anger or sadness \cite{shouldfacebookusereactions, howarenewspublishersreactingonfacebook}. This way users' emotional feelings about the content can be collected easily and efficiently. Shortly after its release, it was identified that the new feature is very popular and generally engages a wider audience than previous likes and comments \cite{shouldfacebookusereactions}. Study also shows, that reactions are a great way for publishers to get a feedback on the public's opinion \cite{howarenewspublishersreactingonfacebook}. On top of that, reactions offer an easy way for content providers to organize a  by assigning one of the available reactions to the participants and asking the users to react on the content with their favorite's reaction \cite{shouldfacebookusereactions}. The limitation on such polls is that users have the possibility to choose only one of the reactions and hence only one of the participants as their answer to the voting.  

% where is all this research coming from?
Interestingly, most of the SNS-related studies are conducted in the United States of America and Asia \cite{waheed2017investigation}. Only a few studies were conducted in the European region, which allows to conclude that there is a great interest and development possibility in the area. However it is important to highlight, that due to the wide popularity of the international social networks (such as Facebook, Instagram or Pintrest), some part of the data may be derived from users in the European continent. It was also pointed out that some researches focus on sites, that are specific to a particular region or country \cite{waheed2017investigation}, which means the findings are strongly related to the cultural environment of the user base.    

% what does user data analysis reveal in a corporate social media environment? [refer to \cite{guy2016whatsyourorganizationlike}]
Social media platforms in enterprise environment are studied similarly to regular social networks \cite{guy2016whatsyourorganizationlike}. Despite the fact that the two types of social media platforms share many features, analysis performed in enterprise environment can provide great insights about how employees interact and cooperate. Among many other findings, studies have shown that blogs posts in an enterprise social media site tend to be more engaging and contribute to form communities inside the organization \cite{guy2016whatsyourorganizationlike}. Such insights are essential for higher management, because it can be used for instance to identify departments that tend to be less interactive or engaged.  

% structured and unstructured data is being investigated by different researchers

% why are these researches interesting? What do we learn from the society by performing analysis on user data? 
The brief examples above demonstrate the capabilities of data mining in the field of user data and demographic analysis. The proper application of data mining methods allow us to learn more about the society as well as human behavior, which was not possible in the past. This information is essential for business operations, because it gives insights on the user groups, their preferences and what kind of content keeps the audience engaged. In the past these kind of insights were unavailable to managers and content providers. In the present time, access to this information can facilitate business processes, helps determining the future content, analyzing trends and understanding target groups of particular services better. In sum, modern data mining techniques created the potential to study human preferences and behavior from a different angle. 
\section{Discussion}

\section{Conclusions} 

\bibliographystyle{IEEEtran}
\bibliography{references}

\end{document}
