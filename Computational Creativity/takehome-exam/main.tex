\documentclass[english]{tktltiki}
    
    %\usepackage[english]{babel}
    %\usepackage[utf8]{inputenc}
    \usepackage{amsmath}
    \usepackage{graphicx}
    \usepackage{amsfonts}
    %\usepackage[pdftex]{graphicx}
    %\usepackage{subfigure}
    \usepackage{url}
    \usepackage{footnote}

    \begin{document}
    \singlespacing
    
    \title{Computational Creativity: \\
    Take-Home Exam}
    
    \author{P\'eter Ivanics \\
    (014687752)}
    
    \date{\today}
    
    \maketitle
    
    \pagenumbering{arabic}
    
    \section{Project Overview (max 1 page)}
    The main goal of the created system is to generate plot points of a fictional book, movie or a fictional story. The output is a readable, sequential list of events that happen in the story, written in English. The generated stories are comprehensive and are easily readable without technical background. 
    
    The Knowledge Base of the system can be divided into two parts: character and object database and the logic of the story simulation. The former database is constructed from The Non-Official Characterization (NOC) List\footnote{\url{https://github.com/prosecconetwork/The-NOC-List}}, which is a knowledge-base containing semantic triples about famous characters. This knowledge base is used to filter and construct the characters as well as the objects that appear in the generated stories. On top of The NOC List, the knowledge base of the created system incorporates some "hard-coded" values, which explain what kind of interactions are possible between the characters (under certain circumstances). Two characters can perform any of the following actions to eachother based on their relationship: "kill", "beat up", "insult", "converse".
    
    The conceptualization of the stories happens using the data above and the rules explained below. The simulation happens in turns when characters move around in the world.  Characters' relationships are represented with a floating number on the range of $[-1, 1]$, where negative values mean hatred, $0$ is neutrality and positive value is friendship. As characters interact, their relationship changes directly and indirectly. Relationships are initialized as $-1$ between opposing characters. Characters may have their own goal to chase during the simulation. Goals can be to find an object in the world, to become friends with somebody or to kill another character. Items may be located at locations or owned by other characters. A shop and a house is generated for every character, which creates the set of locations in the world. The world also has a tavern, which is another location. Characters can move between locations freely. If a character dies, it drops out of the simulation. 

    The generation of stories happens starts based on the input parameters: how many characters to choose, which fictional domain to use and how to pick the characters (randomly or sequentially). Using the rules above, a sequence of events is generated in the story. A snapshot of the world is taken in every turn during the generation, which is later used for the evaluation of the story. The next taken action during the story always motivated by the characters' individual goals or chosen randomly, if they have no goals. A genotype consists of the full sequence of events and snapshots that has happened during the story. 

    The genotypes are evaluated based on how much the final state of the world has changed compared to the initial world. The evaluation takes into consideration if characters were chosen across domains and how the story builds up. This means that the story starts "slowly" with more neutral actions between characters, which turn into more friendly or violent by the end of the story.
    
    \pagebreak
    
    \section{Creativity as Search}
    
    \paragraph{a. Description of the system as search (max 0.6 pages)}
    
    [Describe your project (or a central part of if described above) using Wiggins' framework of creativity as search. Provide a brief rationale for each choice of description; also point out obvious alternatives.]
    
    \paragraph{b. Transformationality of the system (max 0.4 pages)}
    
    [Is your system transformationally creative? If yes, how? If not, describe two different ways of extending the system to be transformational with respect to $\mathcal{T}$ or $\mathcal{R}$ of Wiggins.]
    
    
    \pagebreak
    
    \section{Aspects of self-awareness}
    
    \paragraph{a. Generator-awareness (max 0.5 pages)}
    [Describe generator-awareness in your system.]
    
    \paragraph{b. Goal-awareness (max 0.5 pages)} 
    [Describe goal-awareness in your system.]
    
    
    
    \pagebreak
    \section{Evaluation}
    
    \paragraph{a. How PPPPerspectives are reflected in your system? (max 0.6 pages)}
    
    [Consider Jordanous' four PPPPerspectives on computational creativity in the context of your project. Describe what each P in your project consists of and provide a brief rationale for each choice.]
    
    \paragraph{b. Value in each PPPPerspecive (max 0.4 pages) }
    
    [Consider the value (cf. "novelty and value" as criteria for creativity) of your system from each of the four Perspectives.]
    
    \pagebreak
    \section{Universality of creativity (max 0.4 pages)}
    Creativity is a universal concept, because it fundamentally means the seek after novel and valuable ideas/artifacts (based on Boden's definition). Novelty and value are attributes, which can be identified in practically any domain where the process of creation is present in some way. 
    
    Creativity is also subjective in many cases (e.g. art), which creates a good basis of discussion. On top of that, typically the aspects of value and novelty are easy to relate to, especially in case of "everyday" topics. Depending on the experience and the cultural background of the individual, who is evaluating, the elaboration on the creativity of an artifact may differ. Last but not least, creativity is not a trivial topic in any domains and therefore the process of creative artifact creation cannot be modelled in a straightforward manner. Hence, creativity is universally interesting topic of discussion.
    
    \section{Social creativity and artifact representations}
    
    [Imagine a society of creative agents generating artifacts in the same domain. 
    What are the benefits and challenges for the agents if they share the genotypic 
    representation of the domain artifacts (see Ventura's paper "How to build a CC system")? 
    That is, how can the agents use the genotypic representations in their interaction 
    with other agents: what an individual agent and the society as a whole gains from it.]
    
    \paragraph{a. The agents create artifacts on their own (max. 0.3 pages, 3 points)}
    
    \paragraph{b. the agents co-create artifacts in pairs (max. 0.3 pages, 3 points)}
    
    \end{document}