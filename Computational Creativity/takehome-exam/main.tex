\documentclass[english]{tktltiki}
    
    %\usepackage[english]{babel}
    %\usepackage[utf8]{inputenc}
    \usepackage{amsmath}
    \usepackage{graphicx}
    \usepackage{amsfonts}
    %\usepackage[pdftex]{graphicx}
    %\usepackage{subfigure}
    \usepackage{url}
    \usepackage{footnote}
    \usepackage{enumitem}

    \begin{document}
    \singlespacing
    
    \title{Computational Creativity: \\
    Take-Home Exam}
    
    \author{P\'eter Ivanics \\
    (014687752)}
    
    \date{\today}
    
    \maketitle
    
    \pagenumbering{arabic}
    
    \section{Project Overview}
    The main goal of the created system is to generate plot points of a fictional book, movie or story. The output is a readable, sequential list of events that happen in the story, written in English. The generated stories are comprehensive and are easily readable without technical background. 
    
    The Knowledge Base of the system can be divided into two parts: character and object database and the logic of the story simulation. The former database is constructed from The Non-Official Characterization (NOC) List\footnote{\url{https://github.com/prosecconetwork/The-NOC-List}}, which is a knowledge-base containing semantic triples about famous characters. This knowledge base is used to filter and construct the characters as well as the objects that appear in the generated stories. On top of The NOC List, the knowledge base of the created system incorporates some "hard-coded" values, which explain what kind of interactions are possible between the characters (under certain circumstances). Two characters can perform any of the following actions to eachother based on their relationship: "kill", "beat up", "insult", "converse".
    
    The conceptualization of the stories happens using the data above and the rules explained below. The simulation happens in turns when characters move around in the world.  Characters' relationships are represented with a floating number on the range of $[-1, 1]$, where negative values mean hatred, $0$ is neutrality and positive value is friendship. As characters interact, their relationship changes directly and indirectly. Relationships are initialized as $-1$ between opposing characters. Characters may have their own goal to chase during the simulation. Goals can be to find an object in the world, to become friends with somebody or to kill another character. Items may be located at locations or owned by other characters. A shop and a house is generated for every character, which creates the set of locations in the world. The world also has a tavern, which is another location. Characters can move between locations freely. If a character dies, it drops out of the simulation. 

    The generation of stories happens starts based on the input parameters: how many characters to choose, which fictional domain to use and how to pick the characters (randomly or sequentially). Using the rules above, a sequence of events is generated in the story. A snapshot of the world is taken in every turn during the generation, which is later used for the evaluation of the story. The next taken action during the story always motivated by the characters' individual goals or chosen randomly, if they have no goals. A genotype consists of the full sequence of events and snapshots that has happened during the story. 

    The genotypes are evaluated based on how much the final state of the world has changed compared to the initial world. The evaluation takes into consideration if characters were chosen across domains and how the story builds up. This means that the story starts "slowly" with more neutral actions between characters, which turn into more friendly or violent by the end of the story.
    
    \pagebreak
    
    \section{Creativity as Search}
    \paragraph{a.}
    The system that we built belongs mainly to the Exploratory creativity type, but it also shows some signs of Combinatorial creativity as it can create stories with characters accross domains. The universe $\mathcal{U}$ in this project would be all sequences of plot points that are ever possible to generate. This would mean any combination of characters, events and interactions, not limited to our choices described on the previous page. 
    
    The rules $\mathcal{R}$ in this project are mainly formulated by the chosen list of characters for every simulation (extracted from the NOC List) and interactions that can happen between them. On top of that, the logic which simulates the changes between the relationship's value between the characters is also an essential part of $\mathcal{R}$. There are many alternatives to generate a genotypes of the stories, such as having entirely different model for the character relationships, a more extensive list of interactions between them or completely different set of characters. It would be interesting to extract these automatically from a corpus of stories automatically instead of relying on the current resources, which would make an entirely different way of generation towards the same goal. 

    The evaluation function $\mathcal{E}$ is the logic that maps the creativity value on the range of $[0, 1]$ to the stories. $\mathcal{E}$ in the current implementation operates on the snapshots of the states during the stories and the events that happen during the simulation. The evaluation favors stories sequences of events, where the plot unfolds and changes significantly between the initial and the terminal states. The reason behind this scoring system is that it implies the unexpectedness and surprisingness of the story.

    The search method $\mathcal{T}$ in Wiggings' classical terminology is not fully implemented, because our system does not combine $\mathcal{R}$ and $\mathcal{E}$ jointly to create artifacts. It would be possible to design a search function which experiments with the input parameters of the generator and orients their selection towards stories with higher value as assigned by $\mathcal{E}$. $\mathcal{T}$ could be also added by combining valuable pieces of multiples stories created during the generation. 
    
    \paragraph{b.} Our system is not transformationally creative. One way to make it could be, is to extend the knowledge base with a corpus of stories and dynamically change how characters behave during every simulation. $\mathcal{R}$ could be transformed in a manner that the characters' movements are not entirely random and the logic that selects actions between them changes over time, depending on what kind of events have happened. The goal-orientation of characters could be developed further, which could change over time depending on what kind of events happen in the world. 
    
    $\mathcal{T}$ could be made transformational by allowing the search function to dynamically modify tha parameters of a generator at some point of the story. For instance, $\mathcal{T}$ could merge the concepts of two valuable genotype stories in order to come up with more interesting artifacts.
    
    \pagebreak
    
    \section{Aspects of self-awareness}
    \paragraph{a. Generator-awareness}

    Our system in its current state does not implement generator awareness. However, generator awareness could be added to the system by introducing an entity (e.g. Generator Manager), which can access the knowledge base, the generator and the evaluation functions. The manager could intuitively experiment with the input parameters of the generator, i.e. the number of characters to choose, the list of domains to choose the characters from, the number of iterations to perform during the story etc.

    This experimentation could mean, for instance that the generator manager generates $\mathcal{N}$ stories using a set of parameters, then mutates the parameters in some way(s) and attempts to generate new stories. Depending on the average difference between the success score of the stories, the manager could decide to keep changing the parameters in the same way (i.e. continue searching in the same direction) or to change its strategy in tuning the parameters (i.e. changing direction in the search). 
    
    For instance, if the average score for stories with 5 characters is 0.5, while a 7-character story produces 0.55, it might be a good idea to keep increasing the number of characters further until certain point. This could be applied to all of the parameters so that the search space can be walked thoroughly. However, it is important to keep in mind that the simulation includes certain level of randomness and hence one single story with a given parameters might not be representative.
    
    \paragraph{b. Goal-awareness} 
    Our system in its current state does not implement goal awareness. The system could be enhanced to be goal aware by allowing more goal-oriented parameters or the introduction of a Goal Manager. The generator already can be parameterized in terms of how many steps the generator should take and how many characters to use, however some additions could be made, such as: 

    \begin{itemize}
        \item which characters to use specifically,
        \item what are the initial/final relationships between the characters,
        \item which characters (do not) survive the story,
        \item what goals the individual characters have (this is randomized currently),
        \item which of the characters' goals are (not) achieved.
    \end{itemize}

    These parameters could be expressed in terms of some ways and passed to the generator. The generator then should be able to figure a way out how to generate a story, where all of the goals are met. This would probably mean the limitations on some of the other parameters (e.g. number of simulation turns, in particular), the possibility that the generator runs very long or that the story cannot be generated at all.

    \pagebreak
    \section{Evaluation}
    
    \paragraph{a. How PPPPerspectives are reflected in your system? (max 0.6 pages)}
    
    [Consider Jordanous' four PPPPerspectives on computational creativity in the context of your project. Describe what each P in your project consists of and provide a brief rationale for each choice.]
    
    \paragraph{b. Value in each PPPPerspecive (max 0.4 pages) }
    
    [Consider the value (cf. "novelty and value" as criteria for creativity) of your system from each of the four Perspectives.]
    
    \pagebreak
    \section{Universality of creativity (max 0.4 pages)}
    Creativity is a universal concept, because it fundamentally means the seek after novel and valuable ideas/artifacts (based on Boden's definition). Novelty and value are attributes, which can be identified in practically any domain where the process of creation is present in some way. 
    
    Creativity is also subjective in many cases (e.g. art), which creates a good basis of discussion. On top of that, typically the aspects of value and novelty are easy to relate to, especially in case of "everyday" topics. Depending on the experience and the cultural background of the individual, who is evaluating, the elaboration on the creativity of an artifact may differ. Last but not least, creativity is not a trivial topic in any domains and therefore the process of creative artifact creation cannot be modelled in a straightforward manner. Hence, creativity is universally interesting topic of discussion.
    
    \section{Social creativity and artifact representations}
    
    \paragraph{a.} If the agents create artifacts on their own, they will be more close to the personal background of each individual agent (refer to Cs\'ikszentmih\'alyi). In terms of computing, this would mean the parameters of the generator functions used by the agents. The genotypic representations of artifacts and their scores can be used to select the most creative artifacts from the population. Weakly performing agents' parameters then can be tuned towards values that seem to create better outputs. This way weaker agents learn from the "best" agent in their society, which enhances the creativity of individuals, but not the development of the society as a whole. In other words, there is a risk that the generation gets stuck at a local maximum, which is determined by the output of the "best" agent.

    \paragraph{b.} If the agents co-create artifacts in pairs, the collaboration of the agents in the society is facilitated more. That is, the agents not only exchange information about their personal backgrounds, but combine their strengths in pairs. Theoretically, this way creates more room for new creative artifacts, if the pair of agents are not the same in every turn. This yields in a more advanced society of agents where depending on how the pairs are made, novel artifacts can be created any time and the average value of artifacts might be higher. 
    
    \end{document}