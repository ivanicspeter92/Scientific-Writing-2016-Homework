\documentclass[english]{tktltiki}
\usepackage[pdftex]{graphicx}
\usepackage{subfigure}
\usepackage{url}
\begin{document}
%\doublespacing
%\singlespacing
\onehalfspacing

\title{Visualizing data in the field of mobile workforce management}
\author{P�ter Ivanics}
\date{\today}

\maketitle

\numberofpagesinformation{\numberofpages\ pages + \numberofappendixpages\ appendices}
\keywords{mobile workforce, data visualization}

%?\begin{abstract}
	% \TODO Abstract comes here
%\end{abstract}

\mytableofcontents

\section{Introduction} % \review
	Mobile devices became part of our lives. The number of mobile phones reached more than 4 billion devices worldwide in 2010 already \cite{bau10}. In the present time, almost every person on the planet has at least one mobile device, such as a smart phone or a tablet, which they carry on with themselves all the time, wherever they go. At the same time, new devices are approach the horizon, such as smart watches and glasses, which further increase the popularity mobile devices.
	
	According to Baudisch and Holz, desktop computers are fading into the past \cite{bau10}. Despite all the challenges, such as small resolution, screen size, new interaction techniques, weaker hardware resources etc., many applications that run on personal computers support mobile displays and operating systems. The appreciation from the users' side is storming, because these devices enable them to perform everyday tasks on a device, which is in their pocket \cite{bau10}. 
	
	Mobility has become a key factor, not only for individuals, but businesses as well \cite{wvms06, bud15a}. More and more businesses implement mobile solutions in some ways by providing their employees smart phones in order to perform some (if not all) of their work. Accodring to Buda \cite{bud15a} this means that previous methods, such as paper-based administration is disappearing and being replaced by mobile forms. By bringing the smart phones into the working life of their employees, businesses are moving towards the creation of a mobile workforce. Razip et al. \cite{rma14} agree that the potential and the popularity of mobile hand-held devices is increasing and many businesses started to utilize them recently with the aim to improve business processes.
	
	The goal of a mobile workforce is to become more efficient and productive by replacing the paper-based field operations with mobile solutions \cite{wvms06, bud15a}. However, in many cases the decision makers do not clearly understand the impact and the cost of such change or might not make the best possible choice from the long list of possibilities. This means the there are many different technologies and providers, however not all of the is suitable for every business case. Wang, van de Kar, Meijer and Sol explains \cite{wvms06} that the available solutions are typically not designed for a specific organizational environment and require a lot of adaptation as well as business engineering. In other words, utilization of mobile technologies may have an impact on the organizational structure, processes and individuals' role at a firm. Finally, the new technology brings enormous amount of digital data, which should be analyzed, understood and handled.
	
	This report first introduces the concept of a mobile workforce. Further the study explains the popularity, main goals, challenges and some of application fields of mobile workforce solutions by drawing from relevant literature. The study focuses on visualizing the data generated by a mobile workforce. 
	
	Reslink Solutions is a company situated in Helsinki who provides competitive solutions for mobile workforce management in various business fields. As a case study for mobile workforce management and data visualization, some of the tools and practices in use at Reslink Solutions are presented.

\section{Mobile workforce management} % \TODO
	\subsection{Basic principles and problem sphere}
		Mobile solutions are undoubtedly influental in the present time on a global scale. Case studies show application of a mobile workforce solution in various industries, such as brewery \cite{bud15a}, law enforcement/public security \cite{rma14, pwe09}, public transportation \cite{wvms06} and guarding \cite{bud15b}. 
		\begin{itemize} 
			\item what is a mobile workforce? 
			\item how do we understand mobile workforce in this paper
			\item what problem mobile workforce is intended to solve?
			\item why are mobile workforce solutions powerful
		\end{itemize}
		
	\subsection{Challenges}
		\begin{itemize}
			\item what are the challenges of mobile workforce management
				\begin{itemize}
					\item for businesses
					\item for end-users/employees
					\item for managers
				\end{itemize}
			\item how to represent data in an efficient way?
			\begin{itemize}
				\item on the mobile devices
				\item in the back office
			\end{itemize}
			\item what kind of features are essential to coordinate a mobile workforce
		\end{itemize}

\section{Case studies}
	\subsection{Reslink Solutions Oy}
		\begin{itemize}
			\item what is the company about?
			\item what is the motivation of describing it here? why is this presented as a case study?
			\item which business verticals Reslink is operating in?
			\item which vertical will we focus on in this paper?
			\item what challenges the company is facing? 
			\item how is the data structured and represented? 
			\item how employees and managers get access to the data and what can they see and do with it? 
			\item how did the mobile workforce change the working life of the employees and managers compared to previous solutions? 
		\end{itemize}
		
		\subsection{AllAboard}
		\subsection{Interactive Tabletops and Surfaces}
		\subsection{Crime incident analysis}
			This research addresses the problem and proposes a solution for enhancing public safety based on mobile data collection and analysis. By utilizing this solution, police officers and citizens are able to report incidents (e.g. traffic collisions, interference, violence etc.) from their mobile phones in their surroundings. The system relies on the GIS information retrieved from the mobile phone and registers the incident in a central database. 
		
			The submitted data is available for patrolling officers nearby in real time. This gives the possibility to react to incidents happening on the field immediately. In the back office the shift supervisors have the capability to reallocate officers on patrol, in case an area is overwhelmed with suspicious incidents. Furthermore, all users are able to see statistics of certain areas, incident types, tendencies and forecasts. The solution offers various views to visualize data, such as:
			\begin{itemize}
				\item Time Series View: represents the incident on a timeline,
				\item Calendar View: summarizes the incidents for each day in monthly calendars,
				\item Clock View: summarizes the incidents for each hour of a day on the clock.
			\end{itemize}
		
			In addition to that, maps are utilized to display the places where incidents were encountered. The heat map feature allows users to get a clear understanding about areas that tend to be more dangerous in certain time-frames. 
		
			The system is optimized and available on desktop devices, iOS phones and tablets. From the research it is unclear, if there is any support for other platforms, such as Android or Windows phone. Based on the content of the research paper, there is room for improvement in the analytical and forecasting module as well.  \cite{rma14}

\section{Conclusions} % \TODO

\pagebreak

\nocite{*}
\bibliographystyle{tktl}
\bibliography{bibliorgaphy}

\lastpage
\appendices
\pagestyle{empty}
\end{document}


