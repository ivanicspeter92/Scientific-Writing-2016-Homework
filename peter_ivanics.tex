\documentclass[english]{tktltiki}
\usepackage[pdftex]{graphicx}
\usepackage{subfigure}
\usepackage{url}
\begin{document}
%\doublespacing
%\singlespacing
\onehalfspacing

\title{Data visualization in the field of mobile workforce management}
\author{P�ter Ivanics}
\date{\today}

\maketitle

\numberofpagesinformation{\numberofpages\ pages + \numberofappendixpages\ appendices}
\keywords{mobile workforce, data visualization}

%?\begin{abstract}
	% \TODO Abstract comes here
%\end{abstract}

\mytableofcontents

\section{Introduction} % \review
	Mobile devices became part of our lives. The number of mobile phones reached more than 4 billion devices worldwide in 2010 already \cite{bau10}. In the present time, almost every person on the planet has at least one mobile device, such as a smart phone or a tablet, which they carry on with themselves all the time, wherever they go. Additionally, new devices are approach the horizon, such as smart watches and glasses, which further widen the popularity mobile devices.
	
	Baudisch and Holz states, that desktop computers are fading into the past \cite{bau10}. Despite all the challenges, such as small resolution, screen size, new interaction techniques, weaker hardware resources etc., many applications that run on personal computers support mobile displays and operating systems. The appreciation from the users' side is storming, because these devices enable them to perform everyday tasks on a tool, which is in their pocket \cite{bau10}. 
	
	Mobility has become a key factor, not only for individuals, but businesses as well \cite{wvms06, bud15a}. More and more businesses implement mobile solutions in some ways by providing their employees smart phones in order to perform some (if not all) of their work. Accodring to Buda \cite{bud15a} this means that previous methods, such as paper-based administration is being replaced by mobile forms. By bringing the smart phones into the working life of their employees, businesses are moving towards the creation of a mobile workforce. Razip et al. \cite{rma14} explain that the business potential of mobile hand-held devices is widely known and therefore many businesses started to utilize them recently with the aim to improve business processes.
	
	The goal of a mobile workforce is to become more efficient and productive by replacing the paper-based field operations with mobile solutions \cite{wvms06, bud15a}. However, in many cases the decision makers do not clearly understand the impact and the cost of such change or might not make the best possible choice from the long list of possibilities \cite{wvms06}. This means that there are many different technologies and providers to choose from, however typically they are not suitable for every business case. Wang, van de Kar, Meijer and Sol explains \cite{wvms06} that the available solutions are typically not designed for a specific organizational environment and require a lot of adaptation as well as business engineering. This means that the utilization of mobile technologies is likely to have an impact on the organizational structure, processes and individuals' role at a firm. On top of that, the new technology brings enormous amount of digital data, which should be analyzed, understood and handled.
	
	This report first introduces the concept of a mobile workforce. Further the study explains the popularity, main goals, challenges and some of application fields of mobile workforce solutions by drawing from relevant literature. The study focuses on visualizing the data generated by a mobile workforce. 
	
	Reslink Solutions is a company situated in Helsinki who provides solutions competitive solutions for various business fields. As a case study for mobile workforce management and data visualization, some of the tools and practices in use at Reslink Solutions are presented.

\section{Mobile workforce management} % \TODO
	\subsection{Basic principles and problem sphere}
		Mobile solutions are undoubtedly influental in the present time on a global scale. Case studies show application of a mobile workforce solution in various industries, such as brewery \cite{bud15a}, law enforcement/public security \cite{rma14, pwe09}, public transportation \cite{wvms06} and guarding \cite{bud15b}. The different application fields above prove the flexibility and wide applicability of mobile solutions, furthermore they serve as great examples for successful implementation of a mobile workforce. Despite its popularity, mobile workforce solutions are yet growing and may be understood differently by individuals.
		
		% what is a mobile workforce? 		
		Techopedia defines mobile workforce as a group of employees at a company, who utilize mobile devices to perform their work in the field, at distributed locations \cite{tec16}. By utilizing mobile solutions, employees have the possibility to report when they started and finished their work, prove where they have been, take photos or videos of the circumstances and finally report all of this information to the back-office in real time. Due to the fact that smart devices have access to stable Internet connection most of the time as well as have large data storage space, the collected data can reported to and processed in real time in the back office.
		
		%what problem mobile workforce is intended to solve?
		Depending on the business case, the above aspects can be crucial. For example, Razip et al. \cite{rma14} highlights that geographic information systems (GIS) play a key role in law enforcement, as police officers can immediately react if they are aware of the location of ongoing incidents. Buda states \cite{bud15a} that in brewery industry, proof of visit is essential - managers have to know, that the delegated maintenance job was performed at their customers. In this case, mobile solutions are enablers for proving a completion of an installation by sending GIS coordinates, taking picture of the installed equipment or by asking a signature of the customer. Because of the nearly real time data processing in the back office, managers can get an understanding on what is happening on the field and therefore the value behind such system is enormous. 
		
		%why are mobile workforce solutions powerful
		Today's mobile solutions offer many other benefits. Employees have the possibility to interact with their fellow colleagues free of charge, navigate between locations easily during their workday, keep track of their tasks and plan their workday through calendar and so on. Razip et al. concludes \cite{rma14} that the goal of such mobile solutions is to improve the day-by-day operations of employees by utilizing the tools and practices above. All in all, in comparison to traditional, paper-based work and administration, mobile solutions offer robust solutions to facilitate business processes. 
		
	\subsection{Challenges}
	% the limitations/challenges of the mobiel devices
	Despite all the benefits, mobile devices bring a lot of challenges as well. Razip et al. explains \cite{rma14} that the small screen size and considerably low hardware resources state boundaries on capabilities in most cases. This means that any data represented on the mobile phones must be compact, easy to understand and simple. While desktop devices have more possibilities in the sense of data representation, mobile devices have limitations, which have to be addressed. On top of that, the users must learn how to interact with the devices and learn the touchscreen-interactions in advance, which may be more challenging for elderly employees. Finally, it must be highlighted that some devices may not be the best decision to perform certain task. For instance, filling a long form on a tablet is certainly less demanding than on a mobile phone, while taking a picture with the camera of the device is easier the other way around. All in all, before handing the devices over to the employees, it must be analyzed, what kind of tasks they are desired to accomplish. 
		\begin{itemize}
			\item what are the challenges of mobile workforce management
				\begin{itemize}
					\item for businesses
					\item for end-users/employees
					\item for managers
				\end{itemize}
			\item how to represent data in an efficient way?
			\begin{itemize}
				\item on the mobile devices
				\item in the back office
			\end{itemize}
			\item what kind of features are essential to coordinate a mobile workforce
		\end{itemize}

\section{Case studies}
	\subsection{Reslink Solutions Oy}
		\begin{itemize}
			\item what is the company about?
			\item what is the motivation of describing it here? why is this presented as a case study?
			\item which business verticals Reslink is operating in?
			\item which vertical will we focus on in this paper?
			\item what challenges the company is facing? 
			\item how is the data structured and represented? 
			\item how employees and managers get access to the data and what can they see and do with it? 
			\item how did the mobile workforce change the working life of the employees and managers compared to previous solutions? 
		\end{itemize}
		
\pagebreak
\section{Conclusions} % \TODO

\pagebreak

\nocite{*}
\bibliographystyle{tktl}
\bibliography{bibliorgaphy}

\lastpage
\appendices
\pagestyle{empty}
\end{document}


