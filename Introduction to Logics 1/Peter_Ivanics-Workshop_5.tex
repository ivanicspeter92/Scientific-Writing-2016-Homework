\documentclass[english]{tktltiki}
\usepackage[pdftex]{graphicx}
\usepackage{subfigure}
\usepackage{booktabs}
\usepackage{url}
\usepackage{amsthm,amssymb}
 \usepackage{amsmath}
\begin{document}
\onehalfspacing

\title{Workshop 5}
\author{P�ter Ivanics}
\date{\today}

\maketitle

\section{Problem 1}
In both subproblems below, we prove that there are is no natural deduction of the formulas by presenting a valuation such that $V(A) \neq V(B)$.
\begin{enumerate}
	\item $p_0$ from $p_0 \vee p_1$ \\
		Let $A = p_0$ and $B = p_0 \vee p_1$. Let us assign $V(p_0) = 0$ and $V(p_1) = 1$ . 
		
		In this case 
		\begin{eqnarray*}
		V(A) &=& 0 \\
		V(B) &=& 0 \vee 1 = 1
		\end{eqnarray*}
				
		which yields $V(A) \neq V(B)$.
		
	\item $\neg p_0 \wedge \neg p_1$ from $\neg (p_0 \wedge p_1)$
	
	Let $A = \neg p_0 \wedge \neg p_1$ and $B = \neg (p_0 \wedge p_1)$. Let us assign $V(p_0) = 0$ and $V(p_1) = 1$ . 
		
		In this case 
		\begin{eqnarray*}
			V(A) &=& \neg 0 \wedge \neg 1 = 1 \wedge 0 = 0 \\
			V(B) &=& \neg (0 \wedge 1) = 1 \vee 0 = 1 
		\end{eqnarray*}				
		which yields $V(A) \neq V(B)$.
\end{enumerate}

\section{Problem 2}

\section{Problem 3}

\section{Problem 4}

\section{Problem 5}

\section{Problem 6}

\lastpage

\end{document}