\documentclass[english]{tktltiki}
\usepackage[pdftex]{graphicx}
\usepackage{subfigure}
\usepackage{booktabs}
\usepackage{url}
\usepackage{amsthm,amssymb}
 \usepackage{amsmath}
\begin{document}
\onehalfspacing

\title{Workshop 3}
\author{P�ter Ivanics}
\date{\today}

\maketitle

\section{Problem 1}
The truth table of the truth function defined by the given formula $(p_0 \iff p_1) \rightarrow (p_0 \vee p_1)$ is, as follows: 
\begin{center}
\begin{tabular}{c|c||c|c||c}
\toprule
$p_0$ & $p_1$ & $(p_0 \iff p_1)$ & $(p_0 \vee p_1)$ & $f(p_0, p_1) = (p_0 \iff p_1) \rightarrow (p_0 \vee p_1)$ \\ 
\midrule
0 & 0 & 1 & 0 & 0 \\
0 & 1 & 0 & 1 & 1 \\
1 & 0 & 0 & 1 & 1 \\
1 & 1 & 1 & 1 & 1
\end{tabular}
\end{center}

We can see that this function is the same as the OR ($\vee$) function, which means is equal to $(p_0 \vee p_1)$. 

\section{Problem 2}



\section{Problem 3}
First let us draw the truth table of the NOR ($\downarrow$) function:
\begin{center}
\begin{tabular}{c|c||c}
\toprule
$p_0$ & $p_1$ & $(p_0 \downarrow p_1)$ \\ 
\midrule
0 & 0 & 1 \\
0 & 1 & 0 \\
1 & 0 & 0 \\
1 & 1 & 0 
\end{tabular}
\end{center}

To show that $\{ \downarrow \}$ is a universal set of connectives first we need to observe that the formula is satisfied if and only if all input variables are $0$, formally $f(0, 0, ..., 0) = 1$. Formally, this yields in the following function: 
\begin{center}
	$f(p_0, p_1) = 	
	\begin{cases}
	1, \mbox{if } p_0 = 0 \mbox{ and } p_1 = 0 \\
	0, \mbox{ otherwise}
	\end{cases}
	$
\end{center}

\section{Problem 4}
By definition, disjunctive normal form (DNF) is disjunction of $A_1 \vee A_2 ... \vee A_3$, where $A_n$ is the in the form of $p_1 \wedge p_2 ... \wedge p_m$. In other words, a sentence is in DNF if it is a disjunction and each of its disjuncts ($A_n$) is a conjunction of literals. Intuitively, this means $ (p_0 \wedge p_1 ... \wedge p_m) \vee (p_0 \wedge p_1 ... \wedge p_m) ... \vee (p_0 \wedge p_1 ... \wedge p_m) $ lookalike formulas. Negation of the symbols $p_1, p_2 ... p_m$ is allowed in the DNF. 

According to this definition, the following formulas are evaluated as DNFs. 

\begin{enumerate}
	\item $p_1$: is a standalone variable and does not include any other connectives and only one literal. This formula is in DNF and considered to be a special case of DNF. 
	\item $p_0 \vee p_1$: is similar to the function above but includes two symbols. Due to the fact that these symbols are connected with a single disjunction, it is also its main connective and therefore it is in DNF.
	\item $\neg p_0 \wedge p_1$: this formula is in DNF. Despite the fact that its main connective is an AND ($\wedge$) operator, it is performed on two literals and creates a single clause of the DNF. This is another, special case of the DNF, where the formula contains only a single clause. 
	\item $p_0 \iff p_2$: is not in DNF, because it contains an equivalence ($\iff$) operation. 
	\item $(\neg p_0 \wedge p_1 \wedge p_2) \vee (p_0 \wedge \neg p_1 \wedge p_3) \vee (\neg p_0 \wedge p_2 \wedge p_3)$: is in DNF, because the formula contains only AND ($\wedge$) and OR ($\vee$) connectives in the right format. 
	\item $(\neg p_0 \vee p_1 \vee p_2) \wedge (p_0 \vee \neg p_1 \vee p_3) \wedge (\neg p_0 \vee p_2 \vee p_3)$: is not in DNF. Despite the fact that it contains only the allowed connectives, the format of the formula is not correct, because the symbols are connected directly with OR ($\vee$) connective. These format of formulas are often referred to as as Conjunctive Normal Form (CMF).
\end{enumerate}

\section{Problem 5}
\begin{enumerate}
	\item $p_0 \vee p_1$: 
	
	\begin{center}
		\begin{tabular}{c|c||c}
		\toprule
		$p_0$ & $p_1$ & $p_0 \vee p_1$ \\ 
		\midrule
		0 & 0 &  0 \\
		0 & 1 &  1 \\
		1 & 0 &  1 \\
		1 & 1 &  1
		\end{tabular}
		\end{center}	
		
	In another DNF: $p_0 \vee p_1 = (\neg p_0 \wedge p_1) \vee (p_0 \wedge \neg p_1) \vee (p_0 \wedge p_1)$

	\item $\neg (p_0 \vee p_1)$: to turn this formula into DNF, first we use the De Morgan rule 
	\begin{eqnarray*}
		\neg (A \vee B) = \neg A \wedge \neg B.
	\end{eqnarray*}		
	
	We can easily see that $\neg (p_0 \vee p_1) = \neg p_0 \wedge \neg p_1$
	
		\begin{center}
		\begin{tabular}{c|c||c}
		\toprule
		$p_0$ & $p_1$ & $\neg (p_0 \vee p_1) = \neg p_0 \wedge \neg p_1$ \\ 
		\midrule
		0 & 0 &  1 \\
		0 & 1 &  0 \\
		1 & 0 &  0 \\
		1 & 1 &  0
		\end{tabular}
		\end{center}	
		
		In DNF: $(\neg p_0 \wedge p_1)$
		
		\item $(p_0 \vee p_1) \wedge (\neg p_0 \vee p_2)$:
		
		\begin{center}
		\begin{tabular}{c|c|c||c|c|c}
		\toprule
		$p_0$ & $p_1$ & $p_2$ & $(p_0 \vee p_1)$ & $(\neg p_0 \vee p_2)$ & $(p_0 \vee p_1) \wedge (\neg p_0 \vee p_2)$ \\ 
		\midrule
		0 & 0 &  0 & 0 & 1 & 0 \\
		0 & 0 &  1 & 0 & 1 & 0 \\
		0 & 1 &  0 & 1 & 1 & 1 \\
		0 & 1 &  1 & 1 & 1 & 1 \\
		1 & 0 &  0 & 1 & 0 & 0 \\
		1 & 0 &  1 & 1 & 1 & 1 \\
		1 & 1 &  0 & 1 & 0 & 0 \\		
		1 & 1 &  1 & 1 & 1 & 1
		\end{tabular}
		\end{center}
		In DNF: $(\neg p_0 \wedge p_1 \wedge \neg p_2) \vee (\neg p_0 \wedge p_1 \wedge p_2) \vee (p_0 \wedge \neg p_1 \wedge p_2) \vee (p_0 \wedge p_1 \wedge p_2)$
		
	\item $(\neg p_0 \vee p_1 \vee p_2) \wedge (p_0 \vee \neg p_1 \vee p_3) \wedge (\neg p_0 \vee p_2 \vee p_3) $
	
	\begin{eqnarray*}
	(\neg p_0 \vee p_1 \vee p_2) \wedge (p_0 \vee \neg p_1 \vee p_3) \wedge (\neg p_0 \vee p_2 \vee p_3) 
	= \\
	\neg (\neg (\neg p_0 \vee p_1 \vee p_2) 
	\vee 
	\neg (p_0 \vee \neg p_1 \vee p_3) 
	\vee 
	\neg (\neg p_0 \vee p_2 \vee p_3)) = \\
	\neg ((p_0 \wedge \neg p_1 \wedge \neg p_2) 
	\vee 
	(\neg p_0 \wedge p_1 \wedge \neg p_3) 
	\vee
	( p_0 \wedge \neg p_2 \wedge \neg p_3)) = \\
	\neg ((p_0 \neg p_1 \neg p_2) 
	+ 
	(\neg p_0 p_1 \neg p_3) 
	+
	( p_0 \neg p_2 \neg p_3)) = \\
	\neg (p_0 [(\neg p_1 \neg p_2) + (\neg p_2 \neg p_3)] + (\neg p_0 p_1 \neg p_3)) =	
	\end{eqnarray*}
	\end{enumerate}

\section{Problem 6}


\lastpage

\end{document}